t\chaplbl{Teacher's Guide}{sec:guide}

\seclbl{The Rules}{sec:rules}

\begin{enumerate}

\item Be kind: all else is details.

\item Never teach alone.

\item No lesson survives first contact with learners.

\item Nobody will be more excited about the lesson than you are.

\item Every lesson is too short from the teacher's point of view and too long from the learner's.

\item Never hesitate to sacrifice truth for clarify.

\item Every mistake is a lesson.

\item ``I learned this a long time ago'' is not the same as ``this is easy''.

\item Ninety percent of magic consists of knowing one extra thing.

\item You can't help everyone, but you can always help someone.

\end{enumerate}

\seclbl{Checklists}{sec:checklists}

See Atul Gawande's 2007 article
``\href{http://www.newyorker.com/magazine/2007/12/10/the-checklist}{The
Checklist}'' for a look at how using checklists can save lives (and make
many other things better too).

\begin{checklist}{Scheduling the Event}{check:scheduling-the-event}

\begin{enumerate}
\item
  Decide if it will be in person, online for one site, or online for
  several.
\item
  Talk through expectations with the host(s).

  \begin{itemize}
    \item
    If it is in person, make sure the host knows they're covering travel
    costs for trainers.
  \item
    Determine who is allowed to attend.

    \begin{itemize}
        \item
      We strongly prefer trainees to have attended workshops (as
      learners or helpers).
    \item
      Other criteria may be negotiated by the Executive Directors as
      part of partnership agreements.
    \end{itemize}
  \end{itemize}

\item
  Arrange trainers.

\item
  Arrange space.

  \begin{itemize}
    \item
    Make sure there are breakout rooms for video recording.
  \end{itemize}

\item
  Choose dates.

  \begin{itemize}
    \item
    If it is in person, book travel.
  \end{itemize}

\item
  Get names and email addresses of attendees from host(s).

  \begin{itemize}
    \item
    Register those people in AMY. \{: .checklist\}
  \end{itemize}
\end{enumerate}
\end{checklist}

\begin{checklist}{Setting Up}{check:setting-up}

\begin{enumerate}
\item
  Set up a one-page website for the workshop using
  \href{https://github.com/swcarpentry/training-template} as a starting
  point.
\item
  Send the URL to the admins.
\item
  Check whether any attendees have special needs.
\item
  If it is online:

  \begin{itemize}
    \item
    Test the video conference link.
  \end{itemize}
\item
  Make sure attendees will all have network access.
\item
  Create an Etherpad.
\item
  Email attendees a welcome message that includes:

  \begin{itemize}
    \item
    a link to the workshop home page
  \item
    background readings
  \item
    a description of any pre-requisite tasks \{: .checklist\}
  \end{itemize}
\end{enumerate}
\end{checklist}

\begin{checklist}{During the Event}{check:during-the-event}

\begin{enumerate}
\item
  Remind everyone of the code of conduct.
\item
  Collect attendance.
\item
  Distribute sticky notes.
\item
  Collect participants' GitHub IDs (if they are interested in teaching
  Software Carpentry).
\item
  Go through
  the checkout procedure point by point.
\item
  Explain how we
  format lesson submissions. \{: .checklist\}
\end{enumerate}
\end{checklist}

\begin{checklist}{After the Event}{check:after-the-event}

\begin{enumerate}
\item
  Update AMY.

  \begin{enumerate}
    \item
    Go to More\ldots{} Trainees.
  \item
    Select the training event in the filter on the left of the page.
  \item
    Tick off all the people who participated (there's a ``select all''
    tick box by the ``Name'' column header).
  \item
    Click ``Add'' at the bottom of the page. If anyone didn't show, or
    in your opinion didn't participate, do not give them credit for this
    training.
  \end{enumerate}
\item
  Administer the post-training survey.
\item
  Email attendees about
  the checkout process.
\item
  Debrief with the head of instructor training.
\item
  Oversee final demonstrations and mark them as complete in AMY.

  \begin{enumerate}
    \item
    Go to More\ldots{} Trainees.
  \item
    Enter the person's name in the filter on the left of the page and
    submit.
  \item
    Click the gray \texttt{{[}+{]}} beside their name.
  \item
    Fill in the details of the demo they did and who passed it.
  \item
    Submit. \{: .checklist\}
  \end{enumerate}
\end{enumerate}
\end{checklist}

\begin{checklist}{Between Instructor Training Sessions}{check:between-instructor-training-sessions}

\begin{enumerate}
\item
  Sign up to lead group lesson discussions.
\item
  Monitor \texttt{lessons@software-carpentry.org} for notices from
  trainees that they have submitted PRs and add those PRs to their
  progress on AMY.

  \begin{enumerate}
    \item
    Go to More\ldots{} Trainees.
  \item
    Enter the person's name in the filter on the left of the page and
    submit.
  \item
    Click the gray \texttt{{[}+{]}} beside their name.
  \item
    Select the type of homework (SWC or DC) and enter the URL of their
    PR or issue.
  \item
    Submit. \{: .checklist\}
  \end{enumerate}
\end{enumerate}
\end{checklist}

\begin{checklist}{After Trainees Complete}{check:after-trainees-complete}

\begin{enumerate}
\item
  Send new instructors
  a completion message.
\item
  Badge instructors.

  \begin{enumerate}
    \item
    Go to More\ldots{} Trainees.
  \item
    Enter the person's name in the filter on the left of the page and
    submit.
  \item
    Select the appropriate badge on the right.
  \item
    Fill in the details.
  \item
    Submit.
  \end{enumerate}
\item
  Create and send PDF certificates. \{: .checklist\}
\end{enumerate}
\end{checklist}

Note that trainers do not examine their own trainees: having them
examine each other's helps balance load and maintain consistency of
curriculum and standards.

\seclbl{Messages}{sec:messages}

You may use the following message templates to communicate with
trainees:

\begin{itemize}
\item
  Welcome participants before the workshop.
\item
  Description of checkout procedure sent immediately after the workshop.
\item
  Next steps after receiving pull request.
\item
  Request to do another discussion session when participant was passive.
\item
  Confirm certification and describe next steps.
\item
  Notify participants that they have not completed within the specified time.
\end{itemize}

\seclbl{Introduction}{sec:introduction}

To begin your class, the instructors should give a brief introduction
that will convey their capacity to teach the material,
accessibility/approachability, desire for student success, and
enthusiasm. Tailor your introduction to the students' skill level so
that you convey competence (without seeming too advanced) and
demonstrate that you can relate to the students. Throughout the
workshop, continually demonstrate that you are interested in student
progress and that you are enthusiastic about the topics.

Students should also introduce themselves (preferably verbally). At the
very least, everyone should add their name to the Etherpad, but its also
good for everyone at a given site to know who all is in the group. Note:
this can be done while setting up before the start of the class.

\seclbl{Exercises}{sec:exercises}

\begin{itemize}
\item
  Have students write answers to the initial MCQ in the Etherpad or
  create a copy of this
  \href{http://goo.gl/forms/EHXfBSDmvqBLLVzj1}{Google Form
  Questionnaire}. Briefly summarize the answers.
\item
  Learners do think-pair-share for cognitive maps and multiple-choice
  questions.
\item
  In the two-day versions, have learners read the operations guide as
  their overnight homework and do their demotivational story just before
  lunch on day 2: it means day 2 starts with \emph{their} questions
  (which wakes them up), and the demotivational story is a good lead-in
  to lunchtime discussion.
\item
  Don't have them complete the Teaching Perspectives Inventory or read
  through the pre- or post-assessment questionnaires in class: it kills
  momentum.
\item
  Have them work in pairs for the live coding exercise rather than
  threes, and don't bother recording: the camera can't pick up both the
  speaker's body language and what's on the screen. Afterwards, have
  learners put answers to the following questions in the Etherpad:

  \begin{itemize}
    \item
    What felt different about live coding (vs.~standing up and
    lecturing)? What was harder/easier?
  \item
    Did you make any mistakes? If so, how did you handle them?
  \item
    Did you talk and type at the same time, or alternate?
  \item
    How often did you point at the screen? How often did you highlight
    with the mouse?
  \item
    What will you try to do differently next time?
  \end{itemize}
\item
  When teaching faded examples:

  \begin{enumerate}
      \item
    Divide trainees into groups of 4-5
  \item
    Put the faded example (Python code) from the curriculum into the
    Etherpad.
  \item
    Define the audience for these examples. For example, are these
    beginners who only know some basics programming concepts? Or are
    these learners with some experience in programming but not in
    Python? You need to provide this information so that the trainees
    can decide what intrinsic knowledge these audiences have.
  \item
    Talk the trainees through some examples of intrinsic, germane, and
    extraneous knowledge for the Python faded example for the audiences
    you defined. For example:

    \begin{itemize}
        \item
      Intrinsic load: ability to read and write; understanding what
      variable is
    \item
      Germane load: learning how to loop through a collection in Python
    \item
      Extraneous load: need to know that Python requires indentation
    \end{itemize}
  \item
    Ask the trainees to discuss and identify in groups what different
    loads occur, and write them into the Etherpad.
  \end{enumerate}

  If there are people among the trainees who don't program at all, make
  sure that they are in separate groups and ask to the groups to work
  with that person as a learner to help identify different loads.
  Another option is to have a faded example that is not programming
  specific. But that may be difficult to achieve.
\end{itemize}

\seclbl{Video Recorded Lessons}{sec:video-recorded-lessons}

One of the key elements of this training course is recording trainees
and having them, and their peers, critique those recordings. We were
introduced to this practice by UBC's Warren Code, and it has evolved to
the following:

\begin{enumerate}
\item
  On day 1, show trainees a short clip (3-4 minutes) of someone teaching
  a lesson and have them give feedback as a group. This feedback is
  organized on two axes: positive versus negative, and content versus
  presentation. The first axis is explained as ``things to be repeated
  and emphasized'' versus ``things to be improved'', while the second is
  explained by contrasting people who have good ideas, but can't
  communicate them (all content, no presentation) with people who speak
  well, but don't actually have anything to say.
\item
  Trainees are then asked to work in groups of three. Each person
  rotates through the roles of instructor, audience, and videographer.
  As the instructor, they have two minutes to explain one key idea from
  their research (or other work) as if they were talking to a class of
  interested high school students. The person pretending to be the
  audience is there to be attentive, while the videographer records the
  session using a cellphone or similar device.
\item
  After everyone has taught, the trio sits together and watches all
  three videos in succession, writing out feedback on the same 2x2 grid
  introduced above. Once all the videos have been reviewed, the group
  rejoins the class; each person puts all the feedback on themselves
  into the Etherpad.
\end{enumerate}

In order for this exercise to work well:

\begin{itemize}
\item
  Groups must be physically separated to reduce audio cross-talk between
  their recordings. In practice, this means 2-3 groups in a normal-sized
  classroom, with the rest using nearby breakout spaces, coffee lounges,
  offices, or (on one occasion) a janitor's storage closet.
\item
  Do all three recordings before reviewing any of them, because
  otherwise the person to go last is short-changed on time.
\item
  People must give feedback on themselves, as well as giving feedback on
  each other, so that they can calibrate their impressions of their own
  teaching according to the impressions of other people. (We find that
  most people are harder on themselves than others are, and it's
  important for them to realize this.)
\item
  At the end of day 1, ask trainees to review the lesson episode you
  will use for the live coding demonstration at the start of day 2.
\item
  Try to make at least one mistake during the demonstration of live
  coding so that trainees can see you talk through diagnosis and
  recovery, and draw attention afterward to the fact that you did this.
\end{itemize}

The announcement of this exercise is often greeted with groans and
apprehension, since few people enjoy seeing or hearing themselves.
However, it is consistently rated as one of the most valuable parts of
the class, and also serves as an ice breaker: we want pairs of
instructors at actual workshops to give one another feedback, and that's
much easier to do once they've had some practice and have a rubric to
follow.

\seclbl{Live Coding Demo Videos}{sec:live-coding-demo-videos}

\textbf{\href{https://youtu.be/bXxBeNkKmJE}{Part 1}: how not to do it}

\begin{itemize}
\item
  Instructor ignores a red sticky clearly visible on a learner's laptop.
\item
  Instructor is sitting, mostly looking at the laptop screen.
\item
  Instructor is typing commands without saying them out loud.
\item
  Instructor uses fancy bash prompt.
\item
  Instructor uses small font in not full-screen terminal window with
  black background.
\item
  The terminal window bottom is partially blocked by the learner's heads
  for those sitting in the back.
\item
  Instructor receives a a pop-up notification in the middle of the
  session.
\item
  Instructor makes a mistake (a typo) but simply fixes it without
  pointing it out, and redoes the command.
\end{itemize}

\textbf{\href{https://youtu.be/SkPmwe\_WjeY}{Part 2}: how to do it right}

\begin{itemize}
\item
  Instructor checks if the learner with the red sticky on her laptop
  still needs attention.
\item
  Instructor is standing while instructing, making eye-contact with
  participants.
\item
  Instructor is saying the commands out loud while typing them.
\item
  Instructor moves to the screen to point out details of commands or
  results.
\item
  Instructor simply uses `\$ ` as bash prompt.
\item
  Instructor uses big font in wide-screen terminal window with white
  background.
\item
  The terminal window bottom is above the learner's heads for those
  sitting in the back.
\item
  Instructor makes mistake (a typo) and uses the occasion to illustrate
  how to interpret error-messages.
\end{itemize}

\seclbl{Motivation and Demotivation}{sec:motivation-and-demotivation}

\begin{itemize}
\item
  In the exercise on brainstorming motivational challenges, review the
  comments in the Etherpad. Rather than read all out loud, highlight the
  common themes (i.e.~establish value, positive expectations, promote
  self efficiency) or things that stand out our that you can relate to.
  Note: this exercise can be done before or going through the above
  list.
\item
  In the exercise on brainstorming demotivational experiences, review
  the comments in the Etherpad. Rather than read all out loud, highlight
  a few of the things that could have been done differently. This will
  give everyone some confidence in how to handle these situations in the
  future.
\end{itemize}

\seclbl{The Big Picture}{sec:the-big-picture}

In 2014,
\href{http://www.theguardian.com/commentisfree/2014/jun/16/saving-the-world-promise-not-fear-nature-environmentalism}{George
Monbiot wrote}:

\begin{quote}
If we had set out to alienate and antagonize the people we've been
trying to reach, we could scarcely have done it better. This is how I
feel, looking back on the past few decades of environmental campaigning,
including my own\ldots{}

Experimental work suggests that when fears are whipped up, they trigger
an instinctive survival response. You suppress your concern for other
people and focus on your own interests\ldots{} Terrify the living
daylights out of people, and they will protect themselves at the expense
of others\ldots{}
\end{quote}

A lot of advocates for open science and reproducible research make the
same mistake. They frighten people with talk of papers that have been
retracted when they should talk about all the new science people could
do if they weren't wasting hours trying to figure out how they created
figure number three in the first place.

We have found that we have more impact when we \emph{emphasize how much
more researchers can do when they are computationally competent}. We
have also found it's importance for us to \emph{emphasize that what we
teach and how we teach it is based on the best available evidence}. We
use live coding instead of slides because research shows that people
learn more from doing than watching. Similarly, the tools we teach are
ones that our instructors---who are active researchers themselves---use
daily.

One final point to make in instructor training workshops is that
\emph{our greatest impact may be what we teach our instructors about
teaching and collaborating}. As a species, we know as much about
education as we do about public health, but since most university
lecturers are self-taught teachers, they are completely unaware of this
body of knowledge. At the same time, the massive, open collaboration
that has made Wikipedia and open source software successful has never
taken hold in teaching. Most university lecturers are still the sole
creators and consumers of their lessons, which wastes time and impedes
the spread of good ideas. Changing \emph{that} could have more impact in
the long run than anything to do with for loops and pull requests.

\seclbl{You Are Not Your Learners}{sec:you-are-not-your-learners}

Discussion of the practical implications of learning concepts brings us
to our next big idea: people learn best when they care about the topic
and believe they can master it. Neither fact is particularly surprising,
but their practical implications have a lot of impact on what we teach,
and the order in which we teach it.

First, most scientists don't actually want to program. They want to do
scientific research, and programming is just a tax they have to pay
along the way. They don't care how hash tables work, or even that hash
tables exist; they just want to know how to process data faster. We
therefore have to make sure that everything we teach is useful right
away, and conversely that we don't teach anything just because it's
``fundamental''.

Second, believing that something will be hard to learn is a
self-fulfilling prophecy. This is why it's important not to say that
something is easy: if someone who has been told that tries it, and it
doesn't work, they are more likely to become discouraged.

It's also why installing and configuring software is a much bigger
problem for us than experienced programmers like to acknowledge. It
isn't just the time we lose at the start of boot camps as we try to get
a Unix shell working on Windows, or set up a version control client on
some idiosyncratic Linux distribution. It isn't even the unfairness of
asking students to debug things that depend on precisely the knowledge
they have come to learn, but which they don't yet have. The real problem
is that every such failure reinforces the belief that computing is hard,
and that they'd have a better chance of making next Thursday's
conference submission deadline if they kept doing things the way they
always have. For these reasons, we have adopted a ``teach most
immediately useful first'' approach described in \secref{sec:motivation}.

\begin{callout}{Software Carpentry Is Not Computer Science}{callout:software-carpentry-is-not-computer-science}

Many of the foundational concepts of computer science, such as
computability, inhabit the lower-right corner of the grid described
above. This does \emph{not} mean that they aren't important, or aren't
worth learning, but if our aim is to convince people that they can learn
this stuff, and that doing so will help them do more science faster,
they are less compelling than things like automating repetitive tasks.
\end{callout}

\seclbl{Logistics}{sec:logistics}

This course has been taught as a multi-week online class, as a two-day
in-person class, and as a two-day class in which the learners are in
co-located groups and the instructor participates remotely.

\seclbl{Two-Day In-Person (Currently used)}{sec:two-day-in-person-currently-used}

This was the second method we tried. The biggest change was the
introduction of recorded teaching exercises.

\begin{itemize}
\item
  Several times during the training, participants are divided into
  groups of three and asked to teach a short lesson (typically 2-3
  minutes long). In turn, one person is the teacher, one the audience,
  and one the videographer, who records the teacher using a handheld
  device such as a phone. Group members then rotate roles: the teacher
  becomes the listener, the listener records, and the videographer
  teaches. Once all three have finished teaching, the group reviews all
  three videos, and everyone gives feedback on everyone (including
  themselves). This feedback then goes into the Etherpad for discussion.
\item
  It's important to record all three videos and then watch all three: if
  the cycle is teach-review-teach-review, the last person to teach runs
  out of time. Doing all the reviewing after all the teaching also helps
  put a bit of distance between the teaching and the reviewing, which
  makes the exercise slightly less excruciating.
\item
  This exercise only works if there are breakout rooms available: if
  everyone is trying to record in the same room, the audio cross-talk
  makes the recordings unintelligible.
\item
  We use Etherpad for in-person training, both for note-taking and for
  posting exercise solutions and feedback on recorded lessons Questions
  and discussion are done aloud.
\end{itemize}

\seclbl{Two-Day Online With Groups (Currently used)}{sec:two-day-online-with-groups-currently-used}

\begin{itemize}
\item
  We use Google Hangouts and Etherpad as in the multi-week version. Each
  group of learners is together in a room using one camera and
  microphone, rather than each being on the call separately. We have
  found that having good audio matters more than having good video, and
  that the better the audio, the more learners can communicate with the
  instructor and other rooms by voice rather than by using the Etherpad
  chat.
\item
  We do the video lecture exercise as in the two-day in-person training.
\end{itemize}

\seclbl{Multi-Week Online (Retired)}{sec:multi-week-online-retired}

This was the first method we tried.

\begin{itemize}
\item
  We meet every week or every second week for an hour using Google
  Hangout or BlueJeans. Each meeting is held twice (or even three times)
  to accommodate learners' time zones and because video conferencing
  systems can't handle 60+ people at once. Each meeting also uses an
  Etherpad for shared note-taking, and more importantly for asking and
  answering questions: having several dozen people try to talk on a call
  hasn't worked, so in most sessions, the instructor does the talking
  and learners respond through the Etherpad chat.
\item
  Learners post homework online, then comment on each other's work.

  \begin{itemize}
    \item
    We used a WordPress blog for the first ten rounds of training.
    People found writing and commenting on posts straightforward, but
    setting up dozens of logins was tedious.
  \item
    We tried a GitHub-backed blog in the Winter 2015 class. It didn't
    work nearly as well: a third of the participants found it extremely
    frustrating, and post-publication commentary was awkward.
  \item
    We tried Piazza in the Fall 2015 class. It was better than GitHub,
    but still not as good as a simple WordPress blog. In particular, it
    was hard to find things once there were more than a dozen homework
    categories.
  \end{itemize}
\end{itemize}

\seclbl{Demo Sessions}{sec:demo-sessions}

Checklist for instructor trainers hosting a live-coding demo session as
part of a trainee's checkout procedure.

\subsubsection{Before the Demo}\label{before-the-demo}

\begin{itemize}
\item
  Sign up to lead the demo at
  \href{http://pad.software-carpentry.org/teaching-demos}.
\item
  For each trainee, pick a suitable starting point in the lesson that
  they have chosen. Do not start at the very beginning of the first
  episode, and look for an episode that dives into live coding quickly
  without first explaining a lot. (Example starting points are listed
  below.). Do not have them start in the middle of an episode. Note that
  some lessons (e.g., the Software Carpentry R lesson using inflammation
  data) have supplementary episodes. Do not pick from those.
\item
  If a trainee wants to demo for both SWC and DC, allocate two slots for
  them, but set up schedule that does \textbf{not} have them teaching
  twice in a row.
\end{itemize}

\subsubsection{Shortly Before the Demo}\label{shortly-before-the-demo}

\begin{itemize}
\item
  Prepare the Google hangout (or other online meeting place) and paste
  the link in the etherpad.
\end{itemize}

\subsubsection{During the Demo}\label{during-the-demo}

\begin{itemize}
\item
  Once everyone is in the call (audio and video working), remind them of
  the Code of Conduct, explain the procedure for the demo session, and
  remind them that trainees have to be able to teach from \emph{any}
  episode from their chosen lesson. Ask whether anyone has only prepared
  for 5 minutes from \emph{one} episode instead of the entire lesson,
  and if so, suggest strongly they reschedule.
\item
  Ask those not presenting to mute their microphone, and tell them they
  are to give feedback in the etherpad using the same
  positive-vs-negative and content-vs-presentation rubric used in
  training.
\item
  Hand out the assignment to the first trainee, give them a bit of time
  to set up the demo (they may have to import some packages, load some
  data, move to a certain folder etc).
\item
  Ask them to share their screen.
\item
  Once they are ready, give them a 3-2-1 countdown to start.
\item
  Use a countdown timer which makes a noise once their 5 minutes are up
  (e.g., your phone), or just say ``bong'' really loudly at the end of
  their tie.
\item
  After the five minute timer, allow them to finish their sentence and
  tell them time's up.
\item
  Use a rubric for notes.
\item
  After the trainee is finished, first ask how they themselves thought
  it went, then give constructive feedback based on your notes.
\item
  Do \emph{not} tell the trainee whether they passed or failed: send
  that by email after the session is over.
\item
  Repeat for the other trainees.
\item
  At the end of the season, ask for general questions.
\item
  Tell trainees that once disconnected, you will add your notes to the
  session's etherpad and leave them there for a while.
\item
  Disconnect and add your notes to the session's etherpad.
\end{itemize}

\subsubsection{After the Demo}\label{after-the-demo}

\begin{itemize}
\item
  Sign up for your next session in
  \href{http://pad.software-carpentry.org/teaching-demos}.
\item
  Add pass/fail information for the trainees in AMY.
\end{itemize}

\seclbl{Terminology}{sec:terminology}

\begin{discussion}{Something to Think About}{disc:something-to-think-about}

Throughout the day, take note of how this training is structured. What
pieces exemplify the situated learning perspective, i.e., how are you,
as an instructor-in-training, being brought into a new community of
practice? Are there any places where we are using the cognitivist
ideas/techniques described later in the training?
\end{discussion}

\begin{challenge}{Who Decides?}{chal:who-decides}

In Littky and Grabelle's
\emph{The Big Picture: Education is Everyone's Business} \cite{bib:littky-big-picture},
Kenneth Wesson wrote,
``If poor inner-city children consistently outscored children from
wealthy suburban homes on standardized tests, is anyone naive enough to
believe that we would still insist on using these tests as indicators of
success?'' What are examples in your own experience of ``objective''
assessments that reinforce the status quo?
\end{challenge}

\seclbl{Improvement}{sec:improvement}

This training course only a start. If you'd like to help us make it
better, we would welcome additions discussing:

\begin{itemize}
\item
  how education research is done: qualitative studies, quantitative
  studies, and comparison studies
\item
  Dewey, Piaget, Vygotsky, Freire (or, key figures in 100 words each)
\item
  the history of distance education (or, everything old is new again)
\item
  MOOCs
\item
  computer-based homework/teaching systems
\item
  other forms of assessment
\item
  an overview of research on novice programmers
\item
  problem-solving skills overview
\item
  setting expectations in the classroom
\item
  promoting effective study habits
\end{itemize}

We would also appreciate additions to this list of things we
\emph{don't} do, and explanations of why not:

\begin{description}
\item[peer instruction]
This powerful teaching method has been proven effective, but we are
already asking workshop participants to assimilate a lot of new things,
and picking up a new learning technique while learning the basics of
coding and data wrangling seems too much to ask.
\item[certification]
Many people have asked us to certify workshop participants in the same
way that we certify instructors, but any meaningful certification
process would require a lot of resources to set up and run.
\end{description}

\seclbl{Effecting Change}{sec:effecting-change}

Henderson et al's
``Facilitating Change in Undergraduate STEM Instructional Practices''
\cite{bib:henderson-facilitating}
discusses ways
to get educational institutions to actually change what they teach.
Their findings are summarized in this table:

\begin{tabular}{lll}

\textbf{Aspect of System to be Changed}
&
\textbf{Intended Outcome}
&
\\

\textbf{Individuals}
&
\textbf{Prescribed}
&
  \textbf{I. Disseminating: Curriculum \& Pedagogy}
  \newline
  Change Agent Role: tell/teach individuals about new teaching conceptions and/or practices and encourage their use.
  \newline
  \emph{Diffusion}
  \newline
  \emph{Implementation}
\\

&
\textbf{Emergent}
&
  \textbf{II. Developing: Reflective Teachers}
  \newline
  Change Agent Role: encourage/support individuals to develop new teaching conceptions and/or practices.
  \newline
  \emph{Scholarly Teaching}
  \newline
  \emph{Faculty Learning Communities}
\\

\textbf{Environments and Structures}
&
\textbf{Prescribed}
&
  \textbf{III. Enacting: Policy}
  \newline
  Change Agent Role: enact new environmental features that require/encourage new teaching conceptions and/or practices.
  \newline
  \emph{Quality Assurance}
  \newline
  \emph{Organizational Development}
\\

&
\textbf{Emergent}
&
  \textbf{IV. Developing: Shared Vision}
  \newline
  Change Agent Role: empower/support stakeholders to collectively develop new environmental features that encourage new teaching conceptions and/or practices.
  \newline
  \emph{Learning Organizations}
  \newline
  \emph{Complexity Leadership}
\\
\end{tabular}

The eight italicized approaches are:

\begin{itemize}
\item
  \emph{Diffusion}: STEM undergraduate instruction will be changed by
  altering the behavior of a large number of individual instructors. The
  greatest influences for changing instructor behavior lie in optimizing
  characteristics of the innovation and exploiting the characteristics
  of individuals and their networks.
\item
  \emph{Implementation}: STEM undergraduate instruction will be changed
  by developing research-based instructional ``best practices'' and
  training instructors to use them. Instructors must use these practices
  with fidelity to the established standard.
\item
  \emph{Scholarly Teaching}: STEM undergraduate instruction will be
  changed when more individual faculty members treat their teaching as a
  scholarly activity.
\item
  \emph{Faculty Learning Communities}: STEM undergraduate instruction
  will be changed by groups of instructors who support and sustain each
  other's interest, learning, and reflection on their teaching.
\item
  \emph{Quality Assurance}: STEM undergraduate instruction will be
  changed by requiring institutions (colleges, schools, departments, and
  degree programs) to collect evidence demonstrating their success in
  undergraduate instruction. What gets measured is what gets improved.
\item
  \emph{Organizational Development}: STEM undergraduate instruction will
  be changed by administrators with strong vision who can develop
  structures and motivate faculty to adopt improved instructional
  practices.
\item
  \emph{Learning Organizations}: Innovation in higher education STEM
  instruction will occur through informal communities of practice within
  formal organizations in which individuals develop new organizational
  knowledge through sharing implicit knowledge about their teaching.
  Leaders cultivate conditions for both formal and informal communities
  to form and thrive.
\item
  \emph{Complexity Leadership}: STEM undergraduate instruction is
  governed by a complex system. Innovation will occur through the
  collective action of self-organizing groups within the system. This
  collective action can be stimulated, but not controlled.
\end{itemize}

\seclbl{Why Do(n't) We Teach X?}{sec:why-dont-we-teach-x}

Workshop attendees and trainee instructors often ask why we don't teach
high-performance computing, machine learning, Perl, or a long list of
other topics. Our answer is that as with every curriculum, the question
is not, ``What would we like to add?'' but, ``What are we willing to
take out in order to make room?'' We believe our core topics are the
absolute minimum that researchers need to know in order to work
efficiently and reproducibly. More importantly, we don't know what we
could take out to make space for something else.

One thing we \emph{do} know is that we do not wish to become embroiled
in debates over the relative merits of different languages or operating
systems. No one has ever demonstrated that R programmers are more
productive than Python programmers, and proficient users of Windows seem
just as productive as equally-proficient users of Unix. If a learner
asserts that their favorite tool is better than alternatives in some
way, ask them for their data; if they don't have any, point out as
gently as possible that we're supposed to be scientists, and that if we
want politicians, business leaders, and the general public to pay
attention to our findings on climate change and drug-resistant diseases,
it behooves us to try to meet those same standards ourselves.

\begin{callout}{Evidence and Its Absence}{callout:evidence-and-its-absence}

As far as is practical, our teaching methods are based on the best
available evidence. We wish we could say the same about our content, but
very little research has been done on what researchers actually use and
what impact it has on productivity. An example of what we wish existed
is the summary by Stefik et al of empirical research on the usability of
programming languages \cite{bib:stefik-summary}
(while
their  full-length paper gives an idea of what's possible \cite{bib:stefik-fixme}).
\end{callout}

\seclbl{Why We're Not a MOOC}{sec:why-were-not-a-mooc}

\begin{quote}
If you use robots to teach, you teach people to be robots.
\end{quote}

This difference between what novices are doing when they learn, and what
competent practitioners are doing, is one of the reasons we have stopped
trying to teach via recorded video with auto-graded drill exercises. Any
recorded content is as ineffective for most learners as broadcast
television, or as a professor standing in front of 400 people in a
lecture hall, because neither can intervene to clear up specific
learners' misconceptions. Some people happen to already have the right
conceptual categories for a subject, or happen to form them correctly
early on; these are the ones who stick with most massive online courses,
but many discussions of the effectiveness of such courses ignore this
survivor bias.

\seclbl{Program Assessment}{sec:program-assessment}

The Carpentries' greatest weakness is a lack of systematic assessment:
while we have done some small-scale studies of the impact we have on our
learners, and Dr.~Beth Duckles' studies of
\href{http://software-carpentry.org/files/bib/duckles-instructor-engagement-2016.pdf}{why
instructors join us} and
\href{http://software-carpentry.org/files/bib/duckles-non-instructor-report-2016.pdf}{why
people qualify but then don't teach} are very insightful, we still don't
know what learners actually adopt or what effect it has on their
productivity, the reproducibility of their work, and so on.

We have sometimes used this as the basis for an in-class exercise.
Working in groups of four, trainees brainstorm answers to the following:
``Your dean has provisionally agreed to set aside funds to support some
Carpentry workshops over the next year, but wants to know how you will
tell at the end of those workshops whether the money was worth spending.
Given the resources you have, what information can you collect, how
would you analyze it, and why do you think it would be convincing?''
Each group then presents its best idea, which the trainers and other
trainees critique.

This exercise always generates a lot of discussion, but end-of-day
assessment has usually indicated that trainees don't find it
particularly useful. We have therefore cut it, but may re-introduce it
if and when we include a module on program assessment.
