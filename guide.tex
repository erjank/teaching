\chaplbl{Instructor's Guide}{sec:guide}

This appendix includes material that doesn't naturally fit anywhere else.

\seclbl{Further Reading}{sec:guide-reading}

\begin{gitemize}

\item
  Ambrose et al's \emph{How Learning Works} \cite{bib:ambrose-hlw} is
  the best overview of what we know about teaching and learning and
  why we believe it's true.  It's also a great example of secondary
  literature: every topic gets a few pages, with pointers to the
  primary literature for those who want to dive in further.

\item
  Brookfield and Preskill's \emph{The Discussion Book}
  \cite{bib:preskill-discussion} is a catalog of ways to get groups
  (including classes) talking to one another productively.

\item
  Green's \emph{Building a Better Teacher} \cite{bib:green-babt} looks
  at how attempts to reform public education in the United States (and
  other English-speaking countries) have foundered over the last forty
  years because they're focusing on the wrong things.

\item
  Guzdial's \emph{Learner-Centered Design of Computing Education}
  \cite{bib:guzdial-lcd} is an evidence-based argument that we must
  design computing education for everyone, not just people who think
  they are going to become full-time professional programmers, and
  draws material from the author's excellent blog \cite{bib:guzdial-blog}.

\item
  Huston's \emph{Teaching What You Don't Know}
  \cite{bib:huston-dont-know} is a good evidence-based guide to doing
  exactly that.

\item
  Lang's \emph{Small Teaching} \cite{bib:lang-small-teaching} is a
  short guide to evidence-based teaching practices that can be adopted
  without requiring large up-front investments of time and money.

\item
  Lemov's \emph{Teaching Like a Champion} \cite{bib:lemov-champion} is
  a catalog of good pedagogical practices drawn from observations of
  hundreds of teachers over thousands of hours.

\item
  Margolis and Fisher's \emph{Unlocking the Clubhouse}
  \cite{bib:margolis-fisher-clubhouse} and Margolis et al's
  \emph{Stuck in the Shallow End} \cite{bib:margolis-shallow} look at
  how women and some racial minorities are systematically pushed out
  of computing, and what can be done about it.

\end{gitemize}

\seclbl{Starting Out}{sec:guide-starting}

To begin your class, the instructors should give a brief introduction
that will convey their capacity to teach the material,
accessibility and approachability, desire for student success, and
enthusiasm. Tailor your introduction to the students' skill level so
that you convey competence (without seeming too advanced) and
demonstrate that you can relate to the students. Throughout the
workshop, continually demonstrate that you are interested in student
progress and that you are enthusiastic about the topics.

Students should also introduce themselves (preferably verbally). At the
very least, everyone should add their name to the Etherpad, but its also
good for everyone at a given site to know who all is in the group. Note:
this can be done while setting up before the start of the class.

\seclbl{You Are Not Your Learners}{sec:you-are-not-your-learners}

People learn best when they care about the topic and believe they can
master it. Neither fact is particularly surprising, but their
practical implications have a lot of impact on what we teach, and the
order in which we teach it.

First, as noted in \chapref{sec:motivation}, most people don't
actually want to program: they want to build a website or check on
zoning regulations, and programming is just a tax they have to pay
along the way. They don't care how hash tables work, or even that hash
tables exist; they just want to know how to process data faster. We
therefore have to make sure that everything we teach is useful right
away, and conversely that we don't teach anything just because it's
``fundamental''.

Second, believing that something will be hard to learn is a
self-fulfilling prophecy. This is why it's important not to say that
something is easy: if someone who has been told that tries it, and it
doesn't work, they are more likely to become discouraged.

It's also why installing and configuring software is a much bigger
problem for us than experienced programmers like to acknowledge. It
isn't just the time we lose at the start of boot camps as we try to
get a Unix shell working on Windows, or set up a version control
client on some idiosyncratic Linux distribution.

It isn't even the unfairness of asking students to debug things that
depend on precisely the knowledge they have come to learn, but which
they don't yet have. The real problem is that every such failure
reinforces the belief that computing is hard, and that they'd have a
better chance of making next Thursday's deadline at work if they kept
doing things the way they always have. For these reasons, we have
adopted a ``teach most immediately useful first'' approach described
in \secref{sec:motivation}.

\seclbl{Overnight Homework}{sec:homework}

In a two-day class, have learners read the operations checklists as
overnight homework and do their demotivational story just before
lunch on day 2: it means day 2 starts with \emph{their} questions
(which wakes them up), and the demotivational story is a good
lead-in to lunchtime discussion.

\seclbl{Video Recorded Lessons}{sec:video-recorded-lessons}

One of the key elements of this training course is recording trainees
and having them, and their peers, critique those recordings. We were
introduced to this practice by UBC's Warren Code, and it has evolved to
the following:

\begin{genumerate}

\item
  On day 1, show trainees a short clip (3-4 minutes) of someone teaching
  a lesson and have them give feedback as a group. This feedback is
  organized on two axes: positive versus negative, and content versus
  presentation. The first axis is explained as ``things to be repeated
  and emphasized'' versus ``things to be improved'', while the second is
  explained by contrasting people who have good ideas, but can't
  communicate them (all content, no presentation) with people who speak
  well, but don't actually have anything to say.

\item
  Trainees are then asked to work in groups of three. Each person
  rotates through the roles of instructor, audience, and videographer.
  As the instructor, they have two minutes to explain one key idea from
  their research (or other work) as if they were talking to a class of
  interested high school students. The person pretending to be the
  audience is there to be attentive, while the videographer records the
  session using a cellphone or similar device.

\item
  After everyone has taught, the trio sits together and watches all
  three videos in succession, writing out feedback on the same 2x2 grid
  introduced above. Once all the videos have been reviewed, the group
  rejoins the class; each person puts all the feedback on themselves
  into the shared notes.

\end{genumerate}

In order for this exercise to work well:

\begin{gitemize}

\item
  Groups must be physically separated to reduce audio cross-talk
  between their recordings. In practice, this means 2-3 groups in a
  normal-sized classroom, with the rest using nearby breakout spaces,
  coffee lounges, offices, or (on one occasion) a janitor's storage
  closet.

\item
  Do all three recordings before reviewing any of them, because
  otherwise the person to go last is short-changed on time.

\item
  People must give feedback on themselves, as well as giving feedback
  on each other, so that they can calibrate their impressions of their
  own teaching according to the impressions of other people. (We find
  that most people are harder on themselves than others are, and it's
  important for them to realize this.)

\item
  At the end of day 1, ask trainees to review the lesson episode you
  will use for the live coding demonstration at the start of day 2.

\item
  Try to make at least one mistake during the demonstration of live
  coding so that trainees can see you talk through diagnosis and
  recovery, and draw attention afterward to the fact that you did
  this.

\end{gitemize}

The announcement of this exercise is often greeted with groans and
apprehension, since few people enjoy seeing or hearing themselves.
However, it is consistently rated as one of the most valuable parts of
the class, and also serves as an ice breaker: we want pairs of
instructors at actual workshops to give one another feedback, and
that's much easier to do once they've had some practice and have a
rubric to follow.

\subseclbl{Feedback on Live Coding Demo Videos}{sec:live-coding-demo-videos}

\noindent
\textbf{\href{https://youtu.be/bXxBeNkKmJE}{Part 1}: how not to do it}

\begin{gitemize}

\item
  Instructor ignores a red sticky clearly visible on a learner's
  laptop.

\item
  Instructor is sitting, mostly looking at the laptop screen.

\item
  Instructor is typing commands without saying them out loud.

\item
  Instructor uses fancy bash prompt.

\item
  Instructor uses small font in not full-screen terminal window with
  black background.

\item
  The terminal window bottom is partially blocked by the learner's
  heads for those sitting in the back.

\item
  Instructor receives a a pop-up notification in the middle of the
  session.

\item
  Instructor makes a mistake (a typo) but simply fixes it without
  pointing it out, and redoes the command.

\end{gitemize}

\noindent
\textbf{\href{https://youtu.be/SkPmwe\_WjeY}{Part 2}: how to do it right}

\begin{gitemize}

\item
  Instructor checks if the learner with the red sticky on her laptop
  still needs attention.

\item
  Instructor is standing while instructing, making eye-contact with
  participants.

\item
  Instructor is saying the commands out loud while typing them.

\item
  Instructor moves to the screen to point out details of commands or
  results.

\item
  Instructor simply uses `\$ ` as bash prompt.

\item
  Instructor uses big font in wide-screen terminal window with white
  background.

\item
  The terminal window bottom is above the learner's heads for those
  sitting in the back.

\item
  Instructor makes mistake (a typo) and uses the occasion to
  illustrate how to interpret error-messages.

\end{gitemize}

\seclbl{Motivation and Demotivation}{sec:guide-motivation}

In the exercise on brainstorming demotivational experiences, review
the comments in the shared notes. Rather than read all out loud,
highlight a few of the things that could have been done
differently. This will give everyone some confidence in how to handle
these situations in the future.

\seclbl{Logistics}{sec:logistics}

This course has been taught as a multi-week online class, as a two-day
in-person class, and as a two-day class in which the learners are in
co-located groups and the instructor participates remotely.

\seclbl{Multi-Week Online}{sec:multi-week-online-retired}

This was the first format we used, and we no longer recommend it.

\begin{gitemize}

\item
  We met every week or every second week for an hour using Google
  Hangout or BlueJeans. Each meeting is held twice (or even three
  times) to accommodate learners' time zones and because video
  conferencing systems can't handle 60+ people at once.

\item
  Each meeting also uses an Etherpad or Google Doc for shared
  note-taking, and more importantly for asking and answering
  questions: having several dozen people try to talk on a call hasn't
  worked, so in most sessions, the instructor does the talking and
  learners respond through the Etherpad chat.

\item
  Learners post homework online between classes, and comment on each
  other's work.  In practice, it turned out to be hard to get people
  to comment in detail (or at all).

\item
  We used a WordPress blog for the first ten rounds of training.
  People found writing and commenting on posts straightforward, but
  setting up dozens of logins was tedious.

\item
  We tried a GitHub-backed blog in the Winter 2015 class. It didn't
  work nearly as well: a third of the participants found it extremely
  frustrating, and post-publication commentary was awkward.

\item
  We tried Piazza in the Fall 2015 class. It was better than GitHub,
  but still not as good as a simple WordPress blog. In particular, it
  was hard to find things once there were more than a dozen homework
  categories.

\end{gitemize}

\subseclbl{Two-Day In-Person}{sec:two-day-in-person-currently-used}

This was the second method we tried. The biggest change was the
introduction of recorded teaching exercises.

\begin{gitemize}

\item
  Several times during the training, participants are divided into
  groups of three and asked to teach a short lesson (typically 2-3
  minutes long). In turn, one person is the teacher, one the audience,
  and one the videographer, who records the teacher using a handheld
  device such as a phone. Group members then rotate roles: the teacher
  becomes the listener, the listener records, and the videographer
  teaches. Once all three have finished teaching, the group reviews
  all three videos, and everyone gives feedback on everyone (including
  themselves). This feedback then goes into the Etherpad for
  discussion.

\item
  It's important to record all three videos and then watch all three:
  if the cycle is teach-review-teach-review, the last person to teach
  runs out of time. Doing all the reviewing after all the teaching
  also helps put a bit of distance between the teaching and the
  reviewing, which makes the exercise slightly less excruciating.

\item
  We use Etherpad or Google Doc for in-person training, both for
  note-taking and for posting exercise solutions and feedback on
  recorded lessons.  Questions and discussion are done aloud.

\end{gitemize}

\subseclbl{Two-Day Online With Groups}{sec:two-day-online-with-groups-currently-used}

In this format, learners are in groups of 4-12, but those groups are
geographically distributed.

\begin{gitemize}

\item
  We use Google Hangouts and either Etherpad or Google Docs as in the
  multi-week version. Each group of learners is together in a room
  using one camera and microphone, rather than each being on the call
  separately. We have found that having good audio matters more than
  having good video, and that the better the audio, the more learners
  can communicate with the instructor and other rooms by voice rather
  than by using the Etherpad chat.

\item
  We do the video lecture exercise as in the two-day in-person training.

\end{gitemize}

\seclbl{Effecting Change}{sec:effecting-change}

This guide is aimed primarily at people working outside mainstream
educational institutions, but in order for us to reach and help as
many people as possible, we must eventually find ways to work with
schools as they are.  Henderson et al's ``Facilitating Change in
Undergraduate STEM Instructional Practices''
\cite{bib:henderson-facilitating} discusses ways to get educational
institutions to actually change what they teach.  Their findings are
summarized in \tblref{tbl:change}, and the approaches they identify
are:

\begin{gitemize}

\item
  \emph{Diffusion}: STEM undergraduate instruction will be changed by
  altering the behavior of a large number of individual
  instructors. The greatest influences for changing instructor
  behavior lie in optimizing characteristics of the innovation and
  exploiting the characteristics of individuals and their networks.

\item
  \emph{Implementation}: STEM undergraduate instruction will be
  changed by developing research-based instructional ``best
  practices'' and training instructors to use them. Instructors must
  use these practices with fidelity to the established standard.

\item
  \emph{Scholarly Teaching}: STEM undergraduate instruction will be
  changed when more individual faculty members treat their teaching as
  a scholarly activity.

\item
  \emph{Faculty Learning Communities}: STEM undergraduate instruction
  will be changed by groups of instructors who support and sustain
  each other's interest, learning, and reflection on their teaching.

\item
  \emph{Quality Assurance}: STEM undergraduate instruction will be
  changed by requiring institutions (colleges, schools, departments,
  and degree programs) to collect evidence demonstrating their success
  in undergraduate instruction. What gets measured is what gets
  improved.

\item
  \emph{Organizational Development}: STEM undergraduate instruction
  will be changed by administrators with strong vision who can develop
  structures and motivate faculty to adopt improved instructional
  practices.

\item
  \emph{Learning Organizations}: Innovation in higher education STEM
  instruction will occur through informal communities of practice
  within formal organizations in which individuals develop new
  organizational knowledge through sharing implicit knowledge about
  their teaching.  Leaders cultivate conditions for both formal and
  informal communities to form and thrive.

\item
  \emph{Complexity Leadership}: STEM undergraduate instruction is
  governed by a complex system. Innovation will occur through the
  collective action of self-organizing groups within the system. This
  collective action can be stimulated, but not controlled.

\end{gitemize}

\begin{tbllbl}{Strategies for Facilitating Educational Change}{tbl:change}
\begin{tabular}{|p{\dimexpr 0.2\linewidth-2\tabcolsep}|p{\dimexpr 0.4\linewidth-2\tabcolsep}|p{\dimexpr 0.4\linewidth-2\tabcolsep}|}

\textbf{Aspect of System to be Changed}
&
\textbf{Intended Outcome}
&
\\

\textbf{Individuals}
&
\textbf{Prescribed}
&
  \textbf{I. Disseminating: Curriculum \& Pedagogy}
  \newline
  Change Agent Role: tell/teach individuals about new teaching conceptions and/or practices and encourage their use.
  \newline
  \emph{Diffusion}
  \newline
  \emph{Implementation}
\\

&
\textbf{Emergent}
&
  \textbf{II. Developing: Reflective Teachers}
  \newline
  Change Agent Role: encourage/support individuals to develop new teaching conceptions and/or practices.
  \newline
  \emph{Scholarly Teaching}
  \newline
  \emph{Faculty Learning Communities}
\\

\textbf{Environments and Structures}
&
\textbf{Prescribed}
&
  \textbf{III. Enacting: Policy}
  \newline
  Change Agent Role: enact new environmental features that require/encourage new teaching conceptions and/or practices.
  \newline
  \emph{Quality Assurance}
  \newline
  \emph{Organizational Development}
\\

&
\textbf{Emergent}
&
  \textbf{IV. Developing: Shared Vision}
  \newline
  Change Agent Role: empower/support stakeholders to collectively develop new environmental features that encourage new teaching conceptions and/or practices.
  \newline
  \emph{Learning Organizations}
  \newline
  \emph{Complexity Leadership}
\\
\end{tabular}
\end{tbllbl}

\seclbl{Why We Are Not a MOOC}{sec:why-were-not-a-mooc}

\begin{quote}
  If you use robots to teach, you teach people to be robots.
  \\
  --- variously attributed
\end{quote}

Massive open online courses (MOOCs) in which students watch videos
instead of attending lectures, and then do assignments that are
(usually) robo-graded, were a hot topic a few years ago.  Now that the
hype has worn off, though, it's clear that they aren't as effective as
their more enthusiastic proponents claimed they would
be \cite{bib:ubell-moocs}.

Recorded content is ineffective for most novices learners because it
cannot intervene to clear up specific learners' misconceptions. Some
people happen to already have the right conceptual categories for a
subject, or happen to form them correctly early on; these are the ones
who stick with most massive online courses, but many discussions of
the effectiveness of such courses ignore this survivor bias.
