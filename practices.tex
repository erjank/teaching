\chaplbl{Teaching Practices}{sec:practices}{15}{30}

\seclbl{Challenges}{sec:challenges}

\begin{callout}{Why Not a MOOC?}{callout:why-not-a-mooc}

Many learners find it difficult to get to a workshop, either because
there isn't one locally or because it's difficult to schedule time
around other commitments, so why don't we create video recordings of the
lessons and offer the workshop as a MOOC (Massive Open Online Course)?

The first answer is that we did in 2010-11, but found the maintenance
costs unsustainable. Making a small change to this webpage only takes a
few minutes. but making \emph{any} change to a video takes an hour or
more. In addition, most people are much less comfortable recording
themselves than contributing written material.

The second answer is that doing significantly outperforms watching.
Specifically, a
recent paper by Koedinger et al \cite{bib:koedinger-doing-watching}
estimated ``\ldots{}the learning benefit from
extra doing (1 SD increase) to be more than six times that of extra
watching or reading.'' \emph{Doing}, in this case, refers to completing
an interactive or mimetic task with feedback, while \emph{benefit}
refers to both \emph{completion rates} and \emph{overall performance}.

And while we do not (yet) have empirical data, we believe very strongly
that many novices would give up in despair if required to debug setup
and installation lessons on their own, but are more likely to get past
these obstacles if someone is present to help them.

An intermediate approach that we are experimenting with is real-time
remote instruction, in which the learners are co-located at one (or a
few) sites, with helpers present, while the instructor(s) teaching via
online video. This model has worked well for this instructor training
course, and for a handful of regular workshops, but more work is needed
to figure out its pros and cons.
\end{callout}
