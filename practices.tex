\chaplbl{Live Coding}{sec:practices}{30}{45}

\begin{quote}
  Teaching is theater not cinema.
  \\
  --- Neal Davis
\end{quote}

Teaching is a performance art, just like drama, music, and athletics.
And as in those fields, we have a collection of small tips and tricks
to make teaching work better.

The first of our recommended teaching practices is so central that it
deserves a chapter of its own: live coding.  When they are live
coding, instructors don't use slides.  Instead, they through the
lesson material, typing in the code or instructions, with their
learners following along.  Its advantages are:

\begin{itemize}

\item
  Watching a program being written is more compelling than watching
  someone page through slides that present bits and pieces of the same
  code.

\item
  It enables instructors to be more responsive to ``what if?''
  questions. Where a slide deck is like a railway track, live coding
  allows instructors to go off road and follow their learners'
  interests.

\item
  It facilitates lateral knowledge transfer: people learn more than we
  realized we were teaching by watching \emph{how} instructors do
  things.

\item
  It slows the instructor down: if she has to type in the program as
  she goes along, she can only go twice as fast as her learners,
  rather than ten-fold faster as she could with slides.

\item
  Learners get to see instructors' mistakes \emph{and how to diagnose
  and correct them}. Novices are going to spend most of their time
  doing this, but it's left out of most textbooks.

\item
  Watching instructors make mistakes shows learners that it's all
  right to make mistakes of their.  Most people model the behavior of
  their teachers: if the instructor isn't embarrassed about making and
  talking about mistakes, learners will be more comfortable doing so
  too.

\end{itemize}

Live coding is an example of the ``I/We/You'' approach to teaching
discussed in \chapref{sec:performance}.  It takes a bit of practice
for instructors to get used to thinking aloud while coding in front of
an audience, but most report that it is then no more difficult to do
than talking off a deck of slides.

\begin{callout}{Double Devices}{callout:double-devices}

Many instructors now use two devices when teaching: a laptop plugged
into the projector for learners to see, and a tablet beside it on
which they can view their notes and the shared notes that the learners
are taking.  This seems to be more reliable than displaying one
virtual desktop while flipping back and forth to another.

\end{callout}

Here are some tips to make your live coding better:

\begin{enumerate}

\item
  \emph{Be seen and heard.}  If you are physically able to stand up
  for a couple of hours, do it while you are teaching. When you sit
  down, you are hiding yourself behind others for those sitting in the
  back rows. Make sure to notify the workshop organizers of your wish
  to stand up and ask them to arrange a high table, standing desk, or
  lectern.

  Regardless of whether you are standing or sitting, make sure to move
  around as much as reasonable. You can for example go to the screen
  to point something out, or draw something on the white/blackboard
  (see below). Moving around makes the teaching more lively, less
  monotonous.  It draws the learners' attention away from their
  screens, to you, which helps get the point you are making across.

  Even though you may have a good voice and know how to use it well,
  it may be a good idea to use a microphone, especially if the
  workshop room is equipped with one. Your voice will be less tired,
  and you increase the chance of people with hearing difficulties
  being able to follow the workshop.

\item
  \emph{Take it slow.}  For every command you type, every word of code
  you write, every menu item or website button you click, say out loud
  what you are doing while you do it. Then point to the command and
  its output on the screen and go through it a second time. This not
  only slows you down, it allows learners who are following along to
  copy what you do, or to catch up, even when they are looking at
  their screen while doing it. If the output of your command or code
  makes what you just typed disappear from view, scroll back up so
  learners can see it again - this is especially needed for the Unix
  shell lesson.

  Other options are to execute the same command a second time, or to
  copy and paste the last command(s) into the workshop's shared notes.

\item
  \emph{Mirror your learner's environment as much as possible.}  You
  may have set up your environment to your liking, with a very simple
  or rather fancy Unix prompt, colour schemes for your development
  environment, keyboard shortcuts etc. Your learners usually won't
  have all of this. Try to create an environment that mirrors what
  your learners have, and avoid using keyboard shortcuts. Some
  instructors create a separate `bare-bone' user (login) account on
  their laptop, or a separate `teaching-only' account on the service
  being taught (e.g., Github).

\item
  \emph{Use the screen wisely.}  Use a big font, and maximize the
  window. A black font on a white background works better than a light
  font on a dark background. When the bottom of the projector screen
  is at the same height, or below, the heads of the learners, people
  in the back won't be able to see the lower parts. Draw up the bottom
  of your window(s) to compensate.

  If you can get a second screen, use it! It will usually require its
  own PC or laptop, so you may need to ask a helper to control it. You
  could use the second screen to show the Etherpad content, or the
  lesson material, or illustrations.

  Pay attention to the lighting (not too dark, no lights directly
  on/above the presenter's screen) and if needed, reposition the
  tables so all learners can see the screen, and helpers can easily
  reach all learners.

\item
  \emph{Use illustrations.}  Most lesson material comes with
  illustrations, and these may help learners to understand the stages
  of the lesson and to organize the material. What can work really
  well is when you as instructor generate the illustrations on the
  white/blackboard as you progress through the material. This allows
  you to build up diagrams, making them increasingly complex in
  parallel with the material you are teaching. It helps learners
  understand the material, makes for a more lively workshop (you'll
  have to move between your laptop and the blackboard) and gathers the
  learners' attention to you as well.

\item
  \emph{Avoid distractions.}  Turn off any notifications you may use
  on your laptop, such as those from social media, email, etc. Seeing
  notifications flash by on the screen distracts you as well as the
  learners - and may even result in awkward situations when a message
  pops up you'd rather not have others see.

\item
  \emph{Improvise after you know the material.}  The first time you
  teach a new lesson, you should stick fairly closely to the topics it
  lays out and the order they're in.  It may be tempting to deviate
  from the material because you would like to show a neat trick, or
  demonstrate some alternative way of doing something. Don't do this,
  since there is a fair chance you'll run into something unexpected
  that you then have to explain.

  Once you are more familiar with the material, though, you can and
  should start improvising based on the backgrounds of your learners,
  their questions in class, and what you find most interesting about
  the lesson.  This is like a musician playing a new song: the first
  few times, you stick to the sheet music, but after you're
  comfortable with it, you can start to put your own stamp on it.

  If you really want to use something outside of the material, try it
  out thoroughly before the workshop: run through the lesson as you
  would during the actual teaching and test the effect of your
  modification.

  Some instructors use printouts of the lesson material during
  teaching.  Others use a second device (tablet or laptop) when
  teaching, on which they can view their own notes and the shared
  notes the learners are taking. This seems to be more reliable than
  displaying one virtual desktop while flipping back and forth to
  another.

\item
  \emph{Embrace mistakes.}

  No matter how well prepared you are, you will be making
  mistakes. Typo's are hard to avoid, you may overlook something from
  the lesson instructions, etc. This is OK! It allows learners to see
  instructors' mistakes and how to diagnose and correct them. Some
  mistakes are actually an opportunity to point something out, or
  reflect back on something covered earlier. Novices are going to
  spend most of their time making the same and other mistakes, but how
  to deal with them is left out of most textbooks.

  \begin{quote}
    The typos are the pedagogy.
    \\
    --- Emily Jane McTavish
  \end{quote}

\item
  \emph{Have fun.}  Teaching is performance art and can be rather
  serious business. On the one hand, don't let this scare you - it is
  much easier than performing Hamlet. You have an excellent script at
  your disposal, after all! On the other hand, it is OK to add an
  element of `play', i.e. use humor and improvisation to liven up the
  workshop. How much you are able and willing to do this is really a
  matter of personality and taste - as well as experience. It becomes
  easier when you are more familiar with the material, allowing you to
  relax more. Choose your words and actions wisely, though. Remember
  that we want the learners to have a welcoming experience and a
  positive learning environment - a misplaced joke can ruin this in an
  instant. Start small, even just saying `that was fun' after
  something worked well is a good start. Ask your co-instructors and
  helpers for feedback when you are unsure of the effect you behaviour
  has on the workshop.

\end{enumerate}

\seclbl{Pre-Assessment}{sec:practices-assessment}

Most formal educational systems train people to treat all assessment
as summative, i.e., to think of every interaction with a teacher as an
evaluation, rather than as a chance to shape instruction.  For
example, we use a short pre-assessment questionnaire to profile
learners before workshops to help instructors tune the pace and level
of material. We send this questionnaire out after people have
registered rather than making it part of the sign-up process because
when we did the latter, many people concluded that since they couldn't
answer all the questions, they shouldn't enrol. We were therefore
scaring off many of the people we most wanted to help.

Instead of asking people how easily they could complete specific tasks,
we could just ask them to rate their knowledge of various subjects on a
scale from 1 to 5. However, self-assessments of this kind are usually
inaccurate because of the
\href{https://en.wikipedia.org/wiki/Dunning\%E2\%80\%93Kruger\_effect}{Dunning-Kruger
effect}: the less people know about a subject, the less accurate their
estimate of their knowledge is.

That said, there \emph{are} things we can do:

\begin{itemize}

\item
  Before running a workshop, communicate its level clearly to everyone
  who's thinking of signing up by listing the topics that will be
  covered and showing a few examples of exercises that people will be
  asked to complete.

\item
  Provide multiple exercises for each teaching episode so that more
  advanced learners don't finish early and get bored.

\item
  Ask more advanced learners to help people next to them. They'll
  learn from answering their peers' questions (since it will force
  them to think about things in new ways).

\item
  The helpers and the instructor who isn't teaching the particular
  episode should keep an eye out for learners who are falling behind
  and intervene early so that they don't become frustrated and give
  up.

\end{itemize}

The most important thing is to accept that no class can possibly meet
everyone's individual needs. If the instructor slows down to accommodate
two people who are struggling, the other 38 are not being well served.
Equally, if she spends a few minutes talking about an advanced topic
because two learners are bored, the 38 who don't understand it will feel
left out. All we can do is tell our learners what we're doing and why,
and hope that they'll understand.

\seclbl{Setup}{sec:practices-setup}

Learners tell us that it is important to them to leave the workshop
with their own machine set up to do real work. We therefore continue
to teach on all three major platforms (Linux, Mac OS X, and Windows),
even though it would be simpler to require learners to use just one.

\fixme{Talk about getting learners' machines set up.}

\begin{callout}{Virtual Machines}{callout:practices-vm}

We have experimented with virtual machines (VMs) on learners'
computers to reduce installation problems, but those introduce
problems of their own: older or smaller machines simply aren't fast
enough, and learners often struggle to switch back and forth between
two different sets of keyboard shortcuts for things like copying and
pasting.

Some instructors use VPS over SSH or web browser pages instead. This
solve the installation issues, but makes us dependent on host
institutions' WiFi (which can be of highly variable quality), and has
the issues mentioned above with things like keyboard shortcuts.

\end{callout}

\seclbl{Teaching Online}{sec:online}

Many learners find it difficult to get to a workshop, either because
there isn't one locally or because it's difficult to schedule time
around other commitments, so why don't we create video recordings of
the lessons and offer the workshop as a MOOC (Massive Open Online
Course)?

The first answer is that we did in 2010-11, but found the maintenance
costs unsustainable. Making a small change to this webpage only takes
a few minutes. but making \emph{any} change to a video takes an hour
or more. In addition, most people are much less comfortable recording
themselves than contributing written material.

The second answer is that doing significantly outperforms watching.
Specifically, a recent paper by Koedinger et
al \cite{bib:koedinger-doing-watching} estimated ``{\ldots}the
learning benefit from extra doing (1 SD increase) to be more than six
times that of extra watching or reading.'' \emph{Doing}, in this case,
refers to completing an interactive or mimetic task with feedback,
while \emph{benefit} refers to both \emph{completion rates}
and \emph{overall performance}.

And while we do not (yet) have empirical data, we believe very
strongly that many novices would give up in despair if required to
debug setup and installation lessons on their own, but are more likely
to get past these obstacles if someone is present to help them.

An intermediate approach that has proven quite successful is real-time
remote instruction, in which the learners are co-located at one (or a
few) sites, with helpers present, while the instructor(s) teaching via
online video. This model has worked well for this instructor training
course, and for a handful of regular workshops, but more work is
needed to figure out its pros and cons.

\seclbl{Challenges}{sec:practices-challenges}

\begin{challenge}{The Bad and the Good}{chal:practices-bad-and-good}

Watch this video of
\href{https://youtu.be/bXxBeNkKmJE}{live coding done poorly}
and this video of
\href{https://youtu.be/SkPmwe\_WjeY}{live coding done right}
as a group and then summarize your feedback on both using the usual
$2{\times}2$ grid.  These videos assume learners know what a shell
variable is, know how to use the \texttt{head} command, and are
familiar with the contents of the data files being filtered.

\end{challenge}

\begin{challenge}{See Then Do}{chal:practices-live}

Teach 3-4 minutes of your chosen lesson episode using live coding to a
fellow trainee, then swap and watch while that person live codes for
you. Don't bother trying to record the live coding sessions---we have
found that it's difficult to capture both the person and the screen
with a handheld device---but give feedback the same way you have
previously (positive and negative, content and presentation). If you
decide to instead teach something from the lesson episode you selected
in preparation for this workshop, explain in advance to your fellow
trainee what you will be teaching and what the learners you teach it
to are expected to be familiar with.

How does live coding feel compared to other kinds of teaching you have
done?  What is different, and what is the same?

\end{challenge}
