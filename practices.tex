\chaplbl{Teaching Practices}{s:practices}

We regard teaching as a performance art, no different from drama, music,
or athletics. And as in those fields, we have a collection of small tips
and tricks to make teaching work better.

\subsection{Challenges}\label{challenges}

With live coding, it is easy for some learners to fall behind, and for
other learners to get bored. Given the diversity of our learners'
backgrounds and skills, we will always have a mix of more and less
advanced people in our classes. No matter what we teach, and how fast or
how slow we go, 20\% or more of the room will be lost, and there's a
good chance that a different 20\% will be bored.

The obvious solution is to split people by level, but if we ask them how
much they know about particular things, they regularly under- or
over-estimate their knowledge. We have therefore developed a short
pre-assessment questionnaire that asks them how easily they could do a
small number of specific tasks. It gives instructors some idea of who
they're going to be helping, but we have not validated the questions,
i.e., we have not done the laborious work of interviewing respondents to
ensure that they are interpreting the questions the same way that we
are. We also have not yet done the follow-up to see whether the
questionnaires' categorization of learners matches their actual in-class
proficiency.

\begin{callout}{You Can't Just Ask}{callout:you-cant-just-ask}

Instead of asking people how easily they could complete specific tasks,
we could just ask them to rate their knowledge of various subjects on a
scale from 1 to 5. However, self-assessments of this kind are usually
inaccurate because of the
\href{https://en.wikipedia.org/wiki/Dunning\%E2\%80\%93Kruger\_effect}{Dunning-Kruger
effect}: the less people know about a subject, the less accurate their
estimate of their knowledge is.
\end{callout}

That said, there \emph{are} things we can do:

\begin{itemize}
\item
  Before running a workshop, communicate its level clearly to everyone
  who's thinking of signing up by listing the topics that will be
  covered and showing a few examples of exercises that people will be
  asked to complete.
\item
  Provide multiple exercises for each teaching episode so that more
  advanced learners don't finish early and get bored.
\item
  Ask more advanced learners to help people next to them. They'll learn
  from answering their peers' questions (since it will force them to
  think about things in new ways).
\item
  The helpers and the instructor who isn't teaching the particular
  episode should keep an eye out for learners who are falling behind and
  intervene early so that they don't become frustrated and give up.
\end{itemize}

The most important thing is to accept that no class can possibly meet
everyone's individual needs. If the instructor slows down to accommodate
two people who are struggling, the other 38 are not being well served.
Equally, if she spends a few minutes talking about an advanced topic
because two learners are bored, the 38 who don't understand it will feel
left out. All we can do is tell our learners what we're doing and why,
and hope that they'll understand.

\subsection{Other Practices}\label{other-practices}

\begin{description}
\item[Sticky notes.]
We give each learner two sticky notes of different colors, e.g., red and
green. These can be held up for voting, but their real use is as status
flags. If someone has completed an exercise and wants it checked, they
put the green sticky note on their laptop; if they run into a problem
and need help, the put up the red one. This is better than having people
raise their hands because:

\begin{itemize}
\item
  it's more discreet (which means they're more likely to actually do
  it),
\item
  they can keep working while their flag is raised, and
\item
  the instructor can quickly see from the front of the room what state
  the class is in.
\end{itemize}
\item[Minute cards.]
As explained in \secref{FIXME},
we also use sticky notes as minute cards: before each coffee or meal
break, learners take a minute to write one positive thing on the green
sticky note (e.g., one thing they've learned that they think will be
useful), and one thing they found too fast, too slow, confusing, or
irrelevant on the red one. They can use the red sticky note for
questions that haven't yet been answered. It only takes a few minutes to
cluster these, and allows the instructors to adjust to learners'
interests and speed.
\item[One up, one down.]
We frequently also ask for summary feedback at the end of each day. The
instructors ask the learners to alternately give one positive and one
negative point about the day, without repeating anything that has
already been said. This requirement forces people to say things they
otherwise might not: once all the ``safe'' feedback has been given,
participants will start saying what they really think.

Minute cards are anonymous; the alternating up-and-down feedback is not.
Each mode has its strengths and weaknesses, and by providing both, we
hope to get the best of both worlds.
\item[Learners use their own machines.]
Learners tell us that it is important to them to leave the workshop with
their own machine set up to do real work. We therefore continue to teach
on all three major platforms (Linux, Mac OS X, and Windows), even though
it would be simpler to require learners to use just one.

We have experimented with virtual machines (VMs) on learners' computers
to reduce installation problems, but those introduce problems of their
own: older or smaller machines simply aren't fast enough, and learners
often struggle to switch back and forth between two different sets of
keyboard shortcuts for things like copying and pasting.

Some instructors use VPS over SSH or web browser pages instead. This
solve the installation issues, but makes us dependent on host
institutions' WiFi (which can be of highly variable quality), and has
the issues mentioned above with things like keyboard shortcuts.
\item[Collaborative note-taking.]
We often use \href{http://etherpad.org}{Etherpad} for collaborative
note-taking and to share snippets of code and small data files with
learners. (If nothing else, it saves us from having to ask students to
copy long URLs from the presenter's screen to their computers.) It is
almost always mentioned positively in post-workshop feedback, and
several workshop participants have started using it in their own
teaching.

One of the advantages of collaborative note-taking is that it gives the
more advanced learners in the class something useful to do. Another is
that the notes the learners take are usually more helpful \emph{to them}
than those the instructor would prepare in advance, since the learners
are more likely to write down what they actually found new, rather than
what the instructor predicted would be new. Finally, scanning the
Etherpad is a good way for an instructor to discover that the class
didn't hear something important, or misunderstood it.
\item[Pair programming.]
Pairing is a good practice in real life, and also a
good way to teach \cite{bib:porter-what-works}. Partners can not only help each other out during the
practical, but can also clarify each other's misconceptions when the
solution is presented, and discuss common research interests during
breaks. To facilitate this, we strongly prefer flat (dinner-style)
seating to banked (theater-style) seating; this also makes it easier for
helpers to reach learners who need assistance.

When pair programming is used it's important to put \emph{everyone} in
pairs, not just the learners who are struggling, so that no one feels
singled out. It's also useful to have people sit in new places (and
hence pair with different partners) after each coffee or meal break.
\item[Peak rule.]
The \href{https://en.wikipedia.org/wiki/Peak\%E2\%80\%93end\_rule}{peak
rule} states that people judge an experience primarily based on how they
felt at its most intense point and how they felt at its end. While it
has been criticized for not strongly predicting what's remembered, it's
always worth trying to end a lesson on a high note.
\item[Instructor notes.]
Many of the Software and Data Carpentry lessons have instructor's notes,
with information from instructors who have already taught the material.
This can be a valuable resource when preparing lessons, especially when
teaching a lesson for the first time.\\The Software Carpentry instructor
guides are linked on each lesson page; the instructor guides for Data
Carpentry lessons are on their
main lesson page.
\end{description}

\subsection{The Code of Conduct}\label{the-code-of-conduct}

Beyond the teaching practices and philosophies found in Software and
Data Carpentry workshops, one of the most important characteristics of
our workshops is that they be welcoming and safe spaces. Programming, or
data management are scary topics to novices, and workshops are meant to
be a judgment free space to learn and experiment. The behavior of the
instructor and other partipants may make more of an impression on a
novice learner than any ``technical'' topic you teach.

To support this mission Software Carpentry and Data Carpentry have a
shared code of conduct that
participants in our workshops are required to abide by. Hosts
\emph{must} point people at it during registration, and instructors
\emph{must} remind attendees of it at the start of the workshop. The
Code of Conduct doesn't just tell everyone what the rules are: it tells
them that there \emph{are} rules, and that they can therefore expect a
safe and welcoming learning experience.

\begin{description}
\item[What's the purpose of a Code of Conduct?]
A Code of Conduct cannot stop people from being offensive, any more than
laws against theft stop people from stealing. What the CoC \emph{can} do
is make expectations and consequences clear.
\item[Do workshop participants ever actually violate the Code of
Conduct?]
Very rarely, but it has happened.
\item[But what about free speech?]
People are free to say what they want, but that doesn't mean they are
free to say it in our workshops. As in any classroom, the instructor has
the right to sanction students who are being disruptive.
\item[What should I do if someone violates the Code of Conduct?]
If you are an instructor, and believe that someone in a workshop has
violated the Code of Conduct, you may warn them, ask them to apologize,
and/or expel them, depending on the severity of the violation and
whether or not you believe it was intentional. No matter what you choose
to do, you should contact the appropriate Carpentry administrator at
\texttt{admin@software-carpentry.org}
as soon as you can, and describe what happened in the next online
debriefing session that you're able to attend.

You also have the right as an instructor to walk out of a workshop if
you feel that the participants or hosts are not supporting your attempts
to enforce the Code of Conduct. Again, please contact us as soon as
possible if this happens.
\end{description}

A Code of Conduct is one way to create a safe learning space - making
sure that you are properly motivating (and not de-motivating) your
students is another. In our next module, we will talk in more detail
about what we can do to support motivation in our learners.

\begin{callout}{Why Not a MOOC?}{callout:why-not-a-mooc}

Many learners find it difficult to get to a workshop, either because
there isn't one locally or because it's difficult to schedule time
around other commitments, so why don't we create video recordings of the
lessons and offer the workshop as a MOOC (Massive Open Online Course)?

The first answer is that we did in 2010-11, but found the maintenance
costs unsustainable. Making a small change to this webpage only takes a
few minutes. but making \emph{any} change to a video takes an hour or
more. In addition, most people are much less comfortable recording
themselves than contributing written material.

The second answer is that doing significantly outperforms watching.
Specifically, a
recent paper by Koedinger et al \cite{bib:koedinger-doing-watching}
estimated ``\ldots{}the learning benefit from
extra doing (1 SD increase) to be more than six times that of extra
watching or reading.'' \emph{Doing}, in this case, refers to completing
an interactive or mimetic task with feedback, while \emph{benefit}
refers to both \emph{completion rates} and \emph{overall performance}.

And while we do not (yet) have empirical data, we believe very strongly
that many novices would give up in despair if required to debug setup
and installation lessons on their own, but are more likely to get past
these obstacles if someone is present to help them.

An intermediate approach that we are experimenting with is real-time
remote instruction, in which the learners are co-located at one (or a
few) sites, with helpers present, while the instructor(s) teaching via
online video. This model has worked well for this instructor training
course, and for a handful of regular workshops, but more work is needed
to figure out its pros and cons.
\end{callout}
