\chaplbltime{Introductions}{sec:welcome}{15}{15}

There's more to programming than typing in code.  Good programmers
break their software up into functions, automate repetitive tasks, and
keep track of their work using version control.  Once they have
mastered those basics, they start doing code reviews and writing tests
as they go along.  They don't invent these techniques for themselves:
more experienced programmers teach them (either explicitly or by
example), and they in turn pass them on to others.

Similarly, there's more to teaching than talking.  Good teachers break
subjects up into digestible pieces, design lessons with verifiable
goals in mind, check their students' progress at short intervals, and
encourage collaboration and improvisation.  Like good programming
practices, these don't have to be reinvented by every teacher: they
can and should be taught and learned.  And while they can't
automatically make someone a great teacher, they do make people better
teachers.

This training course is a fast, wide-ranging, and (necessarily)
incomplete introduction to modern evidence-based teaching practices.
Its aim is introduce you to what we actually know about teaching and
learning, why we believe it is true, and how you can apply it.  We
will look at:

\begin{gitemize}

\item
  how people's thinking changes as they go from being novices to
  competent practitioners and then to being experts;

\item
  how to tell if your learners are keeping up with you, and what to
  do or say when they're not;

\item
  how to design and improve lessons efficiently and collaboratively;

\item
  how and why live coding is a better way to teach programming than
  lectures or self-directed practice; and

\item
  how insights and techniques borrowed from the performing arts can
  make you a better teacher.

\end{gitemize}

We can't cover everything you need to know to teach well.  In fact,
we can barely scratch the surface.  But we hope that what we show you
will be useful, and will convince you that better is possible.

\section*{History}

I started teaching people how to program in the late 1980s.  At first,
I was pretty bad at: I went too fast, I used too much jargon, and I
assumed that my learners would be interested in the same things I was.
I got better over time, but that doesn't mean I was actually very
good.  And even though I felt I was improving, I had no idea how
effective I was compared to other teachers.

My doubts came to a head after I re-launched Software Carpentry in
2010.  Its aim was (and is) to teach basic computing skills to
researchers from a wide range of disciplines.  What I realized then
was that, ironically, I was trying to teach people how to build and
run software in efficient, repeatable ways, but was mostly ignorant of
equivalent techniques for writing and delivering lessons.

Luckily, I discovered resources like Mark Guzdial's blog
\cite{bib:guzdial-blog} and the book \emph{How Learning
Works} \cite{bib:ambrose-hlw}.  These led me to the work of Lemov,
Huston, Green, and others
\cite{bib:lemov-champion,bib:huston-dont-know,bib:green-babt},
which showed me what would make my teaching better, and why I should
believe it.

I started using these ideas in Software Carpentry in 2012, and the
results were everything I'd hoped for.  Cutting the number of lessons,
shifting to in-person delivery, and using live coding all had an
impact.  What made the biggest difference, though, was creating a
training course to turn learners into teachers.

While I originally delivered the course online over multiple weeks, by
2014 I was teaching it in two intensive days, just like our regular
software skills workshops.  A year later, ``I'' became ``we'' as
people who had taught regular workshops began training new instructors
as well.  That same year, I began teaching half-day and one-day
versions of the course to people who wanted to help children, recent
immigrants, women re-entering the workforce, and a wide variety of
others.

Those experiences are the basis of this book.  I am grateful to
everyone who helped shape the Software Carpentry instructor training
course, including Erin Becker, Karen Cranston, Neal Davis, Rayna
Harris, Kate Hertweck, Christina Koch, Sue McClatchy, Lex Nederbragt,
Elizabeth Patitsas, Aleksandra Pawlik, Ariel Rokem, Tracy Teal, Fiona
Tweedie, Allegra Via, Anelda van der Walt, Belinda Weaver, Jason
Williams, and the hundreds of people who went through it over the
years.  I hope you enjoy what follows.  If you do, I hope you pass on
whatever you find helpful to someone else.

\begin{callout}{Who You Are}{callout:profiles}

\secref{sec:learner-profiles} will explain how to use \emph{learner
profiles} to define who a class is for.  Here, we present profiles of
two typical participants in a workshop based on this book.

\emph{Samira} is an undergraduate student in mechanical engineering
who first encountered the subject in an after-school club for girls
and would now like to pass on her love for it.  She has done one
programming class and one robotics class, and has been a lab assistant
for a couple of weekend introductions to engineering for high school
students at her university, but doesn't know if she's ready to stand
up and teaching.  This class will introduce her to some basic
classroom practices and give her a chance to try them out in front of
a supportive audience.

\emph{Moshe} has been writing accounting software for almost twenty
years.  His children's school doesn't offer a programming class, so he
has volunteered to help put one together.  He has never written
lessons before, and after reading a dozen different ``programming for
kids'' books, is feeling more confused than ever.  This class will
show him how to design and deliver lessons tailored for his students
(many of whom have hearing disabilities), and how to tell how well
those lessons are working.

\end{callout}

\section*{Teaching Practices}

This book can be read on its own, but it is more effective when used
as part of an intensive in-person class.  We suggest that workshops
adopt these three practices right from the start:

\begin{gitemize}

\item
  Have a code of conduct (\secref{sec:practices-conduct}).

\item
  Take notes together (\secref{sec:practices-notes}).

\item
  Assess learners' motivation and prior knowledge (\secref{sec:practices-assess}).

\end{gitemize}

\seclbl{Challenges}{sec:welcome-challenges}

\begin{challenge}{Favorite Class}{chal:favorite-class}

In the online notes, write down your name, the best class you ever
took, and what made it so great.

\end{challenge}
