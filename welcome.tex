\chaplbl{Introductions}{sec:welcome}

There's more to programming than typing in code.  Good programmers
break their software up into functions, automate repetitive tasks, and
keep track of their work using version control.  Once they have
mastered those basics, they start doing code reviews and writing tests
as they go along.  They don't invent these techniques for themselves:
more experienced programmers teach them (either explicitly or by
example), and they in turn pass them on to others.

Similarly, there's more to teaching than talking.  Good teachers break
subjects up into digestible pieces, design lessons with verifiable
goals in mind, check their students' progress at short intervals, and
encourage collaboration and improvisation.  Like good programming
practices, these don't have to be reinvented by every teacher: they
can and should be taught and learned.  And while they can't
automatically make someone a great teacher, they do make people better
teachers.

\seclbl{What We Will Cover}{sec:welcome-cover}

This training course is a fast, wide-ranging, and (necessarily)
incomplete introduction to modern evidence-based teaching practices.
Its aim is introduce you to what we actually know about teaching and
learning, why we believe it is true, and how you can apply it.  We
will look at:

\begin{itemize}

\item
  how people's thinking changes as they go from being novices to
  competent practitioners and then to being experts;

\item
  how to tell if your learners are keeping up with you, and what to
  do or say when they're not;

\item
  how to design and improve lessons efficiently and collaboratively;

\item
  how and why live coding is a better way to teach programming than
  lectures or self-directed practice; and

\item
  how insights and techniques borrowed from the performing arts can
  make you a better teacher.

\end{itemize}

We can't cover everything you need to know to teach well.  In fact,
we can barely scratch the surface.  But we hope that what we show you
will be useful, and will convince you that better is possible.

\seclbl{History}{sec:welcome-history}

I started teaching people how to program in the late 1980s.  At first,
I was pretty bad at: I went too fast, I used too much jargon, and I
assumed that my learners would be interested in the same things I was,
for the same reasons.

I got better over time, but that doesn't mean I was actually very
good.  My students seemed to like me, but studies have repeatedly
shown that there's no correlation between how people rate a course and
how much they really learn \cite{fixme}.  And even though I felt I was
improving, I had no idea how effective I was compared to other
teachers.  Did they spend the same hours sweating over their lectures
that I did?  Did they have to revise each homework assignment half a
dozen times as students stumbled over omissions and contradictions?
And how did they handle three guys in the back of the class playing
poker online while they were lecturing?

My doubts came to a head after I re-launched the Software Carpentry
project in 2010.  Its aim was (and is) to teach basic computing skills
to researchers from a wide range of disciplines.  What I eventually
realized was that, ironically, I was trying to teach people how to
build and run software in efficient, repeatable ways, but was mostly
ignorant of equivalent techniques for writing and delivering lessons.

Luckily, I also discovered that as a species, we know as much about
teaching and learning as we do about public health.  Mark Guzdial's
blog \cite{fixme} and the book \emph{How Learning Works} \cite{fixme}
led me to the work of Lemov, Huston, Green, Fincher, and others that
showed me what would make my teaching better, and why I should believe
it.  (See the bibliography for an annotated list of readings.)

I restarted Software Carpentry once again in January 2012, and this
time it worked as I'd always hoped it would.  Cutting the number of
lessons, shifting to in-person delivery, and using live coding instead
of slides all had an impact, but what made the biggest difference was
starting an instructor training course to turn learners into teachers.
This approach was inspired by Myles Horton's Highlander School
\cite{fixme}, the Antigonish Movement in Eastern Canada \cite{fixme},
and dozens of similar co-operative grassroots efforts.

While I initially delivered the course online over multiple weeks, by
2014 I was teaching it in two intensive days, just like our regular
software skills workshops.  A year later, ``I'' became ``we'' as
people who had been instructors for regular workshops began training
other instructors as well.  That same year, I began teaching half-day
and one-day versions of the course to people outside Software
Carpentry who wanted to help children, recent immigrants, women
re-entering the workforce, and a wide variety of others learn how to
make computers work for them.

Those readings and experiences are the basis of this book, but its
roots go much deeper.  I am grateful, always, to my own teachers for
starting me down this path, including Louis Biczo, Lon Taylor, Ken
Douglas, and above all, my father, Rob Wilson.  I am also grateful to
the people who helped create the Software Carpentry instructor
training course, including Erin Becker, Karen Cranston, Neal Davis,
Rayna Harris, Kate Hertweck, Christina Koch, Sue McClatchy, Lex
Nederbragt, Elizabeth Patitsas, Aleksandra Pawlik, Ariel Rokem, Tracy
Teal, Fiona Tweedie, Allegra Via, Anelda van der Walt, Belinda Weaver,
Jason Williams, and the hundreds of people who went through it in
various guises over the years.  I hope you enjoy it, and if you do, I
hope you pass on whatever you find helpful to someone else.

\seclbl{Teaching Practices}{sec:welcome-practices}

This book can be read on its own, but it is more effective when used
as part of an intensive in-person class.  Accordingly, we start by
presenting three practices that you can use while taking such a class,
as well as in the workshops you teach yourself.

\subseclbl{Have a Code of Conduct}{sec:welcome-conduct}

An important part of making a class productive is to treat everyone
with respect.  We therefore strongly recommend that every group
offering classes based on this material adopt a Code of Conduct like
the one in \secref{sec:conduct}, and require people taking part in the
class to abide by it.

We believe equally strongly that your actual programming classes
should also have and enforce a Code of Conduct.  Programming is a
scary topic for many novices, and workshops are meant to be a judgment
free space to learn and experiment. The behavior of the instructor and
other participants may make more of an impression on a novice learner
than any ``technical'' topic you teach.

If you do this, hosts should point people at it during registration,
and instructors should remind attendees of it at the start of the
workshop. The Code of Conduct doesn't just tell everyone what the
rules are: it tells them that there \emph{are} rules, and that they
can therefore expect a safe and welcoming learning experience.

If you are an instructor, and believe that someone in a workshop has
violated the Code of Conduct, you may warn them, ask them to
apologize, and/or expel them, depending on the severity of the
violation and whether or not you believe it was intentional.  Whatever
you do:

\begin{itemize}

\item
  Do it in front of witnesses.  Most people will tone down their
  language and hostility in front of an audience, and having someone
  else present ensures that later discussion doesn't degenerate into
  conflicting claims about who said what.

\item
  Contact the organizer or host of your class as soon as you can and
  describe what happened.  Remember, a Code of Conduct is meaningless
  without a procedure for enforcing it.

\end{itemize}

A Code of Conduct cannot stop people from being offensive, any more
than laws against theft stop people from stealing. What it \emph{can}
do is make expectations and consequences clear.  In our experience,
people rarely violate the Code of Conduct in person, though some are
more likely to online, where they feel less inhibited.  And remember,
a Code of Conduct is \emph{not} an infringement on free speech.
People have a right to say what they think, but that doesn't mean they
have a right to speak wherever and whenever they want.  If someone
wishes to say something disparaging about someone else, they can go
and find a space of their own in which to say it.

\subseclbl{Take Notes Together}{sec:welcome-notes}

Many studies have shown that taking notes while learning improves
retention, even if those notes are then thrown away \cite{fixme}.  As
we will discuss in \secref{sec:memory}, this happens because taking
notes forces you to organize and reflect on material as it's coming
in, which in turn increases the likelihood that you will transfer it
to long-term memory in a usable way.

Our experience, and some early research findings \cite{fixme}, lead us
to believe that taking notes \emph{collaboratively} helps learning
even more:

\begin{itemize}

\item
  It allows people to compare what they think they're hearing with
  what other people are hearing, which helps them fill in gaps and
  correct misconceptions right away.

\item
  It gives the more advanced learners in the class something useful to
  do.  Rather than getting bored and checking Twitter during class,
  they often take the lead in recording what's being said, which keeps
  them engaged (and allows less advanced learners to focus more of
  their attention on new material).

\item
  The notes the learners take are usually more helpful \emph{to them}
  than those the instructor would prepare in advance, since the learners
  are more likely to write down what they actually found new, rather than
  what the instructor predicted would be new.

\item
  Glancing at the notes as they're being taken helps the instructor
  discover that the class didn't hear something important, or
  misunderstood it.

\end{itemize}

We usually use \href{http://etherpad.org}{Etherpad} for collaborative
note-taking, though many instructors have shifted to
\href{https://docs.google.com}{Google Docs}, both because it scales
better and because it allows people to add images to the notes.
Whichever is chosen, classes also use it to share snippets of code and
small datasets, and as a way for learners to show instructors their
work (by copying and pasting it in).

Shared note-taking is almost always mentioned positively in
post-workshop feedback.  However, it's also common for participants to
report that they find it distracting, as it's one more thing they have
to keep an eye on.  We believe the positives outweigh the negatives,
but think that some careful controlled studies would tell us whether
we're right, and how to use it better.

\subseclbl{Assess Learners' Motivation and Prior Knowledge}{sec:welcome-assess}

It's important to design lessons with a particular audience in mind.
It's equally important to find out who's in each specific audience,
since this will influence how you introduce yourself, motivate topics,
and pace the lessons.  Before the start of a Software Carpentry
instructor training class, we ask people to fill in a short
questionnaire like the one below.  It doesn't tell us everything we
might want to know, but it does give trainers a pretty clear idea of
who they're speaking to.

\begin{enumerate}

\item
  Have you ever participated in a Software Carpentry or Data Carpentry
  workshop? (Check all that apply.)

  \begin{todolist}
  \item
    Yes, as a learner.
  \item
    Yes, as a helper.
  \item
    Yes, as an organizer.
  \item
    Yes, as an instructor.
  \item
    No, but I am familiar with what is taught at a workshop.
  \item
    No, and I am not familiar with what is taught at a workshop.
  \end{todolist}

\item
  Which of these describes your teaching experience?  (Check all that
  apply.)

  \begin{todolist}
  \item
    I have none.
  \item
    I have taught a seminar, workshop, or other short or informal course.
  \item
    I have been a graduate or undergraduate teaching assistant for a
    college- or university-level course.
  \item
    I have been the instructor-of-record for a college- or
    university-level course.
  \item
    I have taught at the K-12 level.
  \end{todolist}

\item
  Which of these describes your previous formal training in teaching?
  (Please choose only one.)

  \begin{todolist}
  \item
    None
  \item
    A few hours
  \item
    A workshop
  \item
    A certification or short course
  \item
    A full degree
  \end{todolist}

\item
  How frequently do you work with the tools that Data Carpentry and
  Software Carpentry teach, such as R, Python, MATLAB, Perl, SQL, Git,
  OpenRefine, and the Unix Shell?

  \begin{todolist}
  \item
    Every day
  \item
    A few times a week
  \item
    A few times a month
  \item
    A few times a year
  \item
    Never or almost never
  \end{todolist}

\item
  How often would you expect to teach on Software or Data Carpentry
  Workshops after this training?

  \begin{todolist}
  \item
    Not at all
  \item
    Once a year
  \item
    Several times a year
  \end{todolist}

\item
  Why do you want to take this training course?\\
  \line(1,0){275}

\end{enumerate}

\seclbl{Challenges}{sec:welcome-challenges}

\begin{challenge}{Favorite Class}{chal:favorite-class}

In the online notes, write down your name, the best class you ever
took, and what made it so great.

\end{challenge}

\begin{challenge}{Create a Questionnaire}{chal:favorite-class}

Using the questionnaire in \secref{sec:welcome-assess} as a template,
create a short questionnaire you could give learners before teaching a
class of your own.  What do you most want to know about their
background?

\end{challenge}
