\chaplbl{Introductions}{s:intro}

\begin{quote}
Introductions set the stage for learning.\\--- Tracy Teal, Executive
Director, Data Carpentry
\end{quote}

\begin{quote}
\subsection{Favorite Class}\label{favorite-class}

In the Etherpad, write down your name, the best class you ever took (or
one class from your top ten, if you can't decide), and what made it so
great. \{: .challenge\}
\end{quote}

Hello everyone, and welcome to the Data Carpentry and Software Carpentry
instructor training. We're very pleased to have you with us.

\begin{discussion}{Today's Trainers}{disc:todays-trainers}

Each trainer can introduce themselves.

\end{discussion}

Most of you are probably here because you have attended or helped at a
Software or Data Carpentry workshops. To make sure everyone has the same
context, we'll give a brief overview of the Software and Data Carpentry
organizations.

Software and Data Carpentry are both communities of volunteer
researchers, educators, and more who develop lessons and teach two day
workshops on basic computing and data skills for researchers. Software
Carpentry focuses on how researchers can be effective computationally
and developing software; Data Carpentry focuses on how researchers can
effectively manage their data. Both organizations aren't out to teach
specific skills, per se - although those are covered - but instead, the
main goal is to convey best practices that will enable researchers to be
more productive and do better research.

In the same way, this training will cover specific teaching skills, but
one of our main emphases will be the ``best practices'' of teaching. We
want to introduce you to a handful of key educational research findings
and show how they can be used to help people learn better and faster. We
will also be introducing you to the teaching practices that have been
adopted by the Software and Data Carpentry communities, and the overall
philosophy and procedures of both organizations in order to prepare you
to teach at Software and Data Carpentry workshops.

We will not be going over the workshop content in detail (although we
will talk about the lessons tomorrow), but instead focus on developing
teaching skills that are broadly useful across all of our lessons. Part
of this is because this two day training is the first step in getting
fully certified to teach Software and/or Data Carpentry workshops. The
follow-up steps for full certification will require that you dig into
the workshop content yourself and we'll talk about that more tomorrow
afternoon.

To orient yourself, there is a schedule on the workshop webpage.

One aspect of this course's design is its relationship to teaching as a
performance art. Just as musicians learn theory, practice techniques,
and perform for each other, we're going to be looking at some learning
theory (also known as educational psychology) and why it matters to us
as Software and Data Carpentry instructors, create exercises and
learning materials, and practice our teaching. You can expect a lot of
hands-on work and discussion. We'll be using the the Etherpad to collect
answers and help facilitate discussion - if you ever have any questions,
feel free to put them into the notes or chat box of the Etherpad and
we'll make sure we get to them by the end of the day.

One part of making this a productive two days for all of us is a
community effort to treat one another with kindness and respect. This
training, as in all Software/Data Carpentry workshops is subject to the
Software and Data Carpentry Code of Conduct. We will be able to give our
best effort (and have the most fun!) if everyone abides by these
guidelines.

The greatest asset of Software and Data Carpentry is people like
yourself - people who want to help researchers learn about these ideas
and share their own experience and enthusiasm. We hope that this
training gives everyone a chance to meet new people and share ideas.

\subsection{Assessing Trainee Motivation and Prior
Knowledge}\label{assessing-trainee-motivation-and-prior-knowledge}

It's important to first assess the prior knowledge of the workshop
participants because this will influence (to some extent) how you
motivate the activities and how you communicate with the attendees.

\begin{discussion}{Background}{disc:background}

\emph{Have you ever participated in a Software Carpentry or Data
Carpentry Workshop?}

\begin{enumerate}
\def\labelenumi{\arabic{enumi}.}
\itemsep1pt\parskip0pt\parsep0pt
\item
  Yes, I have taken a workshop.
\item
  Yes, I have been a workshop helper.
\item
  Yes, I organized a workshop.
\item
  No, but I am familiar with what is taught at a workshop.
\item
  No, and I am not familiar with what is taught at a workshop.
\end{enumerate}

\emph{Which of these most accurately describes your teaching
experience?}

\begin{enumerate}
\def\labelenumi{\arabic{enumi}.}
\itemsep1pt\parskip0pt\parsep0pt
\item
  I have been a graduate or undergraduate teaching assistant for a
  university/college course.
\item
  I have not had any teaching experience in the past.
\item
  I have taught a seminar, workshop, or other short or informal course.
\item
  I have been the instructor-of-record for my own university/college
  course.
\item
  I have taught at the K-12 level.
\end{enumerate}

\emph{Which of these questions assesses flaws in a student's mental
model of a domain?}

\begin{enumerate}
\def\labelenumi{\arabic{enumi}.}
\itemsep1pt\parskip0pt\parsep0pt
\item
  I'm not sure what a mental model is.
\item
  ``In Python, what is the expected output for the following statement:
  1 + `2'\,''
\end{enumerate}

\begin{enumerate}
\def\labelenumi{(\alph{enumi})}
\itemsep1pt\parskip0pt\parsep0pt
\item
  `12'
\item
  TypeError
\item
  `3'
\item
  3
\end{enumerate}

\begin{enumerate}
\def\labelenumi{\arabic{enumi}.}
\setcounter{enumi}{2}
\itemsep1pt\parskip0pt\parsep0pt
\item
  ``Rate your experience with the R programming language.''
\end{enumerate}

\begin{enumerate}
\def\labelenumi{(\alph{enumi})}
\itemsep1pt\parskip0pt\parsep0pt
\item
  never used it
\item
  beginner
\item
  intermediate
\item
  expert
\end{enumerate}

\begin{enumerate}
\def\labelenumi{\arabic{enumi}.}
\setcounter{enumi}{3}
\itemsep1pt\parskip0pt\parsep0pt
\item
  ``What does the Unix command `cut' do?''
\end{enumerate}

\begin{enumerate}
\def\labelenumi{(\alph{enumi})}
\itemsep1pt\parskip0pt\parsep0pt
\item
  Extracts sections from each line of input.
\item
  Sorts fields of a line
\item
  Searches the input file for lines containing a match to a pattern
\item
  Removes a given input from a line
\end{enumerate}

\end{discussion}

Now that we have a better idea of everyone's prior knowledge and
familiarity with some of the SWC and DC teaching practices, we're ready
to begin our training. Our goal is that by the end, you will have
acquired some new knowledge, confidence, and skills that you can use in
your teaching practice in general and in teaching SWC and DC workshops
specifically.

Our first topic will be cognitive development and mental models, which
will lead into the theory and practice behind multiple choice questions,
like the ones you just answered.
