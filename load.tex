\chaplbltime{Cognitive Load}{sec:load}{20}{20}

In 2006, Kirschner, Sweller, and Clark published a paper titled ``Why
Minimal Guidance During Instruction Does Not Work: An Analysis of the
Failure of Constructivist, Discovery, Problem-Based, Experiential, and
Inquiry-Based Teaching'' \cite{bib:kirschner-minimal}. Its abstract
says:

\begin{quote}

  Although unguided or minimally guided instructional approaches are
  very popular and intuitively appealing\ldots{}these approaches
  ignore both the structures that constitute human cognitive
  architecture and evidence from empirical studies over the past
  half-century that consistently indicate that minimally guided
  instruction is less effective and less efficient than instructional
  approaches that place a strong emphasis on guidance of the student
  learning process. The advantage of guidance begins to recede only
  when learners have sufficiently high prior knowledge to provide
  ``internal'' guidance.

\end{quote}

The paper set off a minor academic firestorm, because beneath the jargon
the authors were claiming that
\href{https://en.wikipedia.org/wiki/Inquiry-based\_learning}{inquiry-based
learning}---allowing learners to ask their own questions, set their
own goals, and find their own path through a subject---is intuitively
appealing, but doesn't actually work very well. Kirschner et al argued
that this was because it requires learners to simultaneously master
both a domain's factual content and its problem-solving strategies.

More specifically, the theory of
\emph{\href{https://en.wikipedia.org/wiki/Cognitive\_load}{cognitive load}}
posits that people have to deal with three things when they're
learning:

\begin{itemize}

\item
  \emph{Intrinsic} load is what people have to keep in mind in order to
  carry out a learning task.  In a programming class, this might be
  understanding what a variable is, or understanding how assignment in
  a programming language is different from creating a reference to a
  cell in a spreadsheet.

\item
  \emph{Germane} load is the (desirable) mental effort required to
  create linkages between new information and old (which is one of the
  things that distinguishes learning from memorization).  An example
  might be learning how to loop through a collection in Python.

\item
  \emph{Extraneous} load is everything else that distracts or gets in
  the way, such as knowing that tabs look like multiple characters but
  only count as one when indenting Python code.

\end{itemize}

According to this theory, searching for a solution strategy is an
extra burden on top of applying that strategy. We can therefore
accelerate learning by giving learners worked examples that show them
a problem and a detailed step-by-step solution, followed by a series
of \emph{faded examples}. The first of these presents a
nearly-complete use of the same problem-solving strategy just
demonstrated, but with a small number of blanks for the learner to
fill in. The next problem is also of the same type, but has more
blanks, and so on until the learner is asked to solve the entire
problem.

For example, someone teaching Python might start by explaining this:

\begin{verbatim}
# total_length(["red", "green", "blue"]) => 12
def total_length(words):
    total = 0
    for word in words:
        total += len(word)
    return total
\end{verbatim}

\noindent
then ask learners to fill in the blanks in:

\begin{verbatim}
# word_lengths(["red", "green", "blue"]) => [3, 5, 4]
def word_lengths(words):
    lengths = ____
    for word in words:
        lengths ____
    return lengths
\end{verbatim}

The next problem might be:

\begin{verbatim}
# concatenate_all(["red", "green", "blue"]) => "redgreenblue"
def concatenate_all(words):
    result = ____
    for ____ in ____:
        ____
    return result
\end{verbatim}

\noindent
and learners would finally be asked to tackle:

\begin{verbatim}
# acronymize(["red", "green", "blue"]) => "RGB"
def acronymize(words):
    ____
\end{verbatim}

Faded examples work because they introduce the problem-solving
strategy piece by piece. At each step, learners have one new problem
to tackle.  This is less intimidating than a blank screen or a blank
sheet of paper.  It also encourages learners to think about the
similarities and differences between various approaches, which helps
create the linkages in the mental model that instructors want them to
form.

The key to constructing a good faded example is to think about the
problem-solving strategy or solution pattern that it is meant to
teach.  For example, the series of problems are all examples of the
\emph{accumulator pattern}, in which the results of processing items
from a collection are repeatedly added to a single variable in some way
to create the final result.

Cognitive load theory has been criticized as being
\href{https://edtechdev.wordpress.com/2009/11/16/cognitive-load-theory-failure/}{unfalsifiable}:
since there's no way to tell in advance of an experiment whether
something is germane or not, any result can be justified after the
fact by labelling things that hurt performance as ``extraneous'' and
things that don't ``germane''.  However, there is no doubt that faded
examples are effective.

\begin{callout}{Split Attention}{callout:split-attention}

Research by Mayer and colleagues on the
\emph{\href{https://en.wikipedia.org/wiki/Split\_attention\_effect}{split-attention
effect}} is closely related to cognitive load theory
\cite{bib:mayer-nine-ways}.  Linguistic and visual input are
processed by different parts of the human brain, and linguistic and
visual memories are stored separately as well. This means that
correlating linguistic, auditory, and visual streams of information
takes cognitive effort: when someone reads something while hearing it
spoken aloud, their brain can't help but check that it's getting the
same information on both channels.

Learning is therefore more effective when redundant information is
\emph{not} being presented simultaneously in two different channels. For
example, people find it harder to learn from a video that has both
narration and on-screen captions than from one that has either the
narration or the captions but not both.

This is also why it's more effective to draw a diagram piece by piece
while teaching rather than presenting the whole thing at once.  If
parts of the diagram appear at the same time as things are being said,
the two will be correlated in the learner's memory, so that pointing
at part of the diagram will trigger recall of what was being said.

\end{callout}

Another way to use cognitive load theory to construct exercises is
called a \emph{Parson's Problem}.  If you are teaching someone to
speak a new language, you could ask them a question, and then give
them the words they need to answer the question, but in jumbled
order.  Their task is to put the words in the right order to answer
the question grammatically, which frees them from having to think
simultaneously about what to say \emph{and} how to say it.

Similarly, when teaching people to program, you can give them the
lines of code they need to sovle a problem, and ask them to put them
in the right order.  This allows them to concentrate on control flow
and data dependencies, i.e., on what has to happen before what,
without being distracted by variable naming or trying to remember what
functions to call.

\begin{challenge}{Create a Faded Example}{chal:create-a-faded-example}

It's very common for programs to count how many things fall into
different categories: for example, how many times different colors
appear in an image, or how many times different words appear in a
paragraph of text.

\begin{enumerate}

\item
  Create a short example (no more than 10 lines of code) that shows
  people how to do this, and then create a second example that solves
  a similar problem in a similar way, but has a couple of blanks for
  learners to fill in.  How did you decide what to fade out?  What
  would the next example in the series be?

\item
  Define the audience for your examples. For example, are these
  beginners who only know some basics programming concepts? Or are
  these learners with some experience in programming but not in
  Python?

\item
  Show your example to a partner, but do \emph{not} tell them what
  level it is intended for.  Once they have filled in the blanks, ask
  them what level they think it is for.

\end{enumerate}

If there are people among the trainees who don't program at all, make
sure that they are in separate groups and ask to the groups to work
with that person as a learner to help identify different loads.

\end{challenge}

\begin{challenge}{Create a Parson's Problem}{chal:parsons-problem}

\begin{enumerate}

\item
  Write five or six lines of code that does something useful, jumble
  them, and ask your partner to put them in order.  If you are using
  an indentation-based language like Python, do not indent any of the
  lines; if you are using a curly-brace language like Java, do not
  include any of the curly braces.

\item
  Create a second example similar to the first, but include one line
  of code that isn't needed to solve the problem.  How much harder is
  it for your partner to put things in order when there's unneeded
  material getting in the way?

\end{enumerate}

\end{challenge}
