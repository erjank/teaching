\chaplbl{Cognitive Load}{s:load}

In our final topic in educational psychology, we'll be learning more
about human memory: specifically how to remove unnecessary ``load'' in
order to facilitate learning.

\subsection{Battling Theories}\label{battling-theories}

In 2006, Kirschner, Sweller, and Clark published a paper titled
``\href{http://www.cogtech.usc.edu/publications/kirschner\_Sweller\_Clark.pdf}{Why
Minimal Guidance During Instruction Does Not Work: An Analysis of the
Failure of Constructivist, Discovery, Problem-Based, Experiential, and
Inquiry-Based Teaching}''. In the abstract, they say:

\begin{quote}
Although unguided or minimally guided instructional approaches are very
popular and intuitively appealing\ldots{}these approaches ignore both
the structures that constitute human cognitive architecture and evidence
from empirical studies over the past half-century that consistently
indicate that minimally guided instruction is less effective and less
efficient than instructional approaches that place a strong emphasis on
guidance of the student learning process. The advantage of guidance
begins to recede only when learners have sufficiently high prior
knowledge to provide ``internal'' guidance.
\end{quote}

The paper set off a minor academic firestorm, because beneath the jargon
the authors were claiming that
\href{https://en.wikipedia.org/wiki/Inquiry-based\_learning}{inquiry-based
learning}---allowing learners to ask their own questions, set their own
goals, and find their own path through a subject---doesn't actually work
very well. Kirschner et al's argument was that it requires learners to
simultaneously master a domain's factual content and its search and
problem-solving strategies. Fostering creativity and independence is
intuitively appealing, but that doesn't mean it works.

One of the paper's authors (Sweller) proposed an alternative based on
the theory of
\emph{\href{https://en.wikipedia.org/wiki/Cognitive\_load}{cognitive
load}}. It posits that people have to deal with three things when
they're learning:

\begin{itemize}
\itemsep1pt\parskip0pt\parsep0pt
\item
  \emph{Intrinsic} load is what they have to keep in mind in order to
  carry out a learning task.
\item
  \emph{Germane} load is the (desirable) mental effort required to
  create linkages between new information and old (which is one of the
  things that distinguishes learning from memorization).
\item
  \emph{Extraneous} load is everything else that distracts or gets in
  the way.
\end{itemize}

Cognitive load theory's proponents claim that eliminating extraneous
cognitive load accelerates learning. Unsurprisingly, it too has
\href{https://edtechdev.wordpress.com/2009/11/16/cognitive-load-theory-failure/}{been
criticized}, most particularly for being unfalsifiable. Critics of
cognitive load theory say that since there's no way to tell in advance
of an experiment whether something is germane or not, any result can be
justified after the fact by labelling things that hurt performance as
``extraneous'' and things that don't ``germane''.

However, some predictions \emph{can} be made. One example is work by
Mayer and colleagues on the
\emph{\href{https://en.wikipedia.org/wiki/Split\_attention\_effect}{split-attention
effect}}. Linguistic and visual input are processed by different parts
of the human brain, and linguistic and visual memories are stored
separately as well. This means that correlating linguistic, auditory,
and visual streams of information takes cognitive effort: when someone
reads something while hearing it spoken aloud, their brain can't help
but check that it's getting the same information on both channels.
Learning is therefore more effective when redundant information is
\emph{not} being presented simultaneously in two different channels. For
example, people find it harder to learn from a video that has both
narration and on-screen captions than from one that has either the
narration or the captions but not both.

\subsection{Faded Examples}\label{faded-examples}

According to cognitive load theory, searching for a solution strategy is
an extra burden on top of applying that strategy. We can therefore
accelerate learning by giving learners worked examples that show them a
problem and a detailed step-by-step solution, followed by a series of
\emph{faded examples}. The first of these presents a nearly-complete use
of the same problem-solving strategy just demonstrated with a small
number of blanks for the learner to fill in. The next problem is also of
the same type, but has more blanks, and so on until the learner is asked
to solve the entire problem.

Faded examples work because they introduce the problem-solving strategy
piece by piece. At each step, learners have one new problem to tackle.
This is less intimidating than a blank screen or a blank sheet of paper.
It also encourages learners to think about the similarities and
differences between various approaches, which helps create the linkages
in the mental model that instructors want them to form.

For example, someone teaching Python might start by explaining this:

\begin{verbatim}
# total\_length(["red", "green", "blue"]) => 12
def total\_length(words):
    total = 0
    for word in words:
        total += len(word)
    return total
\end{verbatim}

\{: .source\}

then ask learners to fill in the blanks in:

\begin{verbatim}
# word\_lengths(["red", "green", "blue"]) => [3, 5, 4]
def word\_lengths(words):
    lengths = \_\_\_\_
    for word in words:
        lengths \_\_\_\_
    return lengths
\end{verbatim}

\{: .source\}

The next problem might be:

\begin{verbatim}
# concatenate\_all(["red", "green", "blue"]) => "redgreenblue"
def concatenate\_all(words):
    result = \_\_\_\_
    for \_\_\_\_ in \_\_\_\_:
        \_\_\_\_
    return result
\end{verbatim}

\{: .source\}

and learners would finally be asked to tackle:

\begin{verbatim}
# acronymize(["red", "green", "blue"]) => "RGB"
def acronymize(words):
    \_\_\_\_
\end{verbatim}

\{: .source\}

The key to constructing a good faded example is to think about the
problem-solving strategy or solution pattern that it is meant to teach.
For example, the series of problems above illustrate the
\emph{accumulator pattern}, in which the results of processing items
from a collection are repeatedly added to a single variable in some way
to create the final result.

\begin{quote}
\subsection{Create a Faded Example from a
Lesson}\label{create-a-faded-example-from-a-lesson}

The following exercise should be done in groups of 2-3.

\begin{enumerate}
\def\labelenumi{\arabic{enumi}.}
\itemsep1pt\parskip0pt\parsep0pt
\item
  Pick a block of code from an existing Software or Data Carpentry
  lesson, or from another lesson you have taught recently.
\item
  Replace 2-3 pieces of the code with a blank.
\item
  Write a question to test the student's ability to correctly fill in
  that blank.
\item
  Paste your faded example in the Etherpad. \{: .challenge\}
\end{enumerate}
\end{quote}

\subsection{Parsons Problems}\label{parsons-problems}

Another kind of exercise designed to reduce cognitive load is a
\emph{Parsons Problem}, in which learners are presented with the jumbled
parts of a solution and asked to put them in order. When learning a
language, for example, students could be asked to order a set of words
to create a grammatically correct response to a question. Similarly, our
learners can be given the lines of code needed to solve a problem and
asked to arrange them. Learners can then be told that they have all the
lines they need save one, and so on.

Here is a really nice online Parsons Problem interactive tool.
\href{http://runestoneinteractive.org/LearningAtScale/parsons.html}{Try
it out!}

\begin{quote}
\subsection{Parsons Problems}\label{parsons-problems-1}

Write 5 or 6 lines of code that does something useful, jumble them, then
add one more line that looks plausible but isn't needed to solve the
problem. How well can your partner tell which line is unnecessary? \{:
.challenge\}
\end{quote}
