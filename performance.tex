\chaplbltime{Teaching as a Performance Art}{sec:performance}{20}{45}

Many people assume that teachers are born, not made. From politicians
to researchers and teachers themselves, reformers have designed
systems to find and promote those who can teach and eliminate those
who can't. But as Elizabeth Green explains in
\emph{Building a Better Teacher} \cite{bib:green-babt}, that
assumption is wrong, which is why educational reforms based on it
have repeatedly failed.

The book is written as a history of the people who have put that puzzle
together in the US. Its core begins with a discussion of what James
Stigler discovered during a visit to Japan in the early 1990s:

\begin{quote}

  Some American teachers called their pattern ``I, We, You'': After
  checking homework, teachers announced the day's topic, demonstrating
  a new procedure (I){\ldots} Then they led the class in trying out a
  sample problem together (We){\ldots} Finally, they let students work
  through similar problems on their own, usually by silently making
  their way through a worksheet (You){\ldots}

  The Japanese teachers, meanwhile, turned ``I, We, You'' inside
  out. You might call their version ``You, Y'all, We.'' They began not
  with an introduction, but a single problem that students spent ten
  or twenty minutes working through alone (You){\ldots} While the
  students worked, the teacher wove through the students' desks,
  studying what they came up with and taking notes to remember who had
  which idea. Sometimes the teacher then deployed the students to
  discuss the problem in small groups (Y'all). Next, the teacher
  brought them back to the whole group, asking students to present
  their different ideas for how to solve the problem on the
  chalkboard{\ldots} Finally, the teacher led a discussion, guiding
  students to a shared conclusion (We).

\end{quote}

It's tempting but wrong to think that this particular teaching
technique is some kind of secret sauce. The actual key is revealed in
the description of Akihiko Takahashi's work. In 1991, he visited the
United States in a vain attempt to find the classrooms described a
decade earlier in a report by the National Council of Teachers of
Mathematics. He couldn't find them. Instead, he found that American
teachers met once a year (if that) to exchange ideas about teaching,
compared to the weekly or even daily meetings he was used to. What was
worse:

\begin{quote}

  The teachers described lessons they gave and things students said,
  but they did not \emph{see} the practices. When it came to observing
  actual lessons---watching each other teach---they simply had no
  opportunity{\ldots} They had, he realized, no \emph{jugyokenkyu}.
  Translated literally as ``lesson study'', \emph{jugyokenkyu} is a
  bucket of practices that Japanese teachers use to hone their craft,
  from observing each other at work to discussing the lesson afterward
  to studying curriculum materials with colleagues. The practice is so
  pervasive in Japanese schools that it is{\ldots}effectively
  invisible.

  And here lay the answer to {[}Akihiko's{]} puzzle. Of course the
  American teachers' work fell short of the model set by their best
  thinkers{\ldots} Without \emph{jugyokenkyu}, his own classes would
  have been equally drab. Without \emph{jugyokenkyu}, how could you
  even teach?

\end{quote}

So what does \emph{jugyokenkyu} look like in practice?

\begin{quote}

  In order to graduate, education majors not only had to watch their
  assigned master teacher work, they had to effectively replace him,
  installing themselves in his classroom first as observers and then,
  by the third week, as a wobbly{\ldots}approximation of the teacher
  himself.  It worked like a kind of teaching relay. Each trainee took
  a subject, planning five days' worth of lessons{\ldots} {[}and
  then{]} each took a day. To pass the baton, you had to teach a day's
  lesson in every single subject: the one you planned and the four you
  did not{\ldots} and you had to do it right under your master
  teacher's nose. Afterward, everyone---the teacher, the college
  students, and sometimes even another outside observer---would sit
  around a formal table to talk about what they saw.

  {[}Trainees{]} stayed in{\ldots}class until the students
  left{\ldots} They talked about what {[}the master teacher{]} had
  done, but they spent more time poring over how the students had
  responded: what they wrote in their notes; the ideas they came up
  with, right and wrong; the architecture of the group
  discussion{\ldots}

  {\ldots}By the time he arrived in {[}the US{]}, {[}Akihiko had{]}
  become{\ldots}famous{\ldots} giving public lessons that attracted
  hundreds, and, in one case, an audience of a thousand. He had a
  seemingly magical effect on children{\ldots} But Akihiko knew he was
  no virtuoso. ``It is not only me,'' he always said{\ldots}
  ``\emph{Many} people.'' After all, it was his mentor{\ldots}who had
  taught him the new approach to teaching{\ldots} And {[}he{]} had
  crafted the approach along with the other math teachers in {[}his
  ward{]} and beyond. Together, the group met regularly to discuss
  their plans for teaching{\ldots} {[}At{]} the end of a discussion,
  they'd invite each other to their classrooms to study the
  results. In retrospect, this was the most important lesson: not how
  to give a lesson, but how to study teaching, using the cycle
  of \emph{jugyokenkyu} to put{\ldots}work under a microscope and
  improve it.

\end{quote}

Putting work under a microscope in order to improve it is commonplace
in sports and music.  A professional musician, for example, will
dissect half a dozen different recordings of ``Body and Soul'' or
``Smells Like Teen Spirit'' before performing it. They would also
expect to get feedback from fellow musicians during practice and after
performances.  Many other disciplines work this way too: the Japanese
drew inspiration from
\href{https://en.wikipedia.org/wiki/W.\_Edwards\_Deming}{Deming}'s
ideas on continuous improvement in manufacturing, while the adoption
of code review over the last 15 years has done more to improve
everyday programming than any number of books or websites.

But this kind of feedback isn't part of teaching culture in the US,
the UK, Canada, or Australia.  There, what happens in the classroom
stays in the classroom: teachers don't watch each other's lessons on a
regular basis, so they can't borrow each other's good ideas. The
result is that \emph{every teacher has to invent teaching on their
own}. They may get lesson plans and assignments from colleagues, the
school board, a textbook publisher, or the Internet, but each teacher
has to figure out on their own how to combine that with the theory
they've learned in education school to deliver an actual lesson in an
actual classroom for actual students.

Demonstration lessons, in which one teacher is in front of a room full
of students while other teachers observe, seem like a way to solve
this.  However, Fincher and her colleagues studied how teaching
practices are actually transferred using both a detailed case
study \cite{bib:fincher-warrens-questions} and analysis of change
stories \cite{bib:fincher-stories-change}.  The abstract of the latter
paper sums up their findings:

\begin{quote}

  Innovative tools and teaching practices often fail to be adopted by
  educators in the field, despite evidence of their effectiveness.
  Na\:{i}ve models of educational change assume this lack of adoption
  arises from failure to properly disseminate promising work, but
  evidence suggests that dissemination via publication is simply not
  effective{\ldots} We asked educators to describe changes they had
  made to their teaching practice and analyzed the resulting
  stories{\ldots} Of the 99 change stories analyzed, only three
  demonstrate an active search for new practices or materials on the
  part of teachers, and published materials were consulted in just
  eight of the stories. Most of the changes occurred locally, without
  input from outside sources, or involved only personal interaction
  with other educators.

\end{quote}

Barker et al found something similar \cite{bib:barker-practice-adoption}:

\begin{quote}

  Adoption is not a ``rational action,'' however, but an iterative
  series of decisions made in a social context, relying on normative
  traditions, social cueing, and emotional or intuitive
  processes{\ldots} Faculty are not likely to use educational research
  findings as the basis for adoption decisions. Faculty become aware
  of innovative practices either because a problem leads them to
  intentionally seek them out, or they hear about them through funded
  initiatives, conferences and journals, or from colleagues. They
  experiment (or not) for several reasons, depending on institutional
  expectations and policies, perceived costs and benefits for
  themselves and students, and the influence of role models. Faculty
  tend to trust other faculty whose work and institutional context is
  more like their own. The choice to try out practices competes with
  the need to ``cover'' material, as well as with classroom
  layouts. Positive student feedback is taken as strong evidence by
  faculty that they should continue a practice.

\end{quote}

This phenomenon is sometimes called \emph{lateral knowledge transfer}:
someone sets out to teach X, but while watching them, their audience
actually learns Y as well (or instead). For example, an instructor
might set out to show people how to do a particular statistical
analysis in R, but what her learners might take away is some new
keyboard shortcuts in R Studio. Live coding makes this much more
likely because it allows learners to see the ``how'' as well as the
``what''.

\fixme{how jugyokenkyu figures into our teaching}

\seclbl{Feedback}{sec:feedback}

\figlbl{Feedback Feelings}{fig:jerk}{fig/deathbulge-jerk.jpg}

As \figref{fig:jerk} suggests, sometimes it can be hard to receive
feedback, especially negative feedback.  The process is easier and
more productive when the people involved share ground rules and
expectations. This is especially important when they have different
backgrounds or cultural expectations about what's appropriate to say
and what isn't.

There are several things you can do to make feedback more effective:

\begin{genumerate}

\item
  \emph{Initiate feedback.} It's better to ask for feedback than to
  receive it unwillingly.

\item
  \emph{Choose your own questions}, i.e., ask for specific
  feedback. It's a lot harder for someone to answer, ``What do you
  think?'' than to answer either, ``What is one thing I could have
  done as an instructor to make this lesson more effective?''  or ``If
  you could pick one thing from the lesson to go over again, what
  would it be?''

  Directing feedback like this is also more helpful to you.  It's
  always better to try to fix one thing at once than to change
  everything and hope it's for the better.  Directing feedback at
  something you have chosen to work on helps you stay focused, which
  in turn increases the odds that you'll see progress.

\item
  \emph{Balance positive and negative feedback.}  One method is a
  ``compliment sandwiches'' made up of one positive, one negative, and
  a second positive observation.  Another (which we discuss below) is
  to ask for at least one point in each of several categories.

\item
  Use a feedback translator. Have a fellow instructor (or other
  trusted person in the room) read over all the feedback and give an
  executive summary. It can be easier to hear ``It sounds like most
  people are following, so you could speed up'' than to read several
  notes all saying, ``this is too slow'' or ``this is boring''.

\item
  Most importantly, \emph{be kind to yourself}.  Many of us are very
  critical of ourselves, so it's always helpful to jot down what we
  thought of ourselves \emph{before} getting feedback from others,
  so that \fixme{normalize}.

\end{genumerate}

The technique we find most useful for giving feedback is to create a
$2{\times}2$ grid and put each piece of feedback in one of its four
squares.  The vertical axis divides positive from negative; the
horizontal divides content from presentation, i.e., what was said from
how it was said.  Even this little bit of structure helps people
figure out what to say, and to separate those who have good ideas that
they can't communicate from those who are eloquent but don't actually
have anything to say.

\begin{callout}{Tells}{callout:tells}

Everyone has nervous habits. For example, many of us become ``Mickey
Mouse'' versions of ourselves when we're nervous, i.e., we talk more
rapidly than usual, in a higher-pitched voice, and wave our arms
around more than we usually would.

Gamblers call nervous habits like this ``tells''.  While these are
often not as noticeable as you would think, it's good to identify ways
to keep yourself from pacing, or fiddling with your jewellery, or not
looking at the audience.

\end{callout}

If you are interested in knowing more about giving and getting
feedback, you may want to read Gormally et al's ``Feedback about
Teaching in Higher Ed'' \cite{bib:gormally-teaching-feedback} and
discuss ways you could make peer-to-peer feedback a routine part of
your teaching.  You may also enjoy Gawande's essay ``Personal Best''
\cite{bib:gawande-personal-best}, which looks at the value of having a
coach.

\seclbl{Challenges}{sec:performance-challenges}

\begin{challenge}{Giving Feedback}{chal:giving-feedback}

\begin{genumerate}

\item
  Watch \href{https://www.youtube.com/watch?v=-ApVt04rB4U}{this video}
  as a group and then give feedback on it. Try to organize feedback
  along two axes: positive vs.\ negative and content vs.\
  presentation.

\item
  \fixme{add all feedback to the whiteboard}

\end{genumerate}

\end{challenge}

\begin{challenge}{Feedback on Your Teaching}{chal:feedback}

\begin{genumerate}

\item
  Split into groups of three.

\item
  Have each person introduce themselves and then explain, in no more
  than 90 seconds, the key idea or ideas from the Carpentry lesson
  episode they chose before the start of the training course to another
  person in the group while the third person records it (video and
  audio) using a cell phone or some other handheld device.

\item
  After the first person finishes, rotate roles (she becomes the
  videographer, her audience becomes the instructor, the person who was
  recording becomes the audience) and then rotate roles again.

\item
  After everyone in the group of three has finished teaching, watch the
  videos as a group. Everyone gives feedback on all three videos, i.e.,
  people give feedback on themselves as well as on others.

\item
  After everyone has given feedback on all of the videos, return to the
  main group and put all of the feeback into the notes.  Again, try to
  divide positive from negative and content from presentation.  Try
  also to identify each person's tells: what do they do that betrays
  nervousness, and how noticeable is it?

\end{genumerate}

\end{challenge}
