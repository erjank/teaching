\chaplbl{References}{s:ref}

\subsection{Top Ten}\label{top-ten}

\begin{enumerate}
\def\labelenumi{\arabic{enumi}.}
\itemsep1pt\parskip0pt\parsep0pt
\item
  Be kind: all else is details.
\item
  Never teach alone.
\item
  No lesson survives first contact with learners.
\item
  Nobody will be more excited about the lesson than you are.
\item
  Every lesson is too short from the teacher's point of view and too
  long from the learner's.
\item
  Never hesitate to sacrifice truth for clarity.
\item
  Every mistake is a lesson.
\item
  ``I learned this a long time ago'' is not the same as ``this is
  easy''.
\item
  You learn with your learners.
\item
  You can't help everyone, but you can always help someone.
\end{enumerate}

\subsection{A Few Other Things}\label{a-few-other-things}

\begin{enumerate}
\def\labelenumi{\arabic{enumi}.}
\itemsep1pt\parskip0pt\parsep0pt
\item
  Everyone in our community is required to abide by our
  \href{\{\{\%20site.swc_site\%20\}\}/conduct/}{Code of Conduct}, both
  at workshops and online, to ensure that everyone else feels welcome.
\item
  You teach our material, not your own, and you need to work through the
  materials before teaching to verify your own understanding and figure
  out where people might have trouble.
\item
  We organize some workshops, but we expect people to organize workshops
  locally as well. You can charge people to attend, and we will charge
  you when we help organize.
\item
  We expect you to teach at least once within a year of certifying in
  exchange for this training.
\item
  Expect a broad range of expertise and experience, and be prepared to
  adapt your teaching to accommodate beginners or those who struggle.
\item
  We use live coding instead of slides: instructors work through the
  lesson material, typing in the code or instructions, while the
  learners follow along.
\item
  Use sticky notes for real-time feedback and minute cards or ``one up,
  one down'' at lunch and at the end of the day in order to find out how
  the class is going while there's still time to fix things.
\item
  The ``I don't know what I'm doing'' feeling never goes away. You just
  learn the ``but I can figure it out'' part. --
  \href{https://twitter.com/sciencegurlz0/status/687739023826235393}{Sciencegurl}
\end{enumerate}

A note on \#2: some instructors start improvising after they've taught
the core lessons as-is a few times, but you should know what you're
improvising around---remember, our materials have been used hundreds of
times, and probably address problems you don't yet know will arise.

\subsubsection{What Kinds of Practice and Feedback Enhance
Learning?}\label{what-kinds-of-practice-and-feedback-enhance-learning}

\begin{enumerate}
\def\labelenumi{\arabic{enumi}.}
\itemsep1pt\parskip0pt\parsep0pt
\item
  Mismatched expectations can be difficult to diagnose and waste much
  time.
\item
  In the absence of structure, learners tend to glide along more
  comfortable paths (i.e., making slides prettier rather than more
  content-rich).
\item
  Deliberate practice without effective feedback can instill new unknown
  bad habits.
\item
  Learning and performance are best fostered when students engage in
  practice that:

  \begin{enumerate}
  \def\labelenumii{\arabic{enumii}.}
  \itemsep1pt\parskip0pt\parsep0pt
  \item
    focuses on a specific goal or criterion for performance,
  \item
    targets an appropriate level of challenge relative to students'
    current performance, and
  \item
    is of sufficient quantity and frequency to meet the performance
    criteria.
  \end{enumerate}
\item
  Articulate goals in measurable ways:

  \begin{enumerate}
  \def\labelenumii{\arabic{enumii}.}
  \itemsep1pt\parskip0pt\parsep0pt
  \item
    Use good metrics that relate to the objective and to possible
    performance from the learner.
  \item
    Include higher-level goals.
  \end{enumerate}
\item
  Concurrent learning can work, but often not at the novice skill level.
\item
  While quality of practice matters, time on task is also important.
\item
  Practice tends to be most effective at improving skills in the
  ``competent'' range. (Novices grapple with known knowns, competent
  practitioners with known unknowns, and experts with unknown unknowns.)
\item
  Instructors should point out progress as it is made so that students
  recognize their accomplishment and discern the change in their
  behavior, especially when gradual.
\item
  Grades and scores provide some information on the degree to which
  students' performance has met the criteria, they do not explain which
  aspects did or did not meet the criteria and how, so more specific
  feedback is necessary.
\end{enumerate}

\subsubsection{How Does Students' Prior Knowledge Affect Their
Learning?}\label{how-does-students-prior-knowledge-affect-their-learning}

\begin{enumerate}
\def\labelenumi{\arabic{enumi}.}
\itemsep1pt\parskip0pt\parsep0pt
\item
  Learners come with past experiences and models of knowledge. If we can
  activate that prior knowledge and correctly link it to what we are
  trying to teach, the effect will be increased retention and a greater
  ability to apply what we are teaching to novel problems.
\item
  If the learner's past knowledge is not activated, we lose this
  integration of knowledge and the amplifying effect of their past
  experience and their declarative and procedural knowledge.
\item
  If the learner comes with incorrect information or misunderstands how
  the new knowledge relates to their past experience, their learning can
  be hindered until they understand the misconception.
\item
  The nature of misconceptions is that the learner will not realize they
  have them. Specifically, if asked they may well report that they
  understand the situation.
\item
  Testing knowledge (e.g., doing diagnostic assessment with
  well-designed multiple choice questions) will reveal the
  misconceptions and form a basis for correcting them.
\item
  Well-crafted challenges will provide the learner with information
  about their understanding. Faded examples support the student and
  provide a ``win'' at the start and indicate a lack when they stop
  being able to complete the challenge.
\item
  Successfully completing a challenge while still holding a
  misconception about the subject of the challenge is a very bad thing,
  because it increases the learners' confidence in their incorrect
  model.
\item
  Using examples that involve universal activities rather than domain
  specific or highly technical examples will maximize the number of
  correct connections that form the basis of transferring knowledge. It
  is a delicate balance: if the problem is too simple, students may
  dismiss it as unimportant or and switch off to conserve energy because
  they believe they already understand. Some humour or an interesting
  story will allow you to keep engagement while speaking directly to
  most people's experiences.
\item
  Analogies are useful in connecting past understanding to a current
  problem, but be explicit about how it applies to the situation because
  the learner may not understand where the analogy breaks down or stops
  being applicable.
\end{enumerate}

\subsubsection{Why Do Student Development and Course Climate Matter for
Student
Learning?}\label{why-do-student-development-and-course-climate-matter-for-student-learning}

\begin{enumerate}
\def\labelenumi{\arabic{enumi}.}
\itemsep1pt\parskip0pt\parsep0pt
\item
  Make uncertainty safe: support students who are uncomfortable with
  ambiguity (i.e.~there are various solutions to an answer).
\item
  Resist a single right answer: acknowledge that we are teaching them
  one way to do things, but there are many many tools that could be used
  to do the same thing (version control, visualization, etc).
\item
  Examine your assumptions about students: don't expect people to be
  able (or unable) to do a specific task based on their race, gender,
  age, experience).
\item
  Reduce anonymity: try to remember names (or encourage learners to wear
  their badges), provide opportunity for learners to interact during
  breaks, and as instructor, remember to interact as well rather than
  reading emails during break and looking aloof.
\item
  Establish and reinforce ground rules for interaction: refer to the
  code of conduct and have a plan of action for when the code of conduct
  is breached.
\item
  Use the syllabus and first day of class to establish the course
  climate: sticking to the published agenda and course timings is one of
  the things that we get the most consistent positive comment about.
  Tell learners what they can expect and keep to what you told them. If
  things have to change, inform them promptly. Once people start to feel
  like they don't know what is going on, it's hard for them to focus on
  learning new skills.
\item
  Set up processes to get feedback: collect sticky notes before lunch
  and before the end of day for two-day workshops and develop another
  strategy for workshops that run over multiple half days or other
  formats. Go through the feedback immediately and act upon the
  suggestions that can be dealt with immediately. A responsive
  instructor gain the trust of learners and make them feel important and
  heard.
\item
  Model inclusive language, behaviour, and attitudes: as instructor, try
  to address all learners equally rather than only talking to the ones
  who follow along nicely or demands more of your attention.
\item
  Be mindful of low-ability cues: (an example from \emph{How Learning
  Works} is, ``I'll be happy to help you with this because I know girls
  have trouble with math.'') Think about how you address your students
  and what stereotypes you are reinforcing unintentionally (e.g., with
  jokes).
\item
  Address tensions early: sometimes workshops have one or two very
  experienced computational people in the room filled otherwise with
  novices. Try to think of ways to keep them engaged by maybe asking
  them to rather act as helpers than learners: they might not like it if
  they have paid to attend a course where they hoped to learn something
  new, so find ways to acknowledge their help as well.
\end{enumerate}

\subsubsection{How do Students Become Self-Directed
Learners?}\label{how-do-students-become-self-directed-learners}

\begin{enumerate}
\def\labelenumi{\arabic{enumi}.}
\itemsep1pt\parskip0pt\parsep0pt
\item
  Learners can become self-directed when they can assess the demands of
  a task, evaluate their own knowledge and skills, plan an approach,
  monitor their own progress, and adjust strategy. Emphasizing these
  steps can help learners on their path to achieving competence.
\item
  Learners come to us with a variety of pre-conceptions on \emph{how to
  learn} and also if they are ``good'' at ``computers'',
  ``programming'', etc. We may need to address unfounded
  pre-conceptions.
\item
  Learners can easily ignore instructors/instructions on how to proceed
  with an exercise (e.g. ``Did they even read the assignment?''). As you
  assign challenges, remind learners about the point/purpose of the
  exercises and get their feedback to confirm they understand the goals.
  \textbf{Be more explicit than you may think necessary.}
\item
  Learners can be (and usually are) poor judges of their knowledge and
  skills.
\item
  Novices spend little time in planning approaches and more time trying
  to find solutions. Emphasize planning as a first-line strategy to
  problem solving.
\item
  Learners will typically continue with strategies that work moderately
  well rather than change to a new strategy that would work better.
\item
  Asking for peer assessment can be a positive and productive learning
  experience when everyone is given criteria to give feedback on.
\item
  Help learners set realistic expectations. Learners should be able
  develop a sense of how long it may take to develop particular skills.
\item
  Discuss metacognition in the classroom. The evidence-based teaching
  style of Software Carpentry is not proprietary or hidden, so share
  what you know about learning in the classroom and why you are taking
  certain approaches.
\item
  Provide heuristics for self-correction. Learners need to develop a
  skill for evaluating their own work. While this skill will take time
  to develop, you can provide `guideposts' for what their code and
  results should look like.
\end{enumerate}

\subsection{Motivational Strategies}\label{motivational-strategies}

\begin{itemize}
\itemsep1pt\parskip0pt\parsep0pt
\item
  Strategies to establish value:

  \begin{enumerate}
  \def\labelenumi{\arabic{enumi}.}
  \itemsep1pt\parskip0pt\parsep0pt
  \item
    Connect the material to students' interests.
  \item
    Provide authentic, real-world tasks.
  \item
    Show relevance to students' current academic lives.
  \item
    Demonstrate the relevance of higher-level skills to students' future
    professional lives.
  \item
    Identify and reward what you value.
  \item
    Show your own passion and enthusiasm for the discipline.
  \end{enumerate}
\item
  Strategies to build positive expectations:

  \begin{enumerate}
  \def\labelenumi{\arabic{enumi}.}
  \itemsep1pt\parskip0pt\parsep0pt
  \item
    Ensure alignment of objectives, assessments, and instructional
    strategies.
  \item
    Identify an appropriate level of challenge.
  \item
    Create assignments that provide the appropriate level of challenge.
  \item
    Provide early success opportunities.
  \item
    Articulate your expectations.
  \item
    Provide rubrics.
  \item
    Provide targeted feedback.
  \item
    Be fair.
  \item
    Educate students about the ways we explain success and failure.
  \item
    Describe effective study strategies.
  \end{enumerate}
\item
  Strategies for self-efficacy:

  \begin{enumerate}
  \def\labelenumi{\arabic{enumi}.}
  \itemsep1pt\parskip0pt\parsep0pt
  \item
    Provide flexibility and control.
  \item
    Give students an opportunity to reflect.
  \end{enumerate}
\end{itemize}

\subsection{Books}\label{books}

\begin{description}
\itemsep1pt\parskip0pt\parsep0pt
\item[Susan Ambrose et al:
\emph{\href{http://www.amazon.com/How-Learning-Works-Research-Based-Jossey-Bass/dp/0470484101/}{How
Learning Works: Seven Research-Based Principles for Smart Teaching}}.]
An excellent overview of what we know about education and why we believe
it's true, covering everything from cognitive psychology to social
factors.
\item[Stephen D. Brookfield and Stephen Preskill: \emph{{[}The
Discussion Book{]}{[}amazon-dicsussion{]}}.]
Describes fifty different ways to get groups talking productively.
\item[Elizabeth Green:
\emph{\href{http://www.amazon.com/Building-Better-Teacher-Teaching-Everyone/dp/0393081591}{Building
a Better Teacher}}.]
A well-written look at why educational reforms in the past 50 years have
mostly missed the mark, and what we should be doing instead.
\item[Mark Guzdial:
\emph{\href{http://www.amazon.com/Learner-Centered-Design-Computing-Education-Human-Centered/dp/1627053514/}{Learner-Centered
Design of Computing Education: Research on Computing for Everyone}}.]
A well-researched investigation of what it means to design computing
courses for everyone, not just people who are going to become
professional programmers, from one of the leading researchers in CS
education.
\item[Doug Lemov:
\emph{\href{http://www.amazon.com/Teach-Like-Champion-2-0-Techniques/dp/1118901851/}{Teach
Like a Champion 2.0}}.]
Presents 62 classroom techniques drawn from intensive study of thousands
of hours of video of good teachers in action.
\item[Therese Huston:
\emph{\href{http://www.amazon.com/Teaching-What-You-Dont-Know/dp/0674066170/}{Teaching
What You Don't Know}}.]
A pointed, funny, and very useful book that explores exactly what the
title suggests.
\item[James Lang:
\emph{\href{https://www.amazon.com/Small-Teaching-Everyday-Lessons-Learning/dp/1118944496/}{Small
Teaching}}.]
A short guide to evidence-based teaching practices that can be adopted
without requiring large up-front investments of time and money.
\item[Jane Margolis and Allan Fisher:
\emph{\href{http://www.amazon.com/Unlocking-Clubhouse-Computing-Jane-Margolis/dp/0262632691/}{Unlocking
the Clubhouse: Women in Computing}}.]
A groundbreaking report on the gender imbalance in computing, and the
steps Carnegie-Mellon took to address the problem.
\item[Claude M. Steele:
\emph{\href{http://www.amazon.com/Whistling-Vivaldi-Stereotypes-Affect-Issues/dp/0393339726/}{Whistling
Vivaldi: How Stereotypes Affect Us and What We Can Do}}.]
Explains and explores stereotype threat and strategies for addressing
it.
\end{description}

\subsection{Papers}\label{papers}

\begin{description}
\itemsep1pt\parskip0pt\parsep0pt
\item[Baume:
``\href{\{\{\%20page.root\%20\}\}/files/papers/baume-learning-outcomes-2009.pdf}{Writing
and Using Good Learning Outcomes}'']
A useful detailed guide to constructing useful learning outcomes.
\item[Borrego and Henderson:
``\href{\{\{\%20page.root\%20\}\}/files/papers/borrego-henderson-change-strategies-2014.pdf}{Increasing
the Use of Evidence-Based Teaching in STEM Higher Education: A
Comparison of Eight Change Strategies}'']
Describes eight approaches to effecting change in STEM education that
form a useful framework for thinking about how Software Carpentry and
Data Carpentry can change the world.
\item[Brown and Altadmri:
``\href{\{\{\%20page.root\%20\}\}/files/papers/brown-educator-vs-learner-beliefs-2014.pdf}{Investigating
Novice Programming Mistakes: Educator Beliefs vs Student Data}'']
Compares teachers' opinions about common programming errors with data
from over 100,000 students, and finds only weak consensus amongst
teachers and between teachers and data.
\item[Carroll, Smith-Kerker, Ford, and Mazur-Rimetz:
``\href{http://dx.doi.org/10.1207/s15327051hci0302_2}{The Minimal
Manual}'' \emph{Human--Computer Interaction}, 3:2, 123-153, 1987.]
Outlines an approach to documentation and instruction in which each
lesson is one page long and describes how to accomplish one concrete
task. Its focus on immediate application, error recognition and
recovery, and reference use after training makes it an interesting model
for Software and Data Carpentry.
\item[Crouch and Mazur:
``\href{\{\{\%20page.root\%20\}\}/files/papers/crouch-mazur-peer-instruction-ten-years-2001.pdf}{Peer
Instruction: Ten Years of Experience and Results}'']
An early report on peer instruction and its effects in the classroom.
\item[Deans for Impact:
``\href{\{\{\%20page.root\%20\}\}/files/papers/science-of-learning-2015.pdf}{The
Science of Learning}'']
Summarizes cognitive science research related to how students learn, and
connects it to practical implications for teaching and learning.
\item[Guzdial:
``\href{\{\{\%20page.root\%20\}\}/files/papers/guzdial-mediacomp-retrospective-2013.pdf}{Exploring
Hypotheses about Media Computation}'']
A look back on 10 years of media computation research.
\item[De Bruyckere et al:
``\href{\{\{\%20page.root\%20\}\}/files/papers/de-bruyckere-urban-myths-2015.pdf}{Urban
Myths About Learning and Education}'']
A one-page summary drawn from their book of the same name.
\item[Gormally et al:
``\href{\{\{\%20page.root\%20\}\}/files/papers/gormally-teaching-feedback-2014.pdf}{Feedback
about Teaching in Higher Ed: Neglected Opportunities to Promote
Change}'']
Summarizes best practices for providing instructional feedback and
recommends specific strategies for sharing instructional expertise.
\item[Guzdial:
``\href{\{\{\%20page.root\%20\}\}/files/papers/guzdial-why-hard-to-teach-2011.pdf}{Why
Programming is Hard to Teach}'']
A chapter from
\emph{\href{http://www.amazon.com/Making-Software-Really-Works-Believe/dp/0596808321/}{Making
Software}} that explores why programming seems so much harder to teach
than some other standard subjects.
\item[Kirschner et al:
``\href{\{\{\%20page.root\%20\}\}/files/papers/kirschner-minimal-guidance-fails-2006.pdf}{Why
Minimal Guidance During Instruction Does Not Work: An Analysis of the
Failure of Constructivist, Discovery, Problem-Based, Experiential, and
Inquiry-Based Teaching}'']
Argues that inquiry-based learning is less effective for novices than
guided instruction.
\item[Lee:
``\href{\{\{\%20page.root\%20\}\}/files/papers/lee-create-inclusive-community-2015.pdf}{What
can I do today to create a more inclusive community in CS?}''.]
A brief, practical guide on exactly that with references to the research
literature.
\item[Mayer and Moreno:
``\href{\{\{\%20page.root\%20\}\}/files/papers/mayer-reduce-cognitive-load-2003.pdf}{Nine
Ways to Reduce Cognitive Load in Multimedia Learning}'']
Shows how research into how we absorb and process information can be
applied to the design of instructional materials.
\item[Porter et al:
``\href{\{\{\%20page.root\%20\}\}/files/papers/porter-what-works-2013.pdf}{Success
in Introductory Programming: What Works?}'']
Summarizes the evidence that three techniques---peer instruction, media
computation, and pair programming---can significantly improve outcomes
in introductory programming courses.
\item[Wiggins and McTighe:
``\href{\{\{\%20page.root\%20\}\}/files/papers/wiggins-mctighe-ubd-nutshell.pdf}{UbD
in a Nutshell}'']
A four-page summary of the authors' take on reverse instructional
design.
\item[Wilson et al: ``\href{https://arxiv.org/abs/1609.00037}{Good
Enough Practices in Scientific Computing}''.]
Describes and justifies a minimal set of computing practices that every
researcher could and should adopt.
\item[Wilson et al:
``\href{http://www.plosbiology.org/article/info\%3Adoi\%2F10.1371\%2Fjournal.pbio.1001745}{Best
Practices for Scientific Computing}'']
Describes and justifies the practices that mature scientific software
developers ought to use.
\item[Wilson:
``\href{http://f1000research.com/articles/3-62/v2}{Software Carpentry:
Lessons Learned}'']
Summarizes what we've learned in 17 years of running classes for
scientists.
\end{description}
