\chaplbltime{Expertise and Memory}{sec:memory}{20}{20}

The previous chapter looked at what distinguishes novices from
competent practitioners.  Here, we will look at expertise: what it is,
how people acquire it, and how it can be harmful as well as helpful.
We will then see how concept maps can be used to figure out how to
turn knowledge into lessons.

To start, what do we mean when we say someone is an expert?  The usual
response is that they can solve problems much faster than people who
are ``merely competent'', or that they can recognize and deal with the
cases where the normal rules don't apply.  They also somehow make this
look effortless: in most cases, they just know what the right answer
is.

What makes someone an expert? The answer isn't that they know more
facts: competent practitioners can memorize a lot of trivia without
any noticeable improvement to their performance.  Instead, imagine for
a moment that we store knowledge as a graph in which facts are nodes
and relationships are arcs. (This is emphatically \emph{not} how our
brains work, but it's a useful metaphor.) The key difference between
experts and people who are ``merely competent'' is that experts have
many more connections, i.e., their mental models are much more densely
connected.

This metaphor helps explain many observed aspects of expert behavior:

\begin{genumerate}

\item
  Experts can jump directly from a problem to its solution because
  there actually is a direct link between the two in their mind.
  Where a competent practitioner would have to reason ``A, B, C, D,
  E'', the expert can go from A to E in a single step. We call
  this \emph{intuition}, and it isn't always a good thing: when asked
  to explain their reasoning, experts often can't, because they didn't
  actually reason their way to the solution---they just recognized it.

\item
  Experts are frequently so familiar with their subject that they can
  no longer imagine what it's like to \emph{not} see the world that
  way. As a result, they are often less good at teaching the subject
  than people with less expertise who still remember what it's like to
  have to learn the things. This phenomenon is called \emph{expert
  blind spot}, and while it can be overcome with training, it's part
  of why world-famous researchers are often poor lecturers.

\item
  Densely-connected knowledge graphs are also the basis for
  experts' \emph{fluid representations}, i.e., their ability to switch
  back and forth between different views of a problem
  \cite{bib:petre-expertise}.  For example, when trying to solve a
  problem in mathematics, we might switch between tackling it
  geometrically and representing it as a set of equations to be
  solved.

\item
  Finally, this metaphor also explains why experts are better at
  diagnosis than competent practitioners: more linkages between facts
  makes it easier to reason backward from symptoms to causes. (And
  this in turn is why asking programmers to debug during job
  interviews gives a more accurate impression of their ability than
  asking them to program.)

\end{genumerate}

\begin{callout}{The J Word}{callout:the-j-word}

Experts often betray their blind spot by using the word ``just'' in
explanations, as in, ``Oh, it's easy, you just fire up a new virtual
machine and then you just install these four patches to Ubuntu and
then you just re-write your entire program in a pure functional
language.'' As we discuss later in \secref{sec:motivation}, the J word
(also sometimes called the passive dismissive adjective) should be
banned from classrooms, primarily because using it gives learners the
very clear signal that the instructor thinks their problem is trivial
and that they therefore must be stupid.

\end{callout}

The graph model of knowledge explains why helping learners make
connections is as important as introducing them to facts.  To use
another analogy, the more people you know in a group, the more likely
you are to remain part of that group.  Similarly, the more connections
a fact has to other facts, the more likely the fact is to be
remembered.

\begin{callout}{Repetition vs.\ Deliberate Practice}{callout:repetition-vs-deliberate-practice}

The idea that ten thousand hours of practice will make someone an
expert in some field is widely quoted, but reality is more complex.
Doing exactly the same thing over and over again is much more likely
to solidify bad habits than perfect performance.  What actually works
is \emph{deliberate practice}\footnote{Also sometimes called
\emph{reflective practice}.}, which is doing similar but subtly
different things, paying attention to what works and what doesn't, and
then changing behavior in response to that feedback to get
cumulatively better.

A common progression is for people to go through three stages:

\begin{enumerate}

\item
  They \emph{learn how to do something given feedback from others}.
  For example, they might write an essay about what they did on their
  summer holiday, and get feedback from a teacher telling them how to
  improve it.

\item
  They \emph{learn how to give feedback}.  For example, they might
  write an essay about character development in \emph{The Catcher in
  the Rye}, and get feedback on their critique from a teacher.

\item
  They \emph{apply what they've learned about feedback to themselves}.
  At some point, they start critiquing their own work in real time (or
  nearly so) using the critical skills they've built up in steps 1 and
  2.  Doing this is so much faster than waiting for feedback from
  others that proficiency suddenly starts to take off.

\end{enumerate}

A meta-study conducted in 2014 \cite{bib:macnamara-deliberate} found
that ``{\ldots}deliberate practice explained 26\% of the variance in
performance for games, 21\% for music, 18\% for sports, 4\% for
education, and less than 1\% for professions.'' One explanation for
this variation is that deliberate practice works best when the rules
for evaluating success are very stable, but is less effective when
there are more factors at play (i.e., when it's harder to connect
cause to effect).

\end{callout}

\seclbl{Concept Maps}{sec:concept-maps}

Our tool of choice to represent a knowledge graph (expert or
otherwise) is a \emph{concept map}.  A concept map is simply a picture
of someone's mental model of a domain: facts are bubbles, and
connections are labelled arcs. It is important that they are labelled:
saying ``X and Y are related'' is only helpful if we explain what the
relationship \emph{is}. And yes, one person's fact may be another
person's connection, but one of the benefits of concept mapping is
that it makes those differences explicit.

\begin{callout}{Externalizing Cognition}{callout:externalizing-cognition}

Concept maps are just one way to represent our understanding of a
subject.  For example, Andrew Abela's
\href{http://extremepresentation.typepad.com/files/choosing-a-good-chart-09.pdf}{decision tree}
presents a mental mode of how to choose the right kind of chart for
different kinds of questions and data.  Maps, flowcharts, and
blueprints can also be useful in some contexts.  What each does
is \emph{externalize cognition}, i.e., make thought processes and
mental models visible so that they can be compared, contrasted, and
combined.

\end{callout}

To show what concept maps look like, consider this simple \texttt{for}
loop in Python:

\begin{verbatim}
for letter in "abc":
    print('*' + letter)
\end{verbatim}

\noindent
whose output is:

\begin{verbatim}
*a
*b
*c
\end{verbatim}

\noindent
The three key ``things'' in this loop are shown in
\figrefsub{fig:for-loop-concepts}{a},  but they are only half the
story---and arguably, the less important half.
\figrefsub{fig:for-loop-concepts}{b} shows the \emph{relationships} between
those things.  We can go further and add two more relationships that are
usually (but not always) true as shown in
\figrefsub{fig:for-loop-concepts}{c}.

\figlbl{Concept Maps}{fig:for-loop-concepts}{fig/for-loop-concepts.png}

Concept maps can be used in many ways:

\begin{genumerate}

\item
  Concept maps aid design of a lesson by helping authors figure out
  what they're trying to teach. Crucially, a concept map separates
  content from order: in our experience, people rarely wind up
  teaching things in the order in which they first drew them.

\item
  They also aid communication between lesson designers. Instructors
  with very different ideas of what they're trying to teach are likely
  to pull their learners in different directions. Drawing and sharing
  concept maps isn't guaranteed to prevent this, but it certainly
  helps.

\item
  Concept maps also aid communication with learners. While it's
  possible to give learners a pre-drawn map at the start of a lesson
  for them to annotate, it's better to draw it piece by piece while
  teaching to reinforce the ties between what's in the map and what
  the instructor said. (We will return to this idea when we discuss
  Mayer's work on multimedia learning in \chapref{sec:load}.)

\item
  Concept maps are also a useful for assessment: having learners draw
  concept maps of what they think they just heard shows the instructor
  what was missed and what was mis-understood.  However, reviewing
  learners' concept maps is too time-consuming for use in class, but
  very useful in weekly lectures \emph{once learners are familiar with
  the technique}.  The qualification is necessary because any new way
  of doing things initially slows people down---if a student is trying
  to make sense of basic economics, asking them to figure out how to
  draw their thoughts at the same time is an unfair load.

\end{genumerate}

\begin{callout}{Meetings, Meetings, Meetings}{callout:meetings}

The next time you have a team meeting, give everyone a sheet of paper
and have them spend a few minutes drawing a concept map of the project
you're all working on---separately. On the count of three, have
everyone reveal their concept maps simultaneously. The discussion that
follows everyone's realization of how different their mental models of
the project's aims and organization are is always interesting{\ldots}

\end{callout}

\seclbl{Seven Plus or Minus Two}{sec:seven}

The graph model of knowledge is wrong but useful, but another simple
model has a sound physical basis.  As a rough approximation, human
memory can be divided into two distinct layers. The first is
called \emph{long-term} or \emph{persistent memory}. It is where we
store things like our password, our home address, and what the clown
did at our eighth birthday party that scared us so much. It is
essentially unbounded: barring injury or disease, we will die before
it fills up.  However, it is also slow to access---too slow to help us
handle hungry lions and disgruntled family members.

Evolution has therefore given us a second system
called \emph{short-term} or \emph{working memory}. It is much faster,
but also much smaller: in 1956, Miller estimated that the average
adult's working memory could hold 7$\pm$2 items for a few seconds
before things started to drop out. This is why phone numbers are
typically 7 or 8 digits long: back when phones had dials instead of
keypads, that was the longest string of numbers most adults could
remember accurately for as long as it took the dial to go around and
around. It's also why sports teams tend to have about half a dozen
members, or be broken down into smaller groups (such as the forwards
and backs in rugby).

\begin{callout}{Serial Position Effect}{callout:serial-position}

When we memorize words in a list and are asked to immediately recall
them, the words first presented will have the best chance to be
transferred into long-term memory. On the other hand, the items that
are presented last might still be in short-term memory. These are
referred to as the primacy and recency effects, respectively, and
together they form the \emph{serial position effect}.

\end{callout}

\begin{callout}{Chunking}{callout:chunking}

Our minds can store larger numbers of facts in short-term memory by
creating \emph{chunks}. For example, most of us will remember a word
we read as a single item, rather than as a sequence of letters.
Similarly, the pattern made by five spots on cards or dice is
remembered as a whole rather than as five separate pieces of
information.  Chunks allow us to manage larger problems, but can also
mislead us if we mis-identify something, i.e., see it as something it
isn't.

\end{callout}

7$\pm$2 is probably the most important number in programming. When
someone is trying to write the next line of a program, or understand
what's already there, she needs to keep a bunch of arbitrary facts
straight in her head: what does this variable represent, what value
does it currently hold, etc. If the number of facts grows too large,
her mental model of the program comes crashing down (something we have
all experienced).

7$\pm$2 is also the most important number in teaching. An instructor
cannot push information directly into a learner's long-term
memory. Instead, whatever she presents is first represented in the
learner's short-term memory, and is only transferred to long-term
memory after it has been held there and rehearsed. If we present too
much information too quickly, the new will displace the old before it
has a chance to consolidate in long-term memory.

This is why it's very important to use a technique like concept
mapping a lesson before teaching it - an instructor needs to identify
just how many pieces of separate information will need to be
``stored'' in memory as part of the lesson.

\begin{callout}{Building Concept Maps Together}{callout:building-concept-maps-together}

Concept maps can be used as a classroom discussion exercise. Put
learners in small groups (2-4 people each), give each group some sticky
notes on which a few key concepts are written, and have them build a
concept map on a whiteboard by placing those sticky notes, connecting
them with labelled arcs, and adding any other concepts they think they
need.
\end{callout}

\begin{callout}{What Are We Doing Again?}{callout:what-are-we-doing-again}

Concept maps can also be used to help build a shared understanding of
what a project is trying to accomplish. Everyone independently draws a
concept map to show what they think the project's goals and constraints
are. Those concept maps are then revealed simultaneously. The ensuing
discussion can be\ldots{}vigorous.

\end{callout}

\seclbl{Teaching Practices}{sec:memory-practices}

\subseclbl{One Up, One Down}{sec:memory-up-down}

We frequently ask for summary feedback at the end of each day. The
instructors ask the learners to alternately give one positive and one
negative point about the day, without repeating anything that has
already been said. This requirement forces people to say things they
otherwise might not: once all the ``safe'' feedback has been given,
participants will start saying what they really think.

Minute cards are anonymous; the alternating up-and-down feedback is not.
Each mode has its strengths and weaknesses, and by providing both, we
hope to get the best of both worlds.

\subseclbl{Pair Programming}{sec:memory-pairing}

Pair programming is a good practice in real life, and also a good way
to teach \cite{bib:porter-what-works}. Partners can not only help each
other out during the practical, but can also clarify each other's
misconceptions when the solution is presented, and discuss common
research interests during breaks. To facilitate this, we strongly
prefer flat (dinner-style) seating to banked (theater-style) seating;
this also makes it easier for helpers to reach learners who need
assistance.

When pair programming is used it's important to put \emph{everyone} in
pairs, not just the learners who are struggling, so that no one feels
singled out. It's also useful to have people sit in new places (and
hence pair with different partners) after each coffee or meal break.

\seclbl{Challenges}{sec:memory-challenges}

\begin{challenge}{The Serial Position Effect}{chal:the-serial-position-effect}

Read the following list and try to memorize the items in it: cat,
apple, ball, tree, square, head, house, door, box, car, king, hammer,
milk, fish, book, tape, arrow, flower, key, shoe.

Without looking at the list again, write down as many words from the
list as you can. Compare to other members of the group. What words are
remembered the most?

\href{http://cat.xula.edu/thinker/memory/working/serial}{This website}
implements an interactive version of this exercise.

\end{challenge}

\begin{challenge}{Concept Mapping}{chal:concept-mapping}

Create a hand drawn concept map for something you would teach in five
minutes.  (If possible, do it for the same subject that created a
multiple choice question for earlier.) Trade with a partner, and
critique each other's maps. Do they present concepts or surface
detail? Which of the relationships in your partner's map do you
consider concepts and vice versa?

\end{challenge}
