\chaplbltime{Motivation and Demotivation}{sec:motivation}{20}{30}

In order for learners to step out into new and familiar terrain, they
need encouragement. This chapter discusses typical ways that learners
can be motivated, and more importantly, ways that we can demotivate
them.

People learn best when they care about the topic and believe they can
master it. This presents us with a problem because most people don't
actually want to program: they want to make music or compare changes
to zoning laws with family incomes, and rightly regarding programming
as a tax they have to pay in order to do so. In addition, their early
experiences with programming are often demoralizing, and believing
that something will be hard to learn is a self-fulfilling prophecy.

Imagine a grid whose axes are labelled ``mean time to master'' and
``usefulness once mastered''. Everything that's quick to master, and
immediately useful should be taught first; things in the opposite corner
that are hard to learn and have little near-term application don't
belong in this course.

\figlbl{What to Teach}{fig:what-to-teach}{fig/what-to-teach.png}

\begin{callout}{Actual Time}{callout:actual-time}

Any useful estimate of how long something takes to master must take
into account how frequent failures are and how much time is lost to
them. For example, editing a text file seems like a simple task, but
most graphical editors save things to the user's desktop or home
directory.  If people need to run shell commands on the files they've
edited, a substantial fraction won't be able to navigate to the right
directory without help. If this seems like a small problem to you,
please revisit the discussion of expert blind spot in
\secref{sec:memory}.

\end{callout}

Many of the foundational concepts of computer science, such as
computability, inhabit the ``useful but hard to learn'' corner of the
grid described above. This doesn't mean that they aren't worth
learning, but if our aim is to convince people that they \emph{can}
learn this stuff, and that doing so will help them do more science
faster, they are less compelling than things like automating
repetitive tasks.

We therefore recommend a ``teach most immediately useful first''
approach.  Have learners do something that \emph{they} think is useful
in their daily work within a few minutes of starting each lesson.
This not only motivates them, it also helps build their confidence in
us, so that if it takes longer to get to the payoff of a later topic,
they'll stick with us.

The best-studied use of this idea is the media computation approach
developed by Guzdial and Ericson at Georgia Tech
\cite{bib:guzdial-mediacomp-retrospective}.
Instead of printing ``hello world'' or summing the first ten integers,
their students' first program opens an image, resizes it to create at
thumbnail, and saves the result. This is an \emph{authentic task},
i.e., something that learners believe they would actually do in real
life. It is also \emph{tangible}: if the image comes out the wrong
size, learners have a concrete starting point for debugging.

\begin{callout}{Strategies for Motivating Learners}{callout:strategies-for-motivating-learners}

\emph{How Learning Works} \cite{bib:ambrose-hlw} contains a list of
evidence-based methods to motivate learners.  None of them are
surprising---it's hard to imagine someone saying that we
\emph{shouldn't} identify and reward what we value---but it's
useful to check lessons against these points to make sure they're
doing at least a few of these things.

What's missing from this list is strategies to motivate the
\emph{instructor}. Learners respond to an instructor's enthusiasm, and
instructors need to care about a topic in order to keep teaching it,
particularly when they are volunteers.

\end{callout}

\seclbl{Demotivation}{sec:demotivation}

If you are teaching free-range learners, they are probably already
motivated---if they weren't, they wouldn't be in your classroom. The
challenge is therefore not to demotivate them.  Unfortunately, we can
do this by accident much more easily than you might think.

A most powerful demotivators are \emph{indifference} and
\emph{unfairness}.  If learners believe that the instructor or the
educational system doesn't care about them or the lesson, they won't
care either. And if people believe the class is unfair, they will also
be demotivated, even if it is unfair in their favor (because
consciously or unconsciously they will worry that they will some day
find themselves in the group on the losing end). Finally, a
``holier-than-thou'' or contemptuous attitude from an instructor is a
quick way to alienate a classroom and cause learners to tune out.

Here are some quick ways to demotivate your learners:

\begin{itemize}

\item
  Tell learners they are rubbish because they use Excel and/or Word,
  don't modularize their code, etc.

\item
  Repeatedly make digs about Windows and praise Linux, e.g., say that
  the former is for amateurs.

\item
  Criticize GUI applications (and by implication their users) and
  describe command-line tools as the One True Way.

\item
  Dive into complex or detailed technical discussion with the one or
  two people in the audience who clearly don't actually need to be
  there.

\item
  Pretend to know more than you do. People will actually trust you
  more if you are frank about the limitations of your knowledge, and
  will be more likely to ask questions and seek help.

\item
  Use the J word (``just''). As discussed in \secref{sec:memory}, this
  signals to the learner that the instructor thinks their problem is
  trivial and by extension that they therefore must be stupid for not
  being able to figure it out.

\item
  Feign surprise. Saying things like ``I can't believe you don't know
  X'' or ``you've never heard of Y?'' signals to the learner that they
  do not have some required pre-knowledge of the material you are
  teaching, that they are in the wrong place, and it may prevent them
  from asking questions in the future. (This idea comes from the
  Recurse Center's
  \href{https://www.recurse.com/manual\#sec-environment}{Social Rules}).

\end{itemize}

\begin{callout}{Code of Conduct Revisited}{callout:conduct-revisited}

As noted in \chapref{sec:welcome}, we believe very strongly that
classes should have a Code of Conduct. Its details are important, but
the most important thing about it is that it exists: knowing that we
have rules tells people a great deal about our values and about what
kind of learning experience they can expect.

\end{callout}

\begin{callout}{Never Learn Alone}{callout:never-learn-alone}

One way to support at-risk learners of all kinds is to have people
sign up for workshops in small teams rather than as individuals. If an
entire lab group comes, or if attendees are drawn from the same (or
closely-related) disciplines, everyone in the room will know in
advance that they will be with at least a few people they trust, which
increases the chances of them actually coming. It also helps after the
workshop: if people come with their labmates, they can work together
to implement what they've learned.

\end{callout}

\seclbl{Impostor Syndrome}{sec:impostor-syndrome}

\href{https://en.wikipedia.org/wiki/Impostor\_syndrome}{Impostor
syndrome} is the belief that one is not good enough for a job or
position, that one's achievements are lucky flukes, and an
accompanying fear of being ``found out''. Impostor syndrome seems to
be particularly common among
\href{https://www.usenix.org/blog/impostor-syndrome-proof-yourself-and-your-community}{high
achievers who undertake publicly visible work}.

Academic work is frequently undertaken alone or in small groups but
the results are shared and criticized publicly. In addition, we rarely
see the struggles of others, only their finished work, which can feed
the belief that everyone else finds it easy. Women and minority
groups, who already feel additional pressure to prove themselves in
some settings,
\href{http://www.paulineroseclance.com/pdf/ip\_high\_achieving\_women.pdf}{may
be particularly affected}.

Two ways of dealing with your own impostor syndrome are:

\begin{enumerate}

\item
  Ask for feedback from someone you respect and trust. Ask them for
  their honest thoughts on your strengths and achievements, and commit
  to believing them.

\item
  Look for role models. Who do you know who presents as confident and
  capable? Think about how they conduct themselves. What lessons can
  you learn from them? What habits can you borrow? (Remember, they
  quite possibly also feel as if they are making it up as they go.)

\end{enumerate}

As an instructor, you can help people with their impostor syndrome by
sharing stories of mistakes that you have made or things you struggled
to learn. This reassures the class that it's OK to find topics hard.
Being open with the group makes it easier to build trust and make
students confident to ask questions. (Live coding is great for this:
typos let the class see you're not superhuman.)

You can also emphasize that you want questions: you are not succeeding
as a teacher if no one can follow your class, so you're asking
students for their help to help you learn and improve. Remember, it's
much more important to \emph{be} smart than to \emph{look} smart.

The Ada Initiative has
\href{http://adainitiative.org/continue-our-work/impostor-syndrome-training/}{some
excellent resources} for teaching about and dealing with imposter
syndrome.

\subseclbl{Stereotype Threat}{sec:stereotype-threat}

Reminding people of negative stereotypes, even in subtle ways, makes
them anxious about the risk of confirming those stereotypes, which in
turn reduces their performance. This is called
\emph{\href{https://en.wikipedia.org/wiki/Stereotype\_threat}{stereotype
threat}}, and the clearest examples in computing are gender-related.
Depending on whose numbers you trust, only 12-18\% of programmers are
women, and those figures have actually been getting worse over the
last 20 years. There are many reasons for this (see
\cite{bib:margolis-fisher-clubhouse} and \cite{bib:margolis-shallow}),
and \cite{bib:steele-vivaldi} summarizes what we know about stereotype
threat in general and presents some strategies for mitigating it in
the classroom.

However, while there's lots of evidence that unwelcoming climates
demotivate members of under-represented groups, it's not clear that
stereotype threat is the underlying mechanism. Part of the problem is
that
\href{http://www.europhd.net/html/\_onda02/07/PDF/20th\_lab\_materials/jane/shapiro\_neuberg\_2007.pdf}{the
term has been used in many ways}; another is
\href{https://www.psychologytoday.com/blog/rabble-rouser/201512/is-stereotype-threat-overcooked-overstated-and-oversold}{questions
about the replicability of key studies}. What \emph{is} clear is that we
need to avoid thinking in terms of a deficit model (i.e., we need to
change the members of under-represented groups because they have some
deficit, such as lack of prior experience) and instead use a systems
approach (i.e., we need to change the system because it produces these
disparities).

A great example of how stereotypes work in general was presented in
Patitsas et al's ``Evidence That Computer Science Grades Are Not
Bimodal'' \cite{bib:patitsas}.  This thought-provoking paper showed
that people see evidence for a ``geek gene'' where none exists.  As
the paper's abstract says:

\begin{quote}

  Although it has never been rigourously demonstrated, there is a
  common belief that CS grades are bimodal. We statistically analyzed
  778 distributions of final course grades from a large research
  university, and found only 5.8\% of the distributions passed tests
  of multimodality. We then devised a psychology experiment to
  understand why CS educators believe their grades to be bimodal. We
  showed 53 CS professors a series of histograms displaying ambiguous
  distributions and asked them to categorize the distributions. A
  random half of participants were primed to think about the fact that
  CS grades are commonly thought to be bimodal; these participants
  were more likely to label ambiguous distributions as ``bimodal''.
  Participants were also more likely to label distributions as bimodal
  if they believed that some students are innately predisposed to do
  better at CS. These results suggest that bimodal grades are
  instructional folklore in CS, caused by confirmation bias and
  instructor beliefs about their students.

\end{quote}

It's easy to use language that suggests that some people are natural
programmers and others aren't, but Mark Guzdial has called this belief
\href{http://cacm.acm.org/blogs/blog-cacm/189498-top-10-myths-about-teaching-computer-science/fulltext}{the
biggest myth about teaching computer science}.  

\seclbl{Mindset}{sec:mindset}

Learners can be demotivated in subtler ways as well. For example,
Dweck and others have studied the differences of
\href{https://en.wikipedia.org/wiki/Mindset\#Fixed\_mindset\_and\_growth\_mindset}{fixed
mindset and growth mindset}. If people believe that competence in some
area is intrinsic (i.e., that you either ``have the gene'' for it or you
don't), \emph{everyone} does worse, including the supposedly advantaged.
The reason is that if they don't get it at first, they figure they just
don't have that aptitude, which biases future performance. On the other
hand, if people believe that a skill is learned and can be improved,
they do better on average.

A person's mindset can be shaped by subtle cues. For example, if a
child is told, ``You did a good job, you must be very smart,'' she is
likely to develop a fixed mindset. If on the other hand she is told,
``You did a good job, you must have worked very hard,'' she is likely
to develop a growth mindset, and subsequently achieve more. Studies
have also shown that the simple action of telling learners about the
different mindsets before a course can improve learning outcomes for
the whole group.

As with stereotype threat,
\href{http://www.learningspy.co.uk/psychology/growth-mindset-bollocks/}{there
are concerns} that research on grown mindset has been oversold, or
will be much more difficult to put into practice than its more
enthusiastic advocates have implied.  While some people interpret this
back and forth of claim and counter-claim as evidence than education
research isn't reliable, what it really shows is that anything
involving human subjects is both subtle and difficult.

\seclbl{Accessibility}{sec:accessibility}

Not providing equal access to lessons and exercises is about as
demotivating as it gets. If you look at
\href{http://swcarpentry.github.io/v4/python/flow.html}{the old
Software Carpentry lessons on Python}, for example, the text beside
the slides includes all of the narration---but none of the Python
source code.  Someone using a
\href{https://en.wikipedia.org/wiki/Screen\_reader}{screen reader} would
therefore be able to hear what was being said about the program, but
wouldn't know what the program actually was.

While it may not be possible to accommodate everyone's needs, it
\emph{is} possible to get a good working structure in place without any
specific knowledge of what specific disabilities people might have.
Having at least some accommodations prepared in advance also makes it
clear that hosts and instructors care enough to have thought about
problems in advance, and that any additional concerns are likely to be
addressed.

\begin{callout}{It Helps Everyone}{callout:it-helps-everyone}

\href{https://en.wikipedia.org/wiki/Curb\_cut}{Curb cuts} (the small
sloped ramps joining a sidewalk to the street) were originally created
to make it easier for the physically disabled to move around, but proved
to be equally helpful to people with strollers and grocery carts.
Similarly, steps taken to make lessons more accessible to people with
various disabilities also help everyone else. Proper captioning of
images, for example, doesn't just give screen readers something to say:
it also makes the images more findable by exposing their content to
search engines.

\end{callout}

The first and most important step in making lessons accessible is
to \emph{involve people with disabilities in decision-making}: the
slogan \emph{\href{https://en.wikipedia.org/wiki/Nothing\_About\_Us\_Without\_Us}{nihil
de nobis, sine nobis}} (literally, ``nothing about us, without us'')
predates accessibility rights, but is always the right place to start.
A few other recommendations are:

\begin{itemize}

\item
  \emph{Find out what you need to do.}
  The \href{http://www.w3.org/WAI/training/accessible}{W3C
  Accessibility Initiative's checklist for presentations} is a good
  starting point focused primarily on assisting the visually impaired,
  while Liz Henry's blog post
  about \href{https://modelviewculture.com/pieces/unlocking-the-invisible-elevator-accessibility-at-tech-conferences}{accessibility
  at conferences} has a good checklist for people with mobility
  issues,
  and \href{https://modelviewculture.com/pieces/qa-making-tech-events-accessible-to-the-deaf-community}{this
  interview} with Chad Taylor is a good introduction to issues faced
  by the hearing impaired.

\item
  \emph{Know how well you're doing.} For example, sites
  like \href{http://webaim.org/}{WebAIM} allow you to check how
  accessible your online materials are to visually impaired users.

\item
  \emph{Don't do everything at once.} We don't ask learners in our
  workshops to adopt all our best practices or tools in one go, but
  instead to work things in gradually at whatever rate they can
  manage.  Similarly, try to build in accessibility habits when
  preparing for workshops by adding something new each time.

\item
  \emph{Do the easy things first.} There are plenty of ways to make
  workshops more accessible that are both easy and don't create extra
  cognitive load for anyone: font choices, general text size, checking
  in advance that your room is accessible via an elevator or ramp,
  etc.

\end{itemize}

\seclbl{Inclusivity}{sec:inclusivity}

\emph{Inclusivity} is a policy of including people who might otherwise
be excluded or marginalized. In computing, it means making a positive
effort to be more welcoming to women, people of color, people with
various sexual orientations, the elderly, the physically challenged,
the formerly incarcerated, the economically disadvantaged, and
everyone else who doesn't fit Silicon Valley's white/Asian male
demographic. Lee's paper ``What can I do today to create a more
inclusive community in CS?'' \cite{bib:lee-create-inclusive-community}
is a brief, practical guide to doing that with references to the
research literature. These help learners who belong to one or more
marginalized or excluded groups, but help motivate everyone else as
well; while they are phrased in terms of term-long courses, many can
be applied in our workshops:

\begin{itemize}

\item
  Ask learners to email you before the workshop to explain how they
  believe the training could help them achieve their goals.

\item
  Review notes to make sure they are free from gendered pronouns, that
  they include culturally diverse names, etc.

\item
  Emphasize that what matters is the rate at which they are learning,
  not the advantages or disadvantages they had when they started.

\item
  Encourage pair programming.

\item
  Actively mitigate behavior that some learners may find intimidating,
  e.g., use of jargon or ``questions'' that are actually asked to
  display knowledge.

\end{itemize}

\seclbl{Teaching Practices}{sec:motivation-practices}

\subseclbl{Think-Pair-Share}{sec:think-pair-share}

Think-pair-share is a lightweight technique that helps refine their
ideas and compare them with others'.  Each person starts by thinking
individually about a question or problem and jotting down a few notes.
Participants are then paired to explain their ideas to each another,
and possibly to merge them or select the more interesting ones.
Finally, a few pairs present their ideas to the whole group.

Think-pair-share works because, to paraphrase Oscar Wilde's Lady
Windermere, people often can't know what they're thinking until
they've heard themselves say it.  Pairing gives people new insight
into their own thinking, and forces them to think through and resolve
any gaps or contradictions \emph{before} exposing their ideas to a
larger group.

\seclbl{Challenges}{sec:motivation-challenges}

\begin{challenge}{Authentic Tasks}{chal:authentic-tasks}

Think about something you did this week that uses one or more of the
skills you teach, (e.g., wrote a function, bulk downloaded data, did
some stats in R, forked a repo) and explain how you would use it (or a
simplified version of it) as an exercise or example in class.

Pair up with your neighbor and decide where this exercise fits on a
$2{\times}2$ grid of ``short/longtime to master'' and ``low/high
usefulness''?  In the shared notes, write the task and where it fits
on the grid. As a group, discuss how these relate back to the ``teach
most immediately useful first'' approach.

\end{challenge}

\begin{challenge}{Pick One}{chal:pick-one}

Pick one activity or change in practice from Lee's paper
\cite{bib:lee-create-inclusive-community} that you would like to work
on.  Put a reminder in your calendar three months in the future to
self-check whether you have done something about it.

\end{challenge}

\begin{challenge}{Brainstorming Motivational Strategies}{chal:brainstorming-motivational-strategies}

Think back to a programming course (or any other) that you took in the
past, and identify one thing the instructor did that motivated you.
Pair up with your neighbor and discuss it, and then share the story in
the group notes.

\end{challenge}

\begin{challenge}{Demotivational Experiences}{chal:demotivational-experiences}

Think back to a time when you demotivated a student (or when you were
demotivated as a student). Pair up with your neighbor and discuss what
you could have done differently in the situation, and then share the
story and what could have been done in the group notes.

\end{challenge}

\begin{challenge}{Walk the Route}{chal:walk-the-route}

Find the nearest public transportation drop-off point to your building
and walk from there to your office and then to the nearest washroom,
making notes about things you think would be difficult for someone with
mobility issues. Now borrow a wheelchair and repeat the journey. How
complete was your list of challenges? And did you notice that the first
sentence in this challenge assumed you could actually walk?

\end{challenge}

\begin{challenge}{Who Decides?}{chal:who-decides}

In Littky and Grabelle's
\emph{The Big Picture: Education is Everyone's Business} \cite{bib:littky-big-picture},
Kenneth Wesson wrote, ``If poor inner-city children consistently
outscored children from wealthy suburban homes on standardized tests,
is anyone naive enough to believe that we would still insist on using
these tests as indicators of success?'' What are examples in your own
experience of ``objective'' assessments that reinforce the status quo?

\end{challenge}
