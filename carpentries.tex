\chaplbl{The Carpentries}{s:carpentries}

In becoming an instructor for Software or Data Carpentry, you are also
becoming part of a community of like-minded volunteers. This section
provides some background on both organizations, and on the final steps
toward certification.

\begin{quote}
\subsection{Preparation and Discussion}\label{preparation-and-discussion}

This discussion assumes that trainees have read the operations
guide (which is assigned as overnight homework). Instead of going through this material point by
point, trainers should ask each trainee to add one non-overlapping
question to a list, then go through that list. \{: .callout\}
\end{quote}

\subsection{History}\label{history}

\href{\{\{\%20site.swc\_site\%20\}\}}{Software Carpentry} was co-founded
in 1998 by Brent Gorda and Greg Wilson, who identified a need for best
practices training in research computing. After several iterations, the
current model of two-day workshops with a standard curriculum emerged in
2010-11. After intermediate support from various organizations, it
became an independent non-profit organization called the
\href{\{\{\%20site.swc\_site\%20\}\}/scf/}{Software Carpentry Foundation}
(SCF) in 2015. The SCF is now responsible for all aspects of Software
Carpentry's operations.

\begin{quote}
\subsection{History Lesson}\label{history-lesson}

For more on Software Carpentry's history, and on what we've learned
along the way, see
\href{\{\{\%20site.swc\_site\%20\}\}/scf/history/}{this page} on its
website or the paper
``\href{http://f1000research.com/articles/3-62/v2}{Software Carpentry:
Lessons Learned}''. \{: .callout\}
\end{quote}

In 2013, members of the Software Carpentry community identified a need
for training aimed at computational novices that would teach researchers
how to properly handle their data. This led to the creation of
\href{\{\{\%20site.dc\_site\%20\}\}}{Data Carpentry} under the leadership
of Tracy Teal. While separate, the two organization share many aspects
of their operations, long-term goals, and community structure:

\begin{itemize}
\itemsep1pt\parskip0pt\parsep0pt
\item
  Both focus on computational skills.
\item
  Both run two-day workshops taught by volunteer instructors.
\item
  Both strive to fill gaps in current training for researchers.
\end{itemize}

However, they differ in their content and intended audience. Data
Carpentry workshops focus on best practices surrounding data. Its
learners are not people who want to learn about coding, but rather those
who have a lot of data and don't know what to do with it. Accordingly,
Data Carpentry workshops:

\begin{itemize}
\itemsep1pt\parskip0pt\parsep0pt
\item
  are aimed at pure novices,
\item
  domain-specific, and
\item
  present a full two-day curriculum centered around a single data set.
\end{itemize}

Software Carpentry workshops focus on best practices for software
development and use. Its workshops are:

\begin{itemize}
\itemsep1pt\parskip0pt\parsep0pt
\item
  intended for people who need to program more effectively to solve
  their computational challenges,
\item
  not domain-specific, and
\item
  modular---each Software Carpentry lesson is standalone.
\end{itemize}

\begin{figure}[htbp]
\centering
\includegraphics{../fig/SWCvsDC.png}
\caption{Software Carpentry and Data Carpentry Comparison}
\end{figure}

\subsection{Workshop Operations}\label{workshop-operations}

We have recorded what we've learned about writing workshops in an
\href{\{\{\%20site.swc\_site\%20\}\}/workshops/operations/}{operations
guide} and a set of checklists (linked from that page) that describes
what everyone involved in a workshop is expected to do and why.
Questions, corrections, and additions are \emph{very} welcome.

Since January 2015 we have run bi-weekly debriefing sessions for
instructors who have recently taught workshops. In these, instructors
discuss what they actually did, how it worked, how the lessons they
actually delivered differed from our templates, what problems arose, and
how they were addressed. Summaries are posted on our blog shortly after
each meeting, and eventually added to our operations guide.

\begin{challenge}{How We Do Things}{chal:how-we-do-things}

Go to the
\href{\{\{\%20site.swc\_site\%20\}\}/workshops/operations/}{operations
guide} and read the instructions for a regular instructor and for a
workshop host. What situations might come up that these \emph{don't}
answer?
\end{challenge}

\subsubsection{What Costs What?}\label{what-costs-what}

Quoting the \href{\{\{\%20site.swc\_site\}\}/workshops/request/}{Software
Carpentry workshop request page}:

\begin{quote}
Our instructors are volunteers, and so are not paid for their teaching,
but \textbf{host sites are required to cover travel and accommodation
costs for any instructors visiting from out of town}. The Software
Carpentry Foundation offers three fee schedules for workshops:

\textbf{Self-Organized Workshops: Optional Donation}

Software Carpentry welcomes you to organize and run your own workshop
without administrative assistance from the Software Carpentry Foundation
by optional donation. In order to use the Software Carpentry name and
logo at your event, we only require that you follow our curriculum, have
at least one badged instructor teaching and co-organizing your event,
and let us know that you're organizing a workshop. In order to help
Software Carpentry continue operating and offering workshops around the
world, we ask for (but do not require) a donation, and recommend \$500
USD as a suitable amount.

\textbf{Nonprofit Organization: \$2500}

If you are a not-for-profit, such as a university or government lab, the
Software Carpentry Foundation will organize a workshop for you (not
including instructor travel and accommodation costs) for \$2500 USD.

\textbf{For-Profit Institution: \$10000}

If you are a for-profit institution, such as a company, the Software
Carpentry Foundation will organize a workshop for you (not including
instructor travel and accommodation costs) for \$10,000 USD of which
three quarters is used to underwrite workshops at institutions that
could otherwise not afford them.

We strive to be a global project and support diversity in science. If
you wish to offer a workshop that would further these goals, please
contact us regarding a waiver for the administration fee at the
nonprofit and for-profit scales. Waivers are not required for
self-organized workshops.
\end{quote}

Quoting the \href{\{\{\%20site.dc\_site\}\}/workshops-host/}{Data
Carpentry workshops page}:

\begin{quote}
The cost of hosting a workshop is both the Workshop Administration Fee
and travel expenses for the two instructors.

\textbf{Workshop Administration Fee: \$2500 US}

This is the fee is for non-profit organizations, such as universities
and government labs. If you are a for-profit organization, such as a
company, and are interested in a workshop, please get in touch.

Partial or full waivers for fees will be considered on an as-needed
basis.

\textbf{Travel Expenses for Instructors: \textasciitilde{}\$2000 US}

All instructors are volunteers, but the Host needs to cover their travel
expenses. We work to find local instructors, but suggest that you
estimate about \$2000 for the travel, food and accommodation of the
instructors. The details of how you will reimburse the instructors needs
to be established when the workshop is scheduled.
\end{quote}

\begin{quote}
\subsection{Travel Costs for No-Shows}\label{travel-costs-for-no-shows}

In order to protect its reputation, the SCF must do what it can to
ensure that instructors actually show up for workshops they have agreed
to teach. We therefore require that when instructors agree to teach a
workshop, they also agree to give at least one week's notice if they
will be unable to make it. If they do not, they are required for
reimbursing any non-refundable travel or accommodation costs that the
host may already have incurred on their behalf.

The SCF may waive this requirement in special circumstances, but the
decision to do so rests solely with the Steering Committee. In cases
where the requirement \emph{is} waived, the SCF will reimburse the host
for any expenses incurred. If an instructor is required to reimburse
costs, but refuses to do so, the SCF reserves the right to ban that
person from future Software Carpentry activities.

If an instructor fails to provide adequate notice of withdrawal more
than once, the SCF reserves the right to suspend them from the list of
recommended instructors. \{: .callout\}
\end{quote}

\subsubsection{Materials}\label{materials}

All of Software and Data Carpentry's lessons materials are freely
available under a permissive open license. You may use
them whenever and however you want, provided you cite the original
source.

\begin{quote}
\subsection{What's Core?}\label{whats-core}

Our learners have such a wide spread of prior knowledge that no one
fixed lesson could possibly fit everyone's needs. We have therefore
provided more material than most people will get through most of the
time in order to be (reasonably) sure that we have enough for more
advanced classes. In particular:

\begin{enumerate}
\def\labelenumi{\arabic{enumi}.}
\itemsep1pt\parskip0pt\parsep0pt
\item
  Callouts (like this one) contain material that isn't essential to the
  lesson, and which most instructors will skip.
\item
  Most instructors only give learners one or two exercises per episode;
  the other exercises are there for self-study. \{: .callout\}
\end{enumerate}
\end{quote}

\subsubsection{Using the Names}\label{using-the-names}

However, the names ``Software Carpentry'' and ``Data Carpentry'' and
their respective logos are all trademarked. You may only call a workshop
a Software Carpentry or Data Carpentry workshop if:

\begin{itemize}
\itemsep1pt\parskip0pt\parsep0pt
\item
  it covers the core topics,
\item
  at least one instructor is certified,
\item
  you run our standardized pre- and post-workshop assessments and
  provides us with the results, and
\item
  you send us summary information about attendees (at a minimum, the
  number of people who attended).
\end{itemize}

\subsubsection{Who Can Teach What}\label{who-can-teach-what}

Software Carpentry and Data Carpentry share a single instructor training
program, but instructors must certify separately for each at the end:
see the description of the
instructor checkout procedure for details.

\subsubsection{Setting Up}\label{setting-up}

In order to communicate with learners, and to help us keep track of
who's taught what and where, each workshop's instructors create a
one-page website using \href{\{\{\%20site.workshop\_repo\%20\}\}}{this
template}. Once that has been created, the host or lead instructor sends
its URL to the \href{\{\{\%20site.contact\%20\}\}}{workshop
coordinator}, who adds it to our records. The workshop will show up on
our websites shortly thereafter.

\begin{challenge}{Practice With SWC Infrastructure}{chal:practice-with-swc-infrastructure}

Go to the \href{\{\{\%20site.workshop\_repo\%20\}\}}{workshop template
repository} and follow the directions to create a workshop website using
your local location and today's date.
\end{challenge}

We also have
a small installer for Windows to help people set up their environment,
which is maintained in
\href{https://github.com/swcarpentry/windows-installer}{this GitHub
repository}. This installer runs \emph{after} the installer that puts
Git and Bash on Windows, and does the following:

\begin{itemize}
\itemsep1pt\parskip0pt\parsep0pt
\item
  Installs GNU Make and makes it accessible from msysGit
\item
  Installs nano and makes it accessible from msysGit
\item
  Installs SQLite and makes it accessible from msysGit
\item
  Creates a \textasciitilde{}/nano.rc with links to syntax highlighting
  configurations
\item
  Provides standard nosetests behavior for msysGit
\item
  Adds R's bin directory to the path (if we can find it)
\end{itemize}

Please see the setup instructions in the workshop template for more
details.

\subsection{The Carpentry Community}\label{the-carpentry-community}

There are several hubs of activity for the Software and Data Carpentry
communities:

\begin{itemize}
\itemsep1pt\parskip0pt\parsep0pt
\item
  Our websites are:

  \begin{itemize}
  \itemsep1pt\parskip0pt\parsep0pt
  \item
    \href{\{\{\%20site.swc\_site\%20\}\}}{Software Carpentry}

    \begin{itemize}
    \itemsep1pt\parskip0pt\parsep0pt
    \item
      \href{\{\{\%20site.swc\_site\%20\}\}/blog/}{Blog}
    \item
      \href{\{\{\%20site.swc\_site\%20\}\}/join/}{Get Involved}
    \end{itemize}
  \item
    \href{\{\{\%20site.dc\_site\%20\}\}}{Data Carpentry}

    \begin{itemize}
    \itemsep1pt\parskip0pt\parsep0pt
    \item
      \href{\{\{\%20site.dc\_site\%20\}\}/blog/}{Blog}
    \item
      \href{\{\{\%20site.dc\_site\%20\}\}/involved/}{Get Involved}
    \end{itemize}
  \end{itemize}
\item
  Our lessons are hosted on GitHub; contributions to them and discussion
  of changes happens via GitHub pull requests and issues, and the
  lessons are published using GitHub Pages. More details are given
  below.

  \begin{itemize}
  \itemsep1pt\parskip0pt\parsep0pt
  \item
    \href{https://github.com/datacarpentry}{Data Carpentry on GitHub}
  \item
    \href{https://github.com/swcarpentry}{Software Carpentry on GitHub}
  \end{itemize}
\item
  Both Software and Data Carpentry have public discussion lists that
  host everything from lively discussion on teaching practices to job
  postings and general announcements.
\item
  Data Carpentry also has a
  \href{http://discuss.datacarpentry.org/}{discussion forum}
\item
  And you can find us on Twitter:

  \begin{itemize}
  \itemsep1pt\parskip0pt\parsep0pt
  \item
    \href{https://twitter.com/swcarpentry}{Software Carpentry on
    Twitter}
  \item
    \href{https://twitter.com/datacarpentry}{Data Carpentry on Twitter}
  \end{itemize}
\end{itemize}

\begin{challenge}{Get Connected}{chal:get-connected}

Join our discussion lists, subscribe to our blogs, and follow us on
Twitter.
\end{challenge}

\subsubsection{A Culture of
Contribution}\label{a-culture-of-contribution}

The administration, policies, practices and content of Software
Carpentry and Data Carpentry rest on the shoulders of the communities
that support them. In the same way that we hope to promote a culture of
openness, sharing, and reproducibility in science and research through
training researchers with the tools they need, the Carpentry
organizations themselves aim to be open, collaborative, and based on
best practices. Just as we encourage researchers to use packages and
modules in their code, to create re-usable pieces, we want to draw
together the collective expertise of our teaching community to create
collaborative lessons, share other materials, and improve the lessons
via ``bug fixes'' as we go along.

\subsubsection{Lesson Contribution}\label{lesson-contribution}

The lesson materials for Software and Data Carpentry are hosted on
GitHub
and are developed collaboratively---in 2015 alone, almost 200 people
made contributions to various lessons. Each lesson is in a separate
repository, and consists of narrative lesson material and an associated
directory containing the data or scripts needed in the lesson. This
source material is also then served as a website, using GitHub's
``gh-pages'' feature.

Lesson contribution is managed within the repository using ``issues''
and ``pull requests''. New problems or suggestions can be introduced as
issues, discussed by the community, and addressed via a pull request,
which serves as a ``request'' to make changes, and can also be discussed
before changes are merged.

\subsubsection{Lesson Incubation}\label{lesson-incubation}

Maybe this instructor training has inspired you to go home and write
your own fantastic lesson! If you'd like to model it after the Software
and Data Carpentry lesson format, you can go to
\href{\{\{\%20site.example\_repo\%20\}\}}{this repository} for a template
and instructions.

Writing a new lesson can be a lot of work. While some people have
written new lessons on their own, other people have asked people in the
community to help them. If you think there are other people who would be
interested in your lesson idea, you can email the Software and Data
Carpentry discussion lists to find out if anyone is interested. If so,
one person will typically take the lead and be the lesson's director and
maintainer until it's ready to be taught.

\begin{quote}
\subsection{Many Ways to Contribute}\label{many-ways-to-contribute}

We recognize that the medium of GitHub may be restrictive to those who
wish to contribute to our lessons. We are always searching for ways to
make the process more friendly to all, whether that be contribution
training, or alternative routes to contribution. If you have any ideas
how we might make contribution more contributor-friendly, please let us
know. \{: .callout\}
\end{quote}

\subsubsection{Beyond Lessons}\label{beyond-lessons}

While contribution is frequently seen in terms of contributing to
specific lessons in either organization, there are many, many ways to
contribute and participate in the Software and Data Carpentry
communities.

\begin{itemize}
\itemsep1pt\parskip0pt\parsep0pt
\item
  Contributing to discussion and development of lessons and policies,
  via discussions on Github issues/pull requests or email discuss lists
\item
  Writing blog posts and bringing important ideas/news to the community
\item
  Developing tools
\item
  Hosting and organizing workshops
\item
  Contributing to lessons via raising issues or submitting pull requests
\item
  Leadership and administrative positions, including being a lesson
  maintainer, or serving on a committee.
\end{itemize}

Here are some examples of ways that people have contributed to the
community:

\begin{itemize}
\itemsep1pt\parskip0pt\parsep0pt
\item
  \href{http://lists.software-carpentry.org/pipermail/discuss/2015-October/003396.html}{This
  email thread} is a good example of many instructors chiming in on a
  topic, resulting in a
  \href{http://software-carpentry.org/blog/2015/10/pulling-along-those-behind.html}{blog
  post}, summarizing the discussion.
\item
  \href{https://github.com/swcarpentry/r-novice-gapminder/pull/89}{Discussion
  on a significant pull request}
\end{itemize}

So being part of a friendly, open discussion, is of equal or greater
importance to the community than submitting the perfect lesson change.
The checkout process to
become a fully-fledged instructor will be one way to start connecting to
the community and find which area will allow you to contribute best.

\subsubsection{Governance}\label{governance}

Software Carpentry is a democracy: its seven-member
\href{\{\{\%20site.swc\_site\%20\}\}/scf/}{Steering Committee} is elected
annually by and from its membership, which includes every instructor who
has taught in the two years leading up to the election. The Steering
Committee has final say on all strategic and financial decisions; if you
would like Software Carpentry to take a new direction, or would like to
do more than teach or develop lessons, you are very welcome to put your
name forward as a candidate.

\begin{challenge}{Feedback on Assessment}{chal:feedback-on-assessment}

Go through the pre-assessment questionnaire given to you by your
instructor and critique its questions. (Remember, critiquing means
commenting on positive aspects as well as negative ones.) How long do
you think it will take the average learner to fill it in? How useful do
you think the information it gathers will be to you as an instructor?
How could you improve the questions? What would you add, and what would
you drop to make room?
\end{challenge}
