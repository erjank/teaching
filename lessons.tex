\chaplbl{Lessons}{sec:lessons}{0}{0}

To wrap up this section, we will combine the things we have learned
with a few more ideas to create a reliable, repeatable process for
creating and improving lessons.

Most people design lessons as follows:

\begin{enumerate}

\item
  Someone tells you that you have to teach something you haven't
  thought about in ten years.

\item
  You start writing slides to explain what you know about the subject.

\item
  After two or three weeks, you make up an assignment based more or
  less on what you've taught so far.

\item
  You repeat step 3 several times.

\item
  You stay awake into the wee hours of the morning to create a final
  exam.

\end{enumerate}

There's a better way, but to explain it, we first need to explain how
\emph{\href{https://en.wikipedia.org/wiki/Test-driven\_development}{test-driven
development}} (TDD) is used in software development.  Programmers who
are using TDD don't write software and then (possibly) write tests.
Instead, they write the tests first, then write just enough new
software to make those tests pass, and then clean up a bit.

TDD works because writing tests forces programmers to specify exactly
what they're trying to accomplish and what ``done'' looks like. It's
easy to be vague when using a human language like English or Korean;
it's much harder to be vague in Python or R.

TDD also reduces the risk of endless polishing, and also the risk of
confirmation bias: someone who hasn't written a program is much more
likely to be objective when testing it than its original author, and
someone who hasn't written a program \emph{yet} is more likely to test
it objectively than someone who has just put in several hours of hard
work and really, really wants to be done.

A similar ``backward'' method works very well for lesson design.  This
method is something called \emph{reverse instructional design} or
\emph{understanding by design} (after a book by Wiggins and McTighe
with that name \cite{fixme}).  There are several variations, but the
one we recommend has the following steps:

\begin{enumerate}

\item
  Create learner profiles (discussed below) to figure out who you are
  trying to teach, what they care about, what they already know, and
  what special needs they might have.

\item
  Draw concept maps to describe the mental model you want them to
  construct.

\item
  Create a summative assessment, such as a final exam or performance,
  that will show you whether learning has actually taken place.

\item
  Create formative assessments that will give the learners a chance to
  practice the things they'll be asked to demonstrate in the summative
  assessment, and tell you and them whether they're making progress
  and where they need to focus their work.

\item
  Put the formative assessments in order based on their complexity and
  dependencies.

\item
  Write just enough to get learners from one formative assessment to
  the next.  An actual classroom lesson will typically then consist of
  three or four such episodes, each building toward a short check that
  learners are keeping up.

\end{enumerate}

This method helps to keep teaching focused on its objectives. It also
ensures that learners don't face anything on the final exam that the
course hasn't prepared them for.

\begin{callout}{How and Why to Fake It}{callout:how-and-why-to-fake-it}

One of the most influential papers in the history of software
engineering was Parnas and Clements' ``A Rational Design Process: How
and Why to Fake It''.  In it, the authors pointed out that in real
life we move back and forth between gathering requirements, interface
design, programming, and testing, but when we write up our work it's
important to describe it as if we did these steps one after another so
that other people can retrace our steps. The same is true of lesson
design: while we may change our mind about what we want to teach based
on something that occurs to us while we're writing an MCQ, we want the
notes we leave behind to present things in the order described above.

\end{callout}

\begin{callout}{Teaching to the Test}{callout:teaching-to-the-test}

Reverse instructional design is \emph{not} the same thing as
``teaching to the test''. When using RID, teachers set goals to aid in
lesson design, and may never actually give the final exam that they
wrote. In many school systems, on the other hand, an external
authority defines assessment criteria for all learners, regardless of
their individual situations, and the outcomes of those summative
assessments directly affect the teachers' pay and promotion.
Green's \emph{Building a Better Teacher} argues that this focus on
measurement is appealing to those with the power to set the tests, but
is unlikely to improve outcomes unless it is coupled with support for
teachers to make improvements based on test outcomes \cite{fixme}.
This is often missing, because as Scott pointed out in \emph{Seeing
Like a State}, large organizations usually value uniformity over
productivity \cite{fixme}.

\end{callout}

\seclbl{Learner Profiles}{sec:learner-profiles}

The first piece - your audience, can be identified in many ways.
Frequently people who are hosting a workshop have a specific audience in
mind, based on their own experience.

One ``creative'' way to characterize the audience for a course is to
write \emph{learner profiles}. This technique is borrowed from user
interface design, where short profiles of typical users are created to
help designers think about their audience's needs, and to give them a
shorthand for talking about specific cases.

Learner profiles have three parts: the person's general background, the
problem they face, and how the course will help them. A learner profile
for Software Carpentry might be:

\begin{quote}
Jo\~{a}o is an agricultural engineer doing his masters in soil physics. His
programming experience is a first year programming course using C. He
was never able to use this low-level programming into his activities,
and never programmed after the first year.

His work consists of evaluating physical properties of soil samples from
different conditions. Some of the soil properties are measured by an
automated device that sends logs in a text format to his machine. Jo\~{a}o
has to open each file in Excel, crop the first and last quarters of
points, and calculate an average.

Software Carpentry will show Jo\~{a}o how to write shell scripts to count
the lines and crop the right range for each file, and how to use R to
read these files and calculate the required statistics. It will also
show him how to put his programs and files under version control so that
he can re-run analyses and figure out which results may have been
affected by changes.
\end{quote}

\seclbl{Writing Learning Objectives}{sec:writing-learning-objectives}

Summative and formative assessments help instructors figure out what
they're going to teach, but in order to communicate that to learners and
other instructors, we should also write \emph{learning objectives}. A
learning objective is a single sentence describing what a learner will
be able to do once they have sat through the lesson, in order to
demonstrate ``learning.'' That requires thinking critically about what
exactly you want people to learn.

It's dangerously easy to come up with fuzzy learning objectives. A broad
statement like ``Understand git'' could mean many different specific
goals, like:

\begin{itemize}
\item
  Learners can revert a change to a file using git.
\item
  Learners will name three benefits of using a version control system
  like git.
\item
  Learners will compare the collaboration features of git and dropbox.
\end{itemize}

What we want are specific, verifiable descriptions of what learners can
do to demonstrate their learning. Each learning objective should have
\emph{a measurable or verifiable verb} specifying what the learner will
do, and should specify the \emph{criteria for acceptable performance}.

Writing these kinds of learning objectives may seem restrictive or
limiting, but will make both you and your learners happier in the long
run. You will end up with clear guidelines for both your teaching and
assessment, and your learners will appreciate the clear expectations.

In order to formulate good learning objectives we need to decide what
kinds of learning we are aiming for. There is a difference between
knowing the atomic weight of fluorine and understanding what elements
it's likely to bond with and why. Similarly, there's a difference
between being able to figure out why a microscope isn't focusing
properly and being able to design a new microscope that focuses more
easily. What we need is a taxonomy of understanding that is
hierarchical, measurable, stable, and cross-cultural.

The best-known attempt to build one is
\href{https://en.wikipedia.org/wiki/Bloom's\_taxonomy}{Bloom's taxonomy},
which was first published in 1956. More recent efforts are Wiggins and
McTighe's \emph{facets of understanding} \cite{bib:wiggins-mctighe} and Fink's taxonomy
\cite{bib:fink-csle}. \tblref{tbl:bloom}
shows some of the verbs typically used in learning objectives written
for each of Bloom et al's original levels.

\begin{tbllbl}{Bloom's Taxonomy}{tbl:bloom}

\begin{enumerate}

\item Knowledge: recalling learned information

\item Comprehension: explaining the meaning of information

\item Application: applying what one knows to novel, concrete situations

\item Analysis: breaking down a whole into its component parts and explaining how each part contributes to the whole

\item Synthesis: assembling components to form a new and integrated whole

\item Evaluation: using evidence to make judgments about the relative merits of ideas and materials

\end{enumerate}

\end{tbllbl}

\fixme{What needs to be in a good learning objective?}
Here is an example of a successively-improved lesson objective:

\begin{tabular}{ll}

Learner will be given opportunities to learn good programming practices.
&
Describes the lesson's content, not the attributes of successful students.
\\

Learner will have a better appreciation for good programming practices.
&
Doesn't start with an active verb or define the level of learning,
and the subject of learning has no context and is not specific.
\\

Learner will understand principles of good programming.
&
Starts with an active verb, but doesn't define the level of learning,
and the subject of learning is still too vague for assessment.
\\

Learner will write one-page data analysis scripts for research purposes using R Studio.
&
Starts with an active verb, defines the level of learning,
and provides context to ensure that outcomes can be assessed.
\\

\end{tabular}

Baume's guide to
writing and using good learning outcomes \cite{bib:baume-fixme} is a good longer discussion of these
issues.

\begin{challenge}{Evaluate SWC and DC Learning Objectives}{chal:evaluate-swc-and-dc-learning-objectives}

Your instructor has posted links to a handful of current Software and
Data Carpentry lessons in the Etherpad. Take a minute to select one
learning objective from one of those lessons, then complete the
following steps to evaluate it and reword it to make it sharper.

\begin{enumerate}
\item
  Identify the learning objective verb.
\item
  Decide what type of learning outcome this applies to
  (i.e.~comprehension, application, evaluation).
\item
  Reword the learning objective for a different learning outcome (e.g.,
  from application to knowledge based outcome or vice versa).
\item
  Pair up to discuss your rewording or help each other with point 3 or 4
  if necessary.
\item
  Share the original and your re-worded learning objectives in the Etherpad.
\end{enumerate}

\end{challenge}

\begin{challenge}{Learner Profiles}{chal:learner-profiles}

Read \href{http://software-carpentry.org/audience/}{Software Carpentry's
learner profiles} and then write one that describes a fictional
colleague of your own. Who are they, what problems do they face, and how
will this training help them? Try to be as specific as possible.
\end{challenge}
